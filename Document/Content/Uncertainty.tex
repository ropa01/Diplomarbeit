\chapter{Uncertainty in process modeling}
\label{chp:uncertainty}
For many real life decisions uncertainty plays a major role. As one might react differently to various 
scenarios that might occur in the future. In terms of process design two types of uncertainties can be 
distinguished. \emph{Internal} and \emph{external} uncertainties., where the internal ones refer to 
transfer-coefficients, diffusivities, efficiencies and other process parameters that cannot be exactly 
predetermined or that might change over time. External uncertainties denote all aspects that are not directly 
part of the process such as future prices for raw materials, energy, or the marketed product. In addition 
feed composition, pressures or ambient conditions are likely to change during process operations. 
All these factors will have an impact on operations and profitability of a given process. 

The aim of the process design is to ensure feasibility for all possible realizations of the uncertain 
parameters. The naive way to account for such uncertainties is to find an optimal solution for nominal 
values of the uncertain parameters and then employ heuristic over design in the hope of maintaining 
feasibility and remain close to an optimal solution. However in practice this strategy will almost always 
fail to deliver an nearly optimal design \cite{Halemane.1983}. 

Therefore several approaches have been developed to incorporate an rational approach to include
flexibility in the design process. Ideally the design optimization would simultaneously recognize 
\emph{feasibility}, \emph{controllability}, \emph{reliability} and \emph{safety} in an multi-objective
optimization. Although all the aspects stem from similar principles they differ considerably in their 
manifestation. While feasibility describes the mere possibility to operate a process under the defined
conditions, controllability denotes wether those conditions can be reached from an initial starting point 
and gives a measure for quality and stability of the process in connection with the dynamic response to 
process disturbances. 

Reliability and safety are qualities of a process that describe the consequences of unit failure on 
the operations of the process or the severity of those consequences with regard to how harmful they 
might be for personnel, environment or wildlife. 

While all these aspects should be considered during the design phase it is questionable how well
they all can be incorporated into a singe optimization scheme. 

\section{Chance contrained optimization}

\section{Recourse models}


