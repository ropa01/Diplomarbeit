\chapter{Mathematical process model}
\label{sec:mathprocess}
    \begin{figure}
        \center
        \footnotesize
        \input{Pictures/ASU_general_color}
        \caption{simplified cryogenic air separation process.}
        \label{fig:asu_simple}
    \end{figure}

    This chapter will illustrate different process units and their mathematical representation.
    As depicted in \figref{fig:asu_simple} three main tasks need to be performed within the cryogenic air separation process.
    These are pre-purification and compression of raw materials, heat exchange with product streams and separation of streams.
    While those different stages are highly interdependent they will be discussed separately and mathematical
    models for the various process units contained in each stage will be presented.

    The first section (\secref{sec:mathpro:steady}) deals with a steady state version of the process model. Especially
    for steady state models the issue of initializing the variables in such a way that the solver can converge to a solution
    becomes crucial. With that in mind the strategies developed to initialize in particular the distillation columns will also
    be elaborated upon.

    The second section (\secref{sec:mathpro:dynamic}) is devoted to a dynamic version of the models for some process units.
    In addition to the models themselves several aspects which arise when considering process dynamics, will also be part
    of that section.

    The third section (\secref{sec:mathpro:econ}) of this chapter takes a closer look at how the economics of the process
    can be captured within the framework of process simulation and optimization. Understandably, the issue of economic
    evaluation is closely linked to the minimum sizing of each process unit. As these aspects are for the most
    part treated uniformly for the steady state case and dynamic case -- since only algebraic equations are involved --
    they will be dealt within a separate section.

    All presented models have been implemented in the equation based process simulator \gproms. Certain aspects arising when
    implementing the models will be discussed in the last section of this chapter \secref{sec:mathpro:implementation}.


    \section{Steady - state unit models}
    \label{sec:mathpro:steady}
        \input{Content/Process_steady}

    \section{Dynamic unit models}
    \label{sec:mathpro:dynamic}
        \input{Content/Process_dynamic}

%    \section{Sizing \& costing models}
%    \label{sec:mathpro:econ}
%        \input{Content/Process_econ}

    \section{Thermodynamic models}
        \label{sec:mathpro:thermo}

        Aside from the unit operation models, the behaviour of materials in a process needs to be adequately
        accounted for. This is done by means of so called equations of state (EOS) and excess Gibbs energy
        models. In terms of thermodynamics there are only a limited amount of variables. Namely the pressure,
        density and temperature as well as composition. While equations of state can model a given system in
        the vapour as well as liquid phase, excess Gibbs energy models only account for the behaviour of a liquid
        and need to be used in conjunction with other models for the vapour phase. However they have shown
        considerable better performance for highly non-ideal systems \cite{AndreasPfennig.2003}. As mentioned
        earlier (\secref{sec:comp_liq}) it is essential to accurately capture the non-idealities of air
        in order to capture the liquefaction process. In the case of cryogenic air separation, the Peng-Robinson
        as well as the Benders equation of state have shown satisfactory performance. The Peng-Robinson equation
        was chosen to be used in the presented model
        \Eq{eq:peng_rob}{
            p & = \frac{RT}{V-b} - \frac{a_c \left[1+m\left(1-\sqrt{T_r}\right)\right]^2}{V^2+2bV-b^2} \\
            m & = 0.37464 + 1.54226 \omega - 0.26992 \omega^2 \\
            a_c & = 0.45724 \frac{R^2T_c^2}{p_c} \\
            b & = 0.077796 \frac{RT_c}{p_c} \\
            \omega & = -1 - \log_{10} \, (p_r^{sat})_{T_r = 0.7}
        }
        \ncr{p}{pressure}{Pa}
        \ncr{m}{parameter in Peng-Robinson EOS}{-}
        \ncr{a_c}{parameter in Peng-Robinson EOS}{\frac{m^5}{mol^2 s^2}}
        \ncr{b}{parameter in Peng-Robinson EOS}{\frac{m^3}{mol}}

        However the Peng-Robinson EOS relies on the so called one-fluid theory which models each fluid as pure.
        To model mixtures the pure component parameters have to be ''mixed''
        \Eq{}{
            a & = \sum_{i=1}^C \sum_{j=1}^C y_i y_j a_{ij}, \\
            a_{ij} & = \sqrt{a_i a_j} (1 - k_{ij}), \\
            b & = \sum_{i=1}^C y_i b_i.
        }

        From that EOS numerous relevant properties such as excess enthalpy, fugacity coefficients or densities
        can be calculated. For a list of some relevant equations refer to \secref{app:peng_rob_deriv}.
        \todo{which properties should be included? Or move everything to Appendix?}

    \section{Implementation}
    \label{sec:mathpro:implementation}
        As mentioned the presented models have been implemented in the process simulator \gproms. Although application 
to the cryogenic air separation process served as case where the model would be applied, the aim was to develop 
-- especially in the case of the column models -- a flexible model which could be used for a multitude of 
problems while trying to achieve a reasonable amount of complexity, such that a user mainly used to a pure flowsheeting 
environment should be able to apply the models with relative ease. 

In the following sections first some general aspects of modeling in \gproms will be pointed out and the structure 
of the implemented model discussed in more detail. Subsequently several strategies for initializing the column model 
based on the previously model structure will be presented.  

\subsection{Model structure}
    \begin{figure}
        \center
        \begin{tikzpicture}
    \node (A) [rectdg] at (0,0) {core \\ equations} ;
    \node (B) [rectdg] at (-7,-2.5) {concentration \\ init profiles} ; 
    \node (C) [rectdg] at (-3.5,-2.5) {temperature \\ init profiles} ; 
    \node (D) [rectdg] at (-0,-2.5) {murphee \\ efficiencies} ; 
    \node (E) [rectdg] at (3.5,-2.5) {cost \& \\ sizing} ; 
    \node (F) [rectdg] at (7,-2.5) {hydraulics} ; 
    \node (sec) at (0,-1.25) {} ; 
    \draw [stdline] (A.south) -- (D.north) ;
    \draw [stdline] (B.north) -- (B.north |- sec) -- (F.north |- sec) -- (F.north) ;
    \draw [stdline] (E.north) -- (E.north |- sec) ;
    \draw [stdline] (C.north) -- (C.north |- sec) ;
    \node (E1) [rectch] at (3.75,-4.5) {trayed } ;
    \node (E2) [rectch] at (3.75,-6.5) {structured \\ packing} ;
    \node (F1) [rectch] at (7.25,-4.5) {trayed } ;
    \node (F2) [rectch] at (7.25,-6.5) {structured \\ packing} ;
    \draw [stdline] (2.25,-3.25) -- ++(0,-1.25) -- ++(0.25,0) ;
    \draw [stdline] (2.25,-4.5) -- ++(0,-2) -- ++(0.25,0) ;
    \draw [stdline] (5.75,-3.25) -- ++(0,-1.25) -- ++(0.25,0) ;
    \draw [stdline] (5.75,-4.5) -- ++(0,-2) -- ++(0.25,0) ;
\end{tikzpicture}

        \caption{Hierarchical model structure.}
        \label{fig:mathpro:modelstruct}
    \end{figure}

\subsection{Initialization procedures}

