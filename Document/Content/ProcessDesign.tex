\chapter{Process Design}
\label{chp:proces_design}

Once a particular reaction or production sequence has been developed in a laboratory it needs to be scaled 
up to produce the desired product in quantities large enough to meet market demand to a price that allows 
competative prices. This is where process design comes into play. At the begin of the design process often 
only the raw material as well as the desired products, along with the necessary chemical reactions are 
specified. It is then the task of the process engineer to find a series of production steps which lead to the 
production of this product. Each step denotes a specific function and the connected functions form a process.
 At the end each specific function has to be assigned a specific technical realization. 

Usually there are numerous options available to perform certain tasks and in order to be realized any 
endeavor will need to yield acceptable profits. Different options will therefore have to be compared  from an 
engineering and an economic standpoint alike. For complex processes the amount of possible options 
becomes enormous. Too many to be manually examined and evaluated. 
\todo[inline]{Number of options for separation problem -- ref douglas}

Process Engineers have to rely on heuristics to limit the number of options and identify the most promising 
candidates for further studies. These candidates can then be compared in terms of economic performance. 
In order to do so one has to rely on simulation studies and experiments. As experiments and the construction 
of pilot plants are very costly the aim is to minimize those. WIth the rise in available computing power the 
numerical simulation of processes has become a very powerful tool to support the design process and has 
the ability to replace numerous experiments. Not only can possible options be evaluated but also can 
optimal process configurations from a limited pool of options be generated. 

To conduct simulation studies first a process model needs to be created. This models in its most general 
case is a set of partial differential and algebraic equations. When formulating the model great care needs 
to be applied, as any simulation study can only be as accurate as the underlying model. The process 
engineer is faces with the task to find an adequate representation of the physical system which delivers 
meaningful results within an reasonable amount of time. Further enhanced is this problem when the model 
is to be used in model predictive control, where (approximate) solutions of complex systems might need to 
be available within seconds or fractions of a second. 

In the following sections some general considerations on the physical and economic modeling of chemical process will be given, before in the next chapter those will be applied to real-life problem at hand. AS uncertainty 

\section{Process Model}
\label{sec:process_model}

\section{Economic Considerations}
\label{sec:uncertainty}

Aside from question wether a certain process is capable of producing products according to its 
specifications, it needs to be investigated if it does so in an economically viable manner. The modeling 
of process economics is a powerful tool to estimate project profitability. The evaluation of process 
economics has three major aims in the design phase. 

\regitemize{%
	\item{Compare design options with regard to profitability.}
	\item{Economically optimize a given design.}
	\item{Estimate project profitability}
}%

In any case the total cost of the project as well as the cash flow structure will have to be analyzed to supply 
an accurate estimate of the economic conditions. Furthermore an adequate measure to compare and 
analyze a project in economic terms needs to be employed. 

The following sections will first describe how to estimate the total project cost in different stages of the 
design process. Subsequently different ways of measuring a projects profitability will be discussed. 

\subsection{Project cost}
An important factor in every given project is the total cost. During the design of a chemical process many 
important aspects of the future cost structure are unknown, as the final design is in development. In 
general the total cost of implementing and operating a production site can be broken down into several 
subcategories. 

\regitemize{%
	\item{Battery limit investment}
	\item{Utility investment}
	\item{Off-Site investment}
	\item{Engineering fees}
	\item{Working capital}
}%

We will now look closer at each of these subcategories. 

\subsubsection{Battery limit investment} 
The battery limit investment denotes all investments necessary to have all required equipment 
for process operations installed on-site. This includes structures necessary to house the process as well as
delivery and installation of all individual assets.  One major part of these cost will the the process 
equipment. As the exact manufacturers and models of the equipment will not be known in early design 
stages, a different approach has to be chosen. Various process parameters such as temperatures, 
pressures and handled volumes will have an effect of the required equipment. High pressure tanks will 
most likely be more expensive as stronger materials and thicker walls will be necessary. Each piece of 
equipment will have an specific feature that most influences its cost. For vessels and reactors this might be 
volume, while for heat exchangers the required heat exchange area. The equipment price $C_E$ can be 
approximated by a simple power law
\eq{
	C_E = C_B \left( \frac{Q}{Q_B} \right)^M. 
	\label{eq:equip_cost}
}%
\nomenclature{$C_E$}{Equipment cost \nomunit{\$}}
\nomenclature{$C_B$}{Reference equipment cost \nomunit{\$}}
\nomenclature{$Q$}{Equipment specific quantity \nomunit{-}}
\nomenclature{$Q_B$}{Equipment specific reference quantity \nomunit{-}}
\nomenclature{$M$}{Equipment specific factor \nomunit{-}}

A reference price $C_B$ is multiplied by the determining quantity $Q$ normalized to a reference state 
$Q_B$ and raised to the power $M$ specific to each piece of equipment. Reference prices for various 
installations can be obtained from literature. 
\todo[inline]{add reference}
As the overall price structure may change dramatically over time these prices will have to be adjusted to 
current time. To do so several indices have been developed
\eq{
	\frac{C_1}{C_2} = \frac{index_1}{index_2}. 
	\label{eq:index_ratio}
}%
\nomenclature{$C_1$}{Reference equipment cost at time $1$ \nomunit{\$}}
\nomenclature{$C_2$}{Reference equipment cost at time $2$ \nomunit{\$}}
\nomenclature{$index_1$}{Cost index at time $1$ \nomunit{-}}
\nomenclature{$index_2$}{Cost index at time $2$ \nomunit{-}}
While for each index an arbitrary reference year is chosen. Among the most common indices are the 
Marshall \& Swift index (1926: $index = 100$) or the Nellson-Farrar-Index (1946: $index = 100$)

As mentioned before aside from the mere size of the equipment the process conditions will also affect the 
price. The predominant factors to that respect are pressure, temperature and the question wether corrosive 
or reactive media, which will require more resistant materials, will be present. All these effects are
accounted for by factors applied to the base cost of the equipment
\eq{
	C_E = C_B \left( \frac{Q}{Q_B} \right)^M f_M \, f_P \, f_T. 
	\label{eq:equip_cost_mpt}
}%
\nomenclature{$f_M$}{Material selection correction to equipment cost \nomunit{-}}
\nomenclature{$f_P$}{Pressure correction to equipment cost \nomunit{-}}
\nomenclature{$f_T$}{Temperature correction to equipment cost \nomunit{-}}
Where $f_M$ denotes the factor for material selection, $f_P$ adjusts for extreme pressures and 
$f_T$ for temperature. 

In addition to the purchased cost, the costs for installing the process equipment need to be considered. 
The Installation costs include:
\regitemize{%
	\item Installations costs
	\item{Piping ,valves and electrical wiring}
	\item{Control system}
	\item{Structures and foundations}
	\item{Insulation and fire proofing}
	\item{Labour fees}
}%
A word should be said to the cost for the control system. Most obtainable data will most likely refer to 
a decentralized control system, as it has been in use for many years. WIth ever more powerful computers 
a centralized approach, namely model predictive control (MPC) is becoming more relevant. As the 
structure for such a control system may vary significantly from the common designs, the cost factors may 
as well. 
 
\subsubsection{Services}
The utility investments and off-site investments are often referred to as services. Therein included are
all measures necessary to supply the process with the media required for operations. This includes but is
not limited to generation and distribution of energy, steam, process gases. The utility investments  
in this context refer to all investments within the greater production site but out of the battery limits of the 
process. Off-site investments contain everything not contained in the site such as roads, power cables,
communication systems ore waste disposal. All these costs are expressed as fractions of the equipment 
cost at moderate temperature and pressure. This means, when applying these fractions, the factors $f_M$, 
$f_P$ and $f_T$ should not be considered at this point. 

\subsubsection{Working capital}
The working capital includes all investments necessary for process operations. This means raw
materials, payroll, extended credit to customers and so on. In contrast to all other costs the 
working capital can partially be retrieved when the process stops operations.     

\subsubsection{Total investment}
When all contributions to the total investment are considered an estimate for the total price of the 
process can be calculated
\eq{
	C_F = \sum_i \left[ C_{Bi} \left( \frac{Q}{Q_{Bi}} \right)^{Mi} f_{Mi} \, f_{Pi} \, f_{Ti} \right] 
		+ \sum_j f_j \cdot \sum_i C_{Ei}. 
}%
It should be emphasized that in early design stages these calculations will at best yield an order of
magnitude estimate for the expected cost of implementing a chemical process. Most literature sources 
give an accuracy of $\pm 30 \%$  \cite{Peters.2003}. As the poject progresses more and more information 
becomes available an a more accurate estimate can be prepared. Those often rely on actual proposals
from prospective manufacturers and suppliers. 

\subsection{Profitability measures}
The total cost of a project its a very important measure to decide wether to undergo a certain endeavor. 
However in a complex financial system it cannot be taken as the sole factor to compare investment 
alternatives. Different other indices are used to measure the attractiveness of an investment. One 
main destination can be made between different measurements. This is wether the time value of many
is considered. First two measurements -- payback time and return of investment (ROI) -- not considering time 
value will be discussed. By analyzing time dependent cash flows indices can be derived, that yield a more 
realistic view of the economic situation. Out of those the net present value (NPV) as well as the 
discounted cash flow rate of return (DCFRR) are introduced. 

\begin{figure}
	\centering
	\begin{tikzpicture}
	\begin{axis} [
		xlabel={time},
		ylabel=accumulated cash flow,
		width=13cm,
		axis lines=middle,
		axis line style={->},
		xtick={0},
		ytick={0},
		every axis y label/.style={at={(ticklabel cs:0.7)},rotate=90,anchor=near ticklabel},
		%every axis x label/.style={at={(ticklabel cs:1,10pt)},anchor=near ticklabel},
		yscale=0.6
		]
		\addplot [color=ImperialDarkBlue] coordinates {
			(0, 0)
			(4, -1)
		};
		\addplot [color=ImperialDarkBlue] coordinates {
			(0, 0)
			(8, -1.2)
		};
		\addplot [color=ImperialDarkBlue, domain=4:50] {ln(x-2)-1.693};
		\addplot [color=ImperialDarkBlue, domain=8:50] {ln(1.8*(x-6.88889))-1.893};
	\end{axis}
\end{tikzpicture}
	\caption{Accumulated cash flows over project life cycle.}
	\todo[inline]{add ref: Script PE-VT}
\end{figure}

\subsubsection{Payback time}
 Payback time is the time necessary to earn the total investment of the process. It is also often referred to as 
 the break even point. A profitable venture will form that point on begin to make money. A shorter 
 payback time is a measure for an more attractive investment. 

\subsubsection{Return of investment}
The return of investment is defined as 
\eq{
	ROI = \frac{\textnormal{average return / period}}{\textnormal{capital expenditure}} \cdot 100 \%.
	\label{eq:roi}
}%
In terms of capital expenditure (CAPEX) a congruent measure for all compared options needs to be chosen. 
\todo[inline]{CAPEX als total investment oder Buchwert oder ...}

\subsubsection{Net present value}

\subsubsection{Discounted cash flow rate of return}

\section{Uncertainty in Process Modeling}
\label{sec:design_uncertainty}






