\chapter{Process design}
\label{chp:proces_design}

Once a particular reaction or production sequence has been developed in a laboratory it needs to be scaled 
up to produce the desired product in quantities large enough to meet market demand to a price that allows 
competative prices. This is where process design comes into play. At the begin of the design process often 
only the raw material as well as the desired products, along with the necessary chemical reactions are 
specified. It is then the task of the process engineer to find a series of production steps which lead to the 
production of this product. Each step denotes a specific function and the connected functions form a process.
 At the end each specific function has to be assigned a specific technical realization. 

Usually there are numerous options available to perform certain tasks and in order to be realized any 
endeavor will need to yield acceptable profits. Different options will therefore have to be compared  from an 
engineering and an economic standpoint alike. For complex processes the amount of possible options 
becomes enormous. Too many to be manually examined and evaluated. 
\todo[inline]{Number of options for separation problem -- ref douglas}

Process Engineers have to rely on heuristics to limit the number of options and identify the most promising 
candidates for further studies. These candidates can then be compared in terms of economic performance. 
In order to do so one has to rely on simulation studies and experiments. As experiments and the construction 
of pilot plants are very costly the aim is to minimize those. WIth the rise in available computing power the 
numerical simulation of processes has become a very powerful tool to support the design process and has 
the ability to replace numerous experiments. Not only can possible options be evaluated but also can 
optimal process configurations from a limited pool of options be generated. 

To conduct simulation studies first a process model needs to be created. This models in its most general 
case is a set of partial differential and algebraic equations. When formulating the model great care needs 
to be applied, as any simulation study can only be as accurate as the underlying model. The process 
engineer is faces with the task to find an adequate representation of the physical system which delivers 
meaningful results within an reasonable amount of time. Further enhanced is this problem when the model 
is to be used in model predictive control, where (approximate) solutions of complex systems might need to 
be available within seconds or fractions of a second. 


\section{Process model}
\label{sec:process_model}

