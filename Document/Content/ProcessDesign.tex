\chapter{Process design under uncertainty}
\label{chp:ProcessDesign}

%Once a particular reaction or production sequence has been developed in a laboratory it needs to be scaled 
%up to produce the desired product in quantities large enough to meet market demand to a price that allows 
%competative prices. This is where process design comes into play. At the begin of the design process often 
%only the raw material as well as the desired products, along with the necessary chemical reactions are 
%specified. It is then the task of the process engineer to find a series of production steps which lead to the 
%production of this product. Each step denotes a specific function and the connected functions form a process.
% At the end each specific function has to be assigned a specific technical realization. 

%Usually there are numerous options available to perform certain tasks and in order to be realized any 
%endeavor will need to yield acceptable profits. Different options will therefore have to be compared  from an 
%engineering and an economic standpoint alike. For complex processes the amount of possible options 
%becomes enormous. Too many to be manually examined and evaluated. 
%\todo[inline]{Number of options for separation problem -- ref douglas}

%Process Engineers have to rely on heuristics to limit the number of options and identify the most promising 
%candidates for further studies. These candidates can then be compared in terms of economic performance. 
%In order to do so one has to rely on simulation studies and experiments. As experiments and the construction 
%of pilot plants are very costly the aim is to minimize those. WIth the rise in available computing power the 
%numerical simulation of processes has become a very powerful tool to support the design process and has 
%the ability to replace numerous experiments. Not only can possible options be evaluated but also can 
%optimal process configurations from a limited pool of options be generated. 

%To conduct simulation studies first a process model needs to be created. This models in its most general 
%case is a set of partial differential and algebraic equations. When formulating the model great care needs 
%to be applied, as any simulation study can only be as accurate as the underlying model. The process 
%engineer is faces with the task to find an adequate representation of the physical system which delivers 
%meaningful results within an reasonable amount of time. Further enhanced is this problem when the model 
%is to be used in model predictive control, where (approximate) solutions of complex systems might need to 
%be available within seconds or fractions of a second. 

For many real life decisions uncertainty plays a major role. As one might react differently to various 
scenarios that might occur in the future. In terms of process design two types of uncertainties can be 
distinguished. \emph{Internal} and \emph{external} uncertainties., where the internal ones refer to 
transfer-coefficients, diffusivities, efficiencies and other process parameters that cannot be exactly 
predetermined or that might change over time. External uncertainties denote all aspects that are not directly 
part of the process such as future prices for raw materials, energy, or the marketed product. In addition 
feed composition, pressures or ambient conditions are likely to change during process operations. 
All these factors will have an impact on operations and profitability of a given process. 

The aim of the process design is to ensure feasibility for all possible realizations of the uncertain 
parameters. The naive way to account for such uncertainties is to find an optimal solution for nominal 
values of the uncertain parameters and then employ heuristic over design in the hope of maintaining 
feasibility and remain close to an optimal solution. However in practice this strategy will almost always 
fail to deliver an nearly optimal design \cite{Halemane.1983}. 

Therefore several approaches have been developed to incorporate an rational approach to include
flexibility in the design process. Ideally the design optimization would simultaneously recognize 
\emph{feasibility}, \emph{controllability}, \emph{reliability} and \emph{safety} in an multi-objective
optimization. Although all the aspects stem from similar principles they differ considerably in their 
manifestation. While feasibility describes the mere possibility to operate a process under the defined
conditions, controllability denotes wether those conditions can be reached from an initial starting point 
and gives a measure for quality and stability of the process in connection with the dynamic response to 
process disturbances. 

Reliability and safety are qualities of a process that describe the consequences of unit failure on 
the operations of the process or the severity of those consequences with regard to how harmful they 
might be for personnel, environment or wildlife. 

While all these aspects should be considered during the design phase it is questionable how well
they all can be incorporated into a singe optimization scheme. 

\section{Earlier work}
\label{sec:des:EarlierWork}
In its most general form the design problem for chemical processes can be written as:
\program{prog:generic}{
&\minimize{d, z} & & C(d, z, x, \theta) \\
&\st & & h(d, z, x, \theta) = 0 \\
&&& g(d, z, x, \theta) \leq 0 
}
Where $d$, $z$ are the vectors for design and control variables respectively. The vector $x$ contains
all states specified by the physical process model and $\theta$ is comprised of all parameters subject
to uncertainty. The aim of the program is therefore to identify the optimal set of design variables $d$
for which by manipulating the control variables $z$ feasible operations can be ensured for all possible 
realizations o the uncertain parameters $\theta$. 

Much research has been done within the field of process design under uncertainty that meet the 
requirements of optimal process design to different degrees. Most frequently used was the stochastic 
approach. If the probability distribution functions of all uncertain values is available, different minimization 
schemes can be employed. Kittrel and Watson \cite{} \todo{add ref Kittrel and Watson '66 from flexibility}
thought the optimal design to be the one which minimizes the expected value of cost. Wen and Chang 
\cite{} \todo{add ref Wen and Chang from flexibility} proposed a \emph{relative sensitivity} of the cost -- 
defined as fractional change in the cost function from its nominal value -- and minimized either the expected 
value or the maximum probable value of this factional change. Other researcher included penalty functions 
to drive a solution away from constraint violations and then perform an unconstrained minimization or 
employ Monte Carlo simulation to find suitable over-design factors \cite{} \todo[inline]{add ref: flexibility}. 

All the aforementioned approaches do not account for a fact derived from the actual application of the 
problem. In chemical processes, once a certain design is determined by the design variables $d$, the 
control variables $z$ can still be adjusted to meet feasible operations once the uncertain parameters 
are realized. It is therefore advantageous to include this basic difference between the design and 
control variables in the mathematical formulation of the program. Again several different  approaches 
have been studied. A summary of some of those can be found in \cite{}\todo[inline]{add ref: flexibility}.

As mentioned before the main aim in terms of an design engineer when considering uncertainty must 
be to ensure feasible operations of a process for all or or the most likely realization of the uncertain 
parameters with the most attractive investment scheme. 

\stdfig{pgfplots/HardSoftConstraint}{test. \cite{Bernardo.2001}}{test}{}














