The simulation of chemical processes supplies valuable insight into the

    \subsection{Degrees of freedom}
    In order to gain more insight into the behaviour of the dynamic models presented above an analysis
    of the degrees of freedom within the model is at hand. For the degrees of freedom the cost correlations
    will be disregarded, as they for a closed system of equations given the inputs generated form the column
    model. Furthermore interdependencies can be disregarded, as the cost model consist only of ''forward''
    computations. In practical terms this statement can be confirmed since the models can be evaluated
    with and without costing equations. Hence only the stage and hydraulic equations will be considered.

    For a given column without condenser or reboiler the model is made up of $[n_S (5n_C + 24) + n_F]$
    differential algebraic equations in $[n_S (5n_C + 29) + n_F (n_S + n_C + 3) + 1]$ variables. In this
    isolated case all feed flow rates, their composition and enthalpies would be specified. In this case
    the feeds include a hypothetical condenser reflux (the upper most feed) as well as reboiler reflux
    (lowest feed). Along with the feeds and their qualities, the feed splits and reflux split have to be
    assigned. Lastly the column diameter needs to be known. This yields $[n_F (n_S + n_C + 2) + n_S + 1]$
    specifications. To close the system initial conditions for all states have to be given. There
    are a total of $[4 n_S ]$ dynamic stages in a column section.

    The condenser reboiler unit

    \subsection{Steady state single period}
    \label{chp:optexample:ss_single_perid}
    Objective function
    \Eqml{eq:opt:ss_single_objective}{
        CAPEX = \left( \sum_c C_c^{\text{cap}} \right) \cdot \left(q^{-a} \frac{q^a - 1}{q - 1} \right) + \sum_o C_o^{\text{oper}} \\
            \eqannote{c = \left\{ \text{HPC}, \text{LPC}, \text{CAC}, \text{HX}, \text{CP}, \text{CRM}, \text{CRCAC} \right\}
                \quad o = \left\{\text{CP}, \text{EXP} \right\}}
    }
    Constraints :
    Limits on the product purities:
    \Eq{eq:opt:ss_single_constraints_1}{
        y_{1,N_2}^{HPC} & \geq 0.985 \\
        y_{1,N_2}^{LPC} & \geq 0.985 \\
        y_{1,Ar}^{CAC} & \geq 0.985 \\
        x_{reb}^{CRM} & \geq 0.985
    }
    No flooding in the columns:
    \Eq{eq:opt:ss_single_constraints_2}{
        d_{\text{column}}^{\text{HPC}} & \geq d_{min}^{HPC} \\
        d_{\text{column}}^{\text{LPC}} & \geq d_{min}^{LPC} \\
        d_{\text{column}}^{\text{CAC}} & \geq d_{min}^{CAC}
    }
    No entrainment in the trayed column :
    \Eq{eq:opt:ss_single_constraints_3}{
        \left(1- \sum_{k=1}^{j-1} \right) ent_k^{HPC} & \leq 0.1
    }
    Limit on cooling water outlet temperatures to prevent corrosion :
    \Eq{eq:opt:ss_single_constraints_4}{
        T_{\text{w,out}}^{IC1} & \leq 323.15 \\
        T_{\text{w,out}}^{IC2} & \leq 323.15 \\
        T_{\text{w,out}}^{IC3} & \leq 323.15
    }

    Design Variables
    \begin{itemize}
        \item HPC diameter $d_{\text{column}}^{\text{HPC}}$
        \item LPC diameter $d_{\text{column}}^{\text{LPC}}$
        \item CAC diameter $d_{\text{column}}^{\text{CAC}}$
        \item HPC reflux location $\zeta^R_{\text{HPC}}$
        \item LPC reflux location $\zeta^R_{\text{LPC}}$
        \item CAC reflux location $\zeta^R_{\text{CAC}}$
        \item LPC CAC side draw location $\zeta^{SV,\text{LPC}}_{2,j}$
        \item heat exchange area multi-stream heat exchanger $A_{\text{HX}}^{multiHX}$
        \item heat exchange area main condenser reboiler $A_{\text{HX}}^{CRM}$
        \item heat exchange area CAC condenser reboiler $A_{\text{HX}}^{CRCAC}$
    \end{itemize}

    Manipulated Variables
    \begin{itemize}
        \item intercooler outlet temperatures ($T_{\text{out}}^{IC1}, T_{\text{out}}^{IC2}, T_{\text{out}}^{IC3}$)
        \item HPC dimensionless side draw (gaseous $N_2$product) ($s_1^{V, \text{HPC}}$)
        \item LPC dimensionless side draws ($s_i^{V, \text{LPC}}$)
    \end{itemize}

\subsection{Optimization \& control}
	To include the design of a PI control structure in the process, the following constraints need to be added.
	\Eq{}{
		u_m & = b_{m,1} + \sum_{j=1}^{n_m} \left( K_{m,n} \cdot e_{m,n} + I_{m,n} \right),
			\eqannote{m = 1 \dots n_{in}}, \\
		e_{m,n} & = set_{m,n} - meas_{n} \eqannote{m = 1 \dots n_{in}, n = 1 \dots n_m}, \\
		\frac{\mathrm{d} I}{\mathrm{d} t} & = \frac{e_{m,n}}{\tau_{m.n}} \eqannote{m = 1
			\dots n_{in}, n = 1 \dots n_m}, \\
		\frac{\mathrm{d} I}{\mathrm{d} t} & = 0 \eqannote{m = 1 \dots n_{in}, n = 1 \dots n_m}, \\
		set_{m,n} & = set_{m-1,n} \eqannote{m = 2 \dots n_{in}, n = 1 \dots n_m}, \\
		K^L_{m,n} \cdot \zeta^C_{m,n} & \leq K_{m,n}  \leq K^U_{m,n} \eqannote{m = 1 \dots n_{in},
			n = 1 \dots n_m}, \\
		\sum_{m=1}^{n_m} \zeta_{m,n}^C & = 1 \eqannote{m = 1 \dots n_{in}}, \\
		\sum_{n=1}^{n_{in}} \zeta_{m,n}^C & = 1 \eqannote{n = 1 \dots n_{m}},
	}
	Since new states have been introduced, the corresponding initial conditions will have to be included
	\Eq{}{
		I_{m,n} (t = 0) & = 0, \eqannote{m = 1 \dots n_{in}, n = 1 \dots n_m} \\
		e_{m,n} (t = 0) & = 0, \eqannote{m = 1 \dots n_{in}, n = 1 \dots n_m} \\
	}
