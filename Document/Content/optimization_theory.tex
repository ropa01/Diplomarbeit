The purpose of this section is not an exhaustive review of (convex) optimization but rather to mention
some elementary concepts and terms that will prove useful in the following sections. Initially optimality
conditions for constrained problems -- the so called Karush-Kuhn-Tucker (KKT)
conditions -- are briefly discussed. Closely linked to those conditions are the constraint qualifications, 
which are subsequently discussed.

    \subsection{Karush-Kuhn-Tucker conditions}
    \label{sec:opt:theory:kkt}
    This formulation of the KKT conditions is adapted from explanations in \cite{Nocedal.2006}.
    Before the conditions ca be stated, some definitions are needed. Considering the most general case of a NLP
    \program{}{
        & \minimize{x} & & f(x) \\
        & \st & & h_i(x) = 0, \; \; i \in \mathset{E} \\
        & & & g_i(x) \leq 0, \; \; i \in \mathset{I}
    }
    where $f(x): \mathbb{R}^n \mapsto \mathbb{R}$ denotes the objective function, $\mathset{E}$ and $\mathset{I}$
    are sets of indices corresponding to equality and inequality constraints respectively.
    the functions $f, g$ and $h$ are assumed to be differentiable and real-valued.

    The active set denotes the union of the set of equality constraints and the inequality constraints
    where $g_i(x) = 0$ holds. Thus we can write for the active set at any feasible point $x$
    \Eq{}{
       \mathset{A}(x) = \mathset{E} \cup \left\{ i \in \mathset{I} | g_i(x) = 0 \right\}.
    }

    For the constrained problem stated above the Lagrangian is defined as
    \Eq{}{
        \mathset{L}(x,\lambda,\mu) := f(x) - \sum_{i \in \mathset{E}} \lambda_i h_i(x) - \sum_{i \in \mathcal{I}} \mu_i g_i(x)
    }

    With that the optimality conditions for constrained optimization can be stated.
    First order or necessary KKT conditions must hold at the optimal point $(x^{\ast}, \lambda^{\ast}, \mu^{\ast})$
    \Eq{}{
        \nabla_x \mathcal{L}(x^{\ast}, \lambda^{\ast}, \mu^{\ast}) & = 0, \\
        h_i(x^{\ast}) & = 0, & \forall i \in \mathset{E} \\
        g_i(x^{\ast}) & \leq 0, & \forall i \in \mathset{I} \\
        \mu_i^{\ast} & \geq 0, & \forall i \in \mathset{I} \\
        \mu_i^{\ast} g_i(x^{\ast})& = 0, & \forall i \in \mathset{I} \\
        \lambda_i^{\ast} h_i(x^{\ast})& = 0, & \forall i \in \mathset{E}
    }
    The second order or sufficient KKT conditions can be written as
    \todo{check conditions on lagrange multipliers and active set ....}
    \Eq{}{
        w^T \nabla_{xx} \mathcal{L}(x^{\ast}, \lambda^{\ast}, \mu^{\ast})w & > 0, \quad \forall w \neq 0, \\
        \nabla h_i(x^{\ast})^T w & = 0, \quad \forall \; i \in \mathset{E}, \\
        \nabla g_i(x^{\ast})^T w & = 0, \quad \forall \; i \in \mathset{A}(x^{\ast}) \cap \mathset{I}, \; \text{with} \; \mu_i^{\ast} > 0 \\
        \nabla g_i(x^{\ast})^T w & \geq 0, \quad \forall \; i \in \mathset{A}(x^{\ast}) \cap \mathset{I}, \; \text{with} \; \mu_i^{\ast} = 0
    }

    \subsection{Constraint qualifications}
    \label{sec:opt:theory:cq}

    Constraint qualifications are used in the derivation of the optimality conditions. In the form presented here, 
    they are derived from the so call linear-independence constraint qualification. The derivation of 
    optimality conditions relies on approximating the objective function as well as the constraints by a 
    first order Taylor expansion. However there is no insurance that this linear model is in fact 
    an adequate representation of the original model around any given point. To resolve this issue, constraint 
    qualifications are used, to ensure, that the approximative model yields a similar representation of the 
    original one \cite{Nocedal.2006}. The significance of constraint qualifications in the context of this thesis, lies in a later 
    presented solution approach to mixed integer non-linear programs. 

    There are several constraint qualifications proposed in literature, which are differently hard to fulfil
    and verify. In general a minimum point should satisfy some constraint qualifications or regularity conditions.
    Among the most common qualifications are Slater's- (SCQ), linear independence- (LICQ) and the Mangasarian-Fromovitz (MFCQ)
    constraint qualification which are state here. Several more constraint qualifications can be found in literature, 
    but will not be stated here.

    \subsubsection{Slater's constraint qualification (SCQ)}
    Among the most widely used constraint qualifications is Slater's constraint qualification
    \Eq{}{
        \exists \tilde{x}, \forall i \in \mathcal{I} : h_i(\tilde{x}) < 0.
    }
    The main advantage of the SCQ is that the actual optimal point needs not to be known for checking
    if the SCQ is fulfilled.

    \subsubsection{Linear independence constraint qualification (LICQ)}
    Another widely used is called the linear independence constraint qualification. It states,
    given the set of of feasible solutions of the original problem $\mathcal{C} = \{x \in \mathcal{X} \; | \; h_i(x) \leq 0
    \forall i \in \mathcal{I} \}$ and the set of active constraints $\mathcal{A}(\bar{x}) = \{i \in \mathcal{i} \; | \; h_i(\bar{x}) = 0 \}$
    \Eq{}{
        \{ \nabla h_i(\bar{x}) \; | \; i \in \mathcal{A}(\bar{x}) \} \qquad \text{is linearly independent}.
    }

    \subsubsection{Mangasarian-Fromovitz constraint qualification (MFCQ)}
    \Eq{}{
        \exists \tilde{u}, \forall i \in \mathcal{A}(\bar{x}) : \nabla h_i(\bar{x})(\tilde{u}) < 0.
    }



