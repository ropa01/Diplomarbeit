This section will deal with some basic considerations about optimization theory. It purpose is not 
an exhaustive review of (convex) optimization but rather to introduce some concepts and terms that will 
prove useful in the following sections. Foremost a brief discussion of optimality conditions for constrained 
problems -- the so called Karush-Kuhn-Tucker conditions -- will be given. Closely linked to those conditions, 
and in some cases even a perquisite for their validity, are the constraint qualifications which are subsequently
discussed. 

    \subsection{Karush-Kuhn-Tucker conditions}
    \label{sec:opt:theory:kkt}
    \todo{introduce active set}
    Considering the most general case of a NLP
    \program{}{
        & \minimize{x} & & f(x) \\
        & \st & & h_i(x) = 0, \; \; i \in \mathset{E} \\
        & & & g_j(x) \leq 0, \; \; j \in \mathset{I}
    }
    where $f(x): \mathbb{R}^n \mapsto \mathbb{R}$ denotes the objective function, $\mathset{E}$ is the set of equality
    constraints and $\mathset{I}$ the set of inequality constraints.

    The Lagrangian function corresponding to to the problem above can be written as
    \Eq{}{
        \mathcal{L}(x,\lambda,\mu) := f(x) + \lambda^T h(x) + \mu^T g(x)
    }

    First order KKT conditions must hold at the optimal point $(x^{\ast}, \lambda^{\ast}, \mu^{\ast})$
    \Eq{}{
        \nabla_x \mathcal{L}(x^{\ast}, \lambda^{\ast}, \mu^{\ast}) & = 0, \\
        h_i(x^{\ast}) & = 0, \\
        g_j(x^{\ast}) & \leq 0, \\
        \mu_j^{\ast} & \geq 0, \\
        \mu_j^{\ast} g_j(x^{\ast})& = 0, \\
        i \in \mathset{E}, \; \; j \in \mathset{I}.
    }

    \subsection{Constraint qualification conditions}
    \label{sec:opt:theory:cq}
    
    Constraint qualifications, given convexity of the problem, can ensure the existence of strictly positive Lagrange 
    multipliers, such that the unconstrained equivalent problem can be constructed. In general one is interested 
    in the weakest possible perquisite to ensure feasibility \todo{feasibility the right word here?} of the problem. 
    There are several constraint qualifications proposed in literature, which are differently hard to fulfil 
    and verify. 

    \subsubsection{Slater's constraint qualification (SCQ)}
    Among the most widely used constraint qualifications is Slater's constraint qualification
    \Eq{}{
        \exists \tilde{x}, \forall i \in \mathcal{I} : h_i(\tilde{x}) < 0. 
    } 
    This qualifications essentially states, that there in fact is a point which will fulfil the constraints. 

    \subsubsection{Linear independence constraint qualification (LICQ)}
    Another widely used is called the linear independence constraint qualification. It states, 
    given the set of of feasible solutions of the original problem $\mathcal{C} = \{x \in \mathcal{X} \; | \; h_i(x) \leq 0
    \forall i \in \mathcal{I} \}$ and the set of active constraints $\mathcal{A}(\bar{x}) = \{i \in \mathcal{i} \; | \; h_i(\bar{x}) = 0 \}$
    \Eq{}{
        \{ \nabla h_i(\bar{x}) \; | \; i \in \mathcal{A}(\bar{x}) \} \qquad \text{is linearly independent}. 
    } 
    
    \subsubsection{Mangasarian-Fromovitz constraint qualification (MFCQ)}
    \Eq{}{
        \exists \tilde{u}, \forall i \in \mathcal{A}(\bar{x}) : \nabla h_i(\bar{x})(\tilde{u}) < 0. 
    }
