\chapter{Mathematical process model}
\label{chp:MathModel}

\section{Steady state model}
\label{sec:SteadyStateModel}

\subsection{Distillation column}

	\stdfig{pgfplots/ColTray}{Distillation column tray model.}{fig:ColTray}{h}

	\paragraph{Mass balance}
	\Eq{eq:col:MassBalance}{
		0 = F_j^V + F_j^L + V_{j+1} + L_{j-1} - V_j - S_j^V - L_j - S_j^L
		\nc{$F_j^V$}{Vapour feed to tray $j$}{\frac{mol}{s}}
		\nc{$F_j^L$}{Liquid feed to tray $j$}{\frac{mol}{s}}
		\nc{$S_j^V$}{Vapour side flow from tray $j$}{\frac{mol}{s}}
		\nc{$S_j^L$}{Liquid side flow from tray $j$}{\frac{mol}{s}}
		\nc{$V_j$}{Vapour flow from tray $j$}{\frac{mol}{s}}
		\nc{$L_j$}{Liquid flow from tray $j$}{\frac{mol}{s}}
	}%

	\paragraph{Component balances}
	\Eq{eq:col:CompBalance}{
		0 = V_{j+1} y_{i,j+1} + L_{j-1} x_{i,j-1} + F_j^V z^V_{i,j} + F_j^L z_{i,j}^L%
			- \left(V_j + S_j^V\right) y_{i,j} - \left(L_j + S_j^L\right) x_{i,j} 
	}%

	\paragraph{Energy balance}
	\Eq{eq:col:EnergyBalance}{
		0 = V_{j+1} H^V_{j+1} + L_{j-1} H^L_{j-1} + F_j^V H^{FV}_{j} + F_j^L H^{FL}_{j}%
			- \left(V_j + S_j^V\right) H^V_{j} - \left(L_j + S_j^L\right) H^L_{j} 
	}%

	\paragraph{Vapour liquid equilibrium}
	\Eq{eq:col:Kxy}{
		y_{ij} = \gamma_{ij} \cdot K_{ij} \cdot x_{ij}
	}%

\subsubsection{Multi stream heat exchanger}

\subsection{Compressor / expander}
	\stdfig{pgfplots/Compressor}{Compressor model.}{fig:Compressor}{h}

	\paragraph{Mass and component balances}
	\Eq{eq:comp:MassBalance}{
		0 = F^{in} - F^{out}
	}%
	\Eq{eq:comp:CompBalance}{
		0 = z_i^{in} - z_i^{out}
	}%

	\paragraph{Isentropic work}
	\Eq{eq:comp:SInOut}{
		S^{in}(T^{in}, p^{in}, z_i^{in}) = S^{out}(T_S^{out}, p^{out}, z_i^{out})
	}%
	\Eq{eq:comp:IsentropicWork}{
		W_S = F^{in} \cdot H^{in}(T^{in}, p^{in}, z_i^{in}) - F^{out} \cdot H^{out}(T_S^{out}, p^{out}, z_i^{out}) 
	}%

	\paragraph{Compression work}
	\Eq{eq:comp:WEta}{
		W \cdot \eta^C = W_S
	}%
	\Eq{eq:comp:CompressionWork}{
		W_S = F^{in} \cdot H^{in}(T^{in}, p^{in}, z_i^{in}) - F^{out} \cdot H^{out}(T^{out}, p^{out}, z_i^{out}) 
	}%

	\paragraph{Pressure drop}
	\Eq{eq:comp:PressureDrop}{
		p^{out} = p^{in} + \Delta p
	}%
	
\subsection{Centrifugal pump}
	
	\paragraph{Cost estimation}
		The cost for the pump excluding the motor can be approximated by
		\Eq{eq:pump:CostPump}{
			C_B & = \exp \left[ 9.7171 - 0.6019 \ln[S] + 0.0519 (\ln[S])^2 \right] \eqannote{40 \leq S \leq 100000}\\
			C_p & = f_T \, f_M \, C_B. 
		}
		The cost for the centrifugal pump is approximated by the value $S = Q \cdot \sqrt{H} $, where
		$Q$ denotes the voulme flow through the pump in gallons per minute $[gpm]$ and $H$ the 
		pump head in $[ft]$. 
		
		The costs given above is excluding the motor needed. It is then estimated separately. In this 
		case the specific quantity is the power consumption $P_C$ needed for the desired stream transport. 
		
		\Eqml{eq:pump:CostMotor}{
			C_B  = \exp\big[ 5.8259 + 0.13141 \ln[P_C] + 0.053255 (\ln[Pc])^2 \\
				 + 0.028628 (\ln[P_C])^3 - 0.0035549 (\ln[P_C])^4 \big] 
		}	

	\todo[inline]{add formulas for vapour fraction}
	\todo[inline]{check if . or , are used for decimals ...}
\subsubsection{Expander}

\subsubsection{Condenser}

\subsection{Reboiler}

	\stdfig{pgfplots/Reboiler}{Reboiler model.}{fig:Reboiler}{h}

	\paragraph{Mass balance}
	\Eq{eq:reboil:MassBalance}{
		0 = F - V - L
	}%

	\paragraph{Component balance}
	\Eq{eq:reboil:CompBalance}{
		0 = F z_j - V y_j - L x_j
	}%

	\paragraph{Pressure drop}
	\Eq{eq:reboil:PressureDrop}{
		p^R = p^{in} + \Delta p
	}%

\subsection{Heater}

	\stdfig{pgfplots/Heater}{Heater model.}{fig:Heater}{h}

	\paragraph{Mass and component balances}
	\Eq{eq:comp:MassBalance}{
		0 = F^{in} - F^{out}
	}%
	\Eq{eq:comp:CompBalance}{
		0 = z_i^{in} - z_i^{out}
	}%
	
	\paragraph{Pressure drop}
	\Eq{eq:reboil:PressureDrop}{
		p^R = p^{in} + \Delta p
	}%
	
	\todo[inline]{add formulas for vapour fraction}
		
\subsubsection{Pump}

\subsubsection{Separator}

\subsubsection{Joule-Thompson valve}

\subsection{Splitter}
	\stdfig{pgfplots/Splitter}{Splitter model}{fig:Splitter}{h}
	
	\paragraph{Mass and component balances}
	\Eq{eq:comp:MassBalance}{
		0 = F^{in} - \sum_i^N F^{in} \zeta_i
	}%
	\Eq{eq:comp:CompBalance}{
		0 = z_i^{in} - z_i^{out}
	}%
	
	\paragraph{Energy balance}
	\Eq{eq:comp:MassBalance}{
		0 = F^{in} H^{in}(T^{in}, p^{in}, z_i^{in}) - \sum_j^N F_j^{out} H^{out}(T^{out}, p^{out}, z_i^{out})
	}%
	
	\paragraph{Pressure drop}
	\Eq{eq:reboil:PressureDrop}{
		p^{out} = p^{in} + \Delta p
	}%

	\todo[inline]{add formulas for vapour fraction}
	
\section{Economic models}
\label{sec:EconModel}
	As discussed earlier economic cosideration play a major role in porcess design. In order to account 
	for the process economics the cost of the process to be implmented needs to be estimated at the design 
	level. However as limited infrmation is availble estimation methods have to be employed. In 
	\refsec{chp:ProcesEconomics} the gerneral approach for cost estimation of process equipment was
	inroduced, where a specific value such as heat-exchange area or vessel size is used to approximate
	equipment cost. However for more specific units extended models are available, where statistical 
	data is employed to yield a more realistic fit to cost data. The cost functions and correction 
	factors presented in this chapter are, if not stated otherwise, taken from \cite{Seider.2010}. 
	Also unless otherwise stated the unit cost is given for the year 2006 ($CE = 500$). 
	
	\subsection{Destillation column}
		Out of all the process equipment the destillation column probably is the most elaborate
		unit. It also poses the greatest challneges when ot comes to finding an appropriate 
		estimate for its cost. This is die to the fact that the column in itself is rather 
		large and complex. To properly operate a coulumn the vessel needs to have numerous 
		valves, scaffolding and several manholes. Due to its size further factors come into 
		play that need not to be considered for the other relatively small units. Those location 
		dependent factors might include resiliance towards earthquakes, the ability to withstand 
		close winds or intensive ambient tmperatures. However as the scope of this work explicitly 
		focuses on early design stages those location specific influences will be disregarded 
		to arrive at simpler models for cost estimation. 
		
		\subsubsection{Vertical tower}
			The cost for the vessel $C_V$ which is to be vertically errected vertically is dependent 
			on teh weight fo the weight $W$ ($[lbs]$) of the vessel. This includes valves, manholes and 
			other details directly connected with the tower. However the cost for ladders, 
			platforms and railings necessary to porperly operate the column are calculated 
			separately 
			\Eq{eq:cost:column:column}{
				C_p = f_M \, C_V + C_{PL}.
			}
			
			The correlated equation for the cost of the tower is given by
			\Eq{eq:cost:column:vessel}{
				C_V = \exp\big\{ 7.2756 + 0.18255 \cdot \ln[W] + 0.02297 \cdot \pow{\ln[W]}{2} \big\}, \eqannote{9000 \leq W \leq 2.5 \cdot 10^6}.
			}
			
			To the cost of the tower, the cost of the surrounding support structure ist added. It is dependent
			on the inner diamerter of the vessel ($D_i$) as well as the so called tangent to tangent length ($L$). 
			This denotes the length of the tube that makes up the vessel exluding the spherical domes that 
			close the column on each side. With that the additional cost is then computed by
			\Eq{eq:cost:column:support}{
				C_{PL} = 300.9 \cdot \pow{D_i}{0.63316} \cdot \pow{L}{0.80161}.
			}
			
		\subsubsection{Weight}
			As can be seen from the above correlations the weight of the coulumn is a determining factor 
			for the estimated and actual cost. Therefore some thought souhld be put into how this can
			be determined, when the final design is unknown. Again several correlations have been applied
			to real life units which yield satisfactory results. In gerneral the weight of the empty vessel 
			can be computed by determing the volume of the material and multiplying it with its density ($\varrho$)
			\Eq{eq:cost:column:weight}{
				W = \pi (D_i + t_s)(L + 0.8 \cdot D_i) t_s \cdot \varrho.
			} 
			The term $0.8 D_i$ is included to approximate the weight of the domes, whereas $t_s$ is the shell 
			thickness. To determine how thick the walls of teh shell need to be teh ASME prssure vessel code
			formula is often applied
			\Eq{eq:cost:column:wallthickness}{
				t_s = \frac{P_d \, D_i}{2 \, S \, E - 1.2 P_d}. 
			}
			Where the maxilmal allowable stress $S$, which the chosen material can withstand at 
			proces conditions is multiplied by the fractional wel efficiency $E$ to regard the effects
			of the manufacturing process on the material strength. To ensure an error on the side of
			caution teh design pressure $P_d$ is calculated from teh actual operatiing pressure $P_o$
			by means of 
			\Eq{eq:cost:column:designpressure}{
				P_d = \exp\big\{ 0.60600 + 0.91615 \cdot \pow{\ln[P_o]}{} + 0.0015655 \cdot \pow{\ln[P_o]}{2}
			}
			It is important to consider, that the maximum allowable stress especially needs to take into account 
			the operating temperature of the distillation process, as it might have significant effects. 
			
			Furthermore the given formulas only apply to pressures above ambient conditions. Thus low pressure
			or vacuum distillation is not covered by the presented formulas. 
			
		\subsubsection{Column internals}
			While internal support sructures are already considered by the equations given above, the 
			internals responsible to ensure product separation are not. Those make up a very significant 
			amount of the total column cost and are available . 
	
	\subsection{Centrifugal pump}
		Pumps are among the most common units of process equipment. While there are several different 
		kinds of punps that can be used, the crtrifugal pump is one of the most popular choices and
		denotes a very likely choice for the porcess conditions considered in this application. Hence
		other pump types will not be considered at this point. 
		
		\subsubsection{Pump}
			In terms of operations pumps are best described by the volumetric flow transported $Q$ as 
			well as the pumph head $H$, the hight that needs to be overcome. Data taken from the company 
			Mosanto was used to correlate the pump cost to a specific value 
			\Eq{eq:cost:pump:SpecVal}{
				S = Q \sqrt{H}.
			}
			As a reference unit the base price $C_B$ is estimated for a cast iron single-stage 
			vertically split case at 3600 $rpm$
			\Eq{eq:cost:pump:PumpWOMotor}{
				C_B = \exp \left\{ 9.7171 - 0.6019 \cdot \ln[S] + 0.0519 (\ln[S])^2 \right\}, 
					\eqannote{400 \leq S \leq 100000}.
			}
			
			The most influential addition factors for the pump price are the material, which is accounted 
			for in the material factor $f_m$, as well as the rotation, case split orientation (horizontal
			and vertical), the number of stages, covered flow rate range, pump head range and maximum 
			motor power, which are all agglomerated in the type factor $f_T$. Values for these factors 
			are given in \reftab{tab:pump:Type} and \reftab{tab:pump:Material}. 
			
			\stdtab{Tables/PumpFactorsType}{Pump type factors \cite{Seider.2010}.}{tab:pump:Type}
			\stdtab{Tables/PumpFactorsMaterial}{Pump material factors \cite{Seider.2010}.}{tab:pump:Material}
			
		\subsubsection{Electric motor}
			Separately from the pump itself the motor to drive the compression is considered. While the
			voulumeric flow and the pump head certainly are valid choices to correlate motors for pumps 
			esecially, the power consumption is a more general specific value 
			\Eq{eq:cost:pump:MotorPowerConsumption}{
				P_C = \frac{P_T}{\eta_P \eta_M} = \frac{P_B}{\eta_M}
			}
			
			It can be calculated from the theoretic power of the pump $P_T$ and the efficiencies $\eta_P$
			$\eta_M$. While an estimate for the expected power consumption might be already available at
			rather early design stages, the efficiencies will have to be correlated as well if resorting 
			to average values is considered too coarse. Those correlations rely on the volumetric flow in 
			gallons per minute ([gpm]) and the brake horse power $P_B = \frac{P_T}{\eta_P}$. 
			
			\Eq{eq:cost:pump:MotorEfficiancyP}{
				\eta_P = -0.316 + 0.24015 \cdot \ln[Q] - 0.01199 \cdot (\ln[Q])^2 \eqannote{50 \leq Q \leq 5000}
			} 
			\Eq{eq:cost:pump:MotorEfficiancyM}{
				\eta_M = 0.80 + 0.0319 \cdot \ln[P_B] - 0.00182 \cdot (\ln[P_B])^2 \eqannote{1 \leq P_B \leq 1500} 
			}
			
			After having calculated the power which the motor needs to supply its base cost of an open, 
			drip-proof enclosed motor at 3600 $rpm$ can be approximated by
			\Eqml{eq:cost:pump:Motor}{
				C_B = \exp\big\{ 5.8259 + 0.13141 \cdot \ln[P_C] + 0.053255 \cdot (\ln[P_C])^2 \\
					+ 0.028628 \cdot (\ln[P_C])^3 - 0.0035549 \cdot (\ln[P_C])^4 \big\} \eqannote{1 \leq P_C \leq 700}
			}
			
			To adjust the cost for different types of electric motors the type factors from \reftab{tab:pump:MotorTypes}
			
			\stdtab{Tables/MotorTypeFactors}{Type factors for different motor types.}{tab:pump:MotorTypes}
	
	\subsection{Compressor}
		The cost of compressors is correlated with their respective power consuption measured in horsepower.
		Although not the most efficient type of compressor, centrifugal compressors are vrey popular in the
		process industry, as they are easily controlled an deliver a very steady flow. However as differnt 
		types might be employed as well base cost correlations for centrifugal, reciprocation and screw
		compressors are given. 
		
			\subsubsection{Centrifugal compressor}
				\Eq{eq:cost:compressor:centrifugal}{
					C_B = \exp\big\{ 7.5800 + 0.80 \cdot (\ln[P_C]) \big\} \eqannote{200 \leq P_C \leq 30000}
				}
			
			\subsubsection{Reciprocating compressor}
				\Eq{eq:cost:compressor:centrifugal}{
					C_B = \exp\big\{ 7.9661 + 0.80 \cdot (\ln[P_C]) \big\} \eqannote{200 \leq P_C \leq 20000}
				}
				
			\subsubsection{Screw compressor}
				\Eq{eq:cost:compressor:centrifugal}{
					C_B = \exp\big\{ 8.1238 + 0.7243 \cdot (\ln[P_C]) \big\} \eqannote{200 \leq P_C \leq 750}
				}
		
		Again as with most othe requipment types correction factors are used to adjust for different realization 
		of this piece of equipment. Here type of motor as well as the construction material have the biggest 
		effectson the unit price and are explicitly considered. 
		\Eq{eq:cost:sompressor:factors}{
			C_p = f_D \, f_M \, C_B
		}
		
		The alternatives to the electric motor ($f_D = 1.0$) are a steam turbine ($f_D = 1.15$) or a gas turbine
		($f_D = 1.25$). It should however be noted that aside from beein the cheapest choice, the electric motor
		is also the most efficient. Thus teh turbines are mostly considered, when process steam or combustion gas 
		is easily availble, such that the drawbacks might be eleimnated by not havin to supply the electric 
		energy for the electric motor. In terms of construction material all base costs are for cast iron or
		carbon steel. Some appliances may require more resistant and also more expensive materials such as
		stainless steel ($f_M = 2.5$) or an nickel alloy ($f_M = 5.0$).   
		
	\subsection{Reboiler / condenser}
		Reboiler and condenser can be characterized as heat exchangers, and be handeled in the same way, 
		as the main difference is weather heat is transferred to or from the process stream. In that sense 
		they must be distinguished when considerin the operating cost, as the cost for hot or cold 
		auxilliary streams might differ significantly. As customary for heat exchangers the specific 
		quantity for cost correlations is the necessary heat exchange area $A$ measured in $ft$. 
		
		Again the consturction material as well as the operating conditions have an effect on the 
		final cost 
		\Eq{eq:cost:condreb:total}{
			C_p = f_P \, f_M \, C_B. 
		}
		
		The correction for pressures $f_P$ takes into account the operating pressure $P_o$ and 
		is comouted by 
		\Eq{eq:cost:condreb:pcorrect}{
			f_P = 0.8510 + 0.1292 P_o + 0.0198 * P_o^2 . 
		}
		
		The material correction factor $f_M$ 
		\Eq{}{
			f_M = 
		}
		
		\subsubsection{Shell and tube heat exchanger}
		\Eq{eq:cost:condreb}{
			C_B = \exp\big\{ 11.667 - 0.8709 \cdot \pow{\ln[A]}{} + 0.09005 \cdot \pow{\ln[A]}{2} \big\}
		}
		
		\subsubsection{Double pipe}
		\Eq{}{
			C_B = \exp\big\{ 7.146 + 0.1600 \cdot \pow{\ln[A]}{} \big\}
		}
		