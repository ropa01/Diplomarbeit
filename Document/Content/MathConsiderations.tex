\chapter{Mathematical considerations}
\label{chp:MathModel}

\section{Robust column model}
The column model, although rather simple on first sight, form a highly non-linear
and higly coupled set of equations, which need to be solved simultaniously. Due
to that even advanced computers and solution algorithms often fail to find feasible 
steady-state solutions for this higly complex problem. For every solution approach 
good initial estimates are essential when it comes to finding a fasible solution. 
Those problems are amplyfied when non-ideal or very close boiling systems are encountered
in the distillation column. 

In order to adderss these problems Boston and Sullivan \cite{Boston.1974} developed an 
alternative model fomulation alongside a solution algorithm that has been proven to 
to very effectively and effciently converge towards a fesible steady state solution. 
Furthermore they describe a scheme by which initial guesses for all relevant column 
variables can be derived. 

This alternative formulation involves the introduction of several new variables 
in order to decouple the intrinsic interdependencies and facilitate the solution 
of the column model. While the general formulation of the distillation column 
has been presented in \refsec{sec:SSM:dist} the alternative formulation will be 
given in the following sections. The alternative model formuation and solution
algorithm Boston and Sullivan arrived at are described in the following sections. 
They are widely referred to as the ''inside-out'' -- algorithm as will be adopted 
in this paper. 
 
\subsubsection{Vapour-liquid equilibrium (VLE)}
In general the vapour-liquid equilibrium (VLE) between a liquid and a vapour phase 
can be described by \cite{AndreasPfennig.2003}
\Eq{eq:io:vle}{
	x_i \, \underbrace{\gamma_i(\vec{x}, T, p^S) \, F_{P, i} \, \varphi_i^0(T, p_i^S(T)) \, p_i^S(T)}_{f_i^L}
		= y_i \, \underbrace{\varphi_i(\vec{y}, T, p^s) \, p^s}_{f_i^V} \eqannote{i = 1, \dots, N_s}
	\nc{x_i}{liquid mole fraction of species $i$}{\frac{mol}{mol}}
	\nc{y_i}{vapour mole fraction of species $i$}{\frac{mol}{mol}}
	\nc{\gamma_i}{liquid activity coefficient of species $i$}{-}
	\nc{\varphi_i}{vapour fugacity coefficient of species $i$}{-}
	\nc{\varphi^0_i}{reference vapour fugacity coefficient of species $i$}{-}
	\nc{p^S_i}{vapour pressure of species $i$}{Pa}
	\nc{p^S}{system pressure}{Pa}
	\nc{F_{p,i}}{liquid compressibility factor for species $i$}{-}
	\nc{f_i^L}{liquid fugacity coefficient for species $i$}{-}
	\nc{f_i^V}{vapour fugacity coefficient for species $i$}{-}
	\nc{T}{system temperature}{K}
}

Here $\gamma_i$ denotes the activity coefficient of sepecies $i$ in the liquid mixture, 
$\varphi_i$ and $\varphi_i^0$ the vapour fucatity coefficient and the reference vapour
fugacity coefficient at the species's vapour pressure respectively. 
They to computed from an applicable equation of state (EOS) such as the Sovae-Redlich-Kwong (SRK)
equation or the Peng-Robinson (PR) equation. For higly non-ideal systems often excess Gibbs 
energy models such as the Wilson or NRTL model are emplyed to compute the liquid 
activity coefficients. Considering, that the mentioned models in itself often form qubic 
or even more complex equation it becomes evident that finding a feasible solution for 
all involved species simultaniouly form a major challange. One predominant obstacle is 
that almost all terms in are temperature dependent which introduces tremendous interdependencies
between all equilibrium equations for all species $N_s$ $N$ on all stages in the distillation column. 

To adress this problem the inside out algorithm includes the so called ''$K_b$ - model'' 
which aims to separate the the temperature dependencies from the composition dependent terms. 

\refeq{eq:io:vle} can be rewritten in terms of a vapour-liquid equilibrium ratio $K_i$
\Eq{eq:io:eqratio}{
	y_{i,n} = \frac{f_{i,n}^L}{f_{i,n}^V} \, x_{i,n} = K_{i,n} \, x_{i,n} \eqannote{i = 1, \dots, N_s}.
	\nc{K_{i,n}}{equilibrium ration for species $i$ on tray $n$}{-}
}
In order to achieve the desired effect the equilibrium ratio was devided in an solely temperature
dependent parameter $K_b$ and the so called volatility parameter $\alpha_i$ defined by
\Eq{eq:io:volatilityparam}{
	K_{i,n} = K_{b,n} \, \alpha_{i,n}, \eqannote{i = 1 \dots N_s}.
	\nc{\alpha_{i,n}}{volatility parameter of species $i$ on tray $n$}{-}
	\nc{K_{b,n}}{$K_b$ parameter}{-}
}
This volatility parameter is mainly dependent on system composition and only mildly dependent 
on temperature. To ensure the for the $K_b$ parameter to absorb the temperature effects is
was required, that the modified bubble point equation (BP) with an temperature independent 
volatility parameter 
\Eq{eq:moathcon:io:modbp}{
	\sum_{i,n} K_b \, \alpha_{i,n} \, x_{i,n} = 1, \eqannote{i = 1 \dots N_s, \, n = 1, \dots, N,}
}
matches the temperature sensitivity of the original bubble point equation, 
\Eq{eq:io:bp}{
	\sum_i K_{i,n} x_{i,n} = 1
}
as closely as possible. This leads directly to \cite{Boston.1974}
\Eq{eq:io:kbb}{
	\fracddpart{\log K_{b,n}}{\frac{1}{T_n}} = \sum_i y_{i,n} \left( \fracddpart{\log K_{i,n}}{\frac{1}{T_n}} 
		\right)_{\vec{x}, \vec{y}} \eqannote{i = 1 \dots N_s} 
}

It was then further assumed, that a simple model would suffice to capture the temperature dependence of
$K_b$ over a small temperature interval
\Eq{}{
	\log K_{b, n} = A_n - \frac{B_n}{T_n}, \eqannote{n = 1, \dots, N}.
	\nc{B_n}{parameter in $K_b$ model for tray $n$}{K}
	\nc{A_n}{parameter in $K_b$ model for tray $n$}{-}
}

The value of $B_n$ is determied by \refeq{eq:io:kbb} while $A_n$ is initially set to 
\Eq{}{
	A_n = \sum_i y_i \, \log K_{i, n} + \frac{B_n}{T_n}
}

\subsubsection{Component mass conservation (CMC)}
In order to transform the component mass conservation equations \refeq{eq:col:CompBalance}
into a system which would be more robust in terms of numerical solution algorithms, 
they were reformulated in terms of the liquid component flows ($\ell_{i, n}$) alone
\Eq{eq:io:modcmc}{
	f_{i, n} = - \ell_{i, n-1} + b_{i, n} \ell_{i, n} - c_{i, n} \ell_{i, n+1}, 
		\eqannote{i = 1, \dots, N_s, \, n = 2, \dots, N-1}. 
	\nc{b_{i, n}}{parameter for CMC equations}{-}
	\nc{c_{i, n}}{parameter for CMC equations}{-}
}
Where the parameters $b_{i, n}$ and $c_{i, n}$ are expressed in terms of the newly 
introduced ''S-parameters'' or splitting factors and the ''R-parameters'' or 
withdrawl factors
\Eq{eq:io:c}{
	b_{i, n} & = R_{L, n} + R_{V_ n} \, S_b \, S_{R, n} \, \alpha_{i, n} 
		& \eqannote{i = 1, \dots, N_s, \, n = 2, \dots, N-1}, \label{eq:io:b} \\
	c_{i, n} & = S_b \, S_{R, n+1} \, \alpha_{i, n+1} 
		& \eqannote{i = 1, \dots, N_s, \, n = 2, \dots, N-1}.
}
With the definitions for the newly introduced variables
\Eq{}{
	R^L_n & = 1 + \frac{S^L_n}{L_n}, \\
	R^V_n & = 1 + \frac{S^V_n}{V_n}, \\
	S_n & = \frac{K_{b, n} \, V_n}{L_n}, \\
	S_{R, n} & = \frac{S_n}{S_b}, \\
	S_b & = \sqrt[N]{\prod_n S_n}.  
	\nc{R^L_n}{liquid withdrawl parameter for stage $n$}{-}
	\nc{R^V_n}{vapour withdrawl parameter for stage $n$}{-}
	\nc{S_n}{splitting factor for stage $n$}{-}
	\nc{S_b}{base splitting factor}{-}
	\nc{S_{R, n}}{realtive splitting factor for stage $n$}{-}
}
From the liquid streams the vapour and liquid compositions are calculated
\Eq{eq:io:molefrac}{
	x_{i, n} & = \frac{\ell_{i, n}}{\sum_i \ell_{i, n}}, \\
	y_{i, n} & = \frac{\alpha_{i, n} \, \ell_{i, n}}{\sum_i \alpha_{i, n} \, \ell_{i, n}}.
}

\subsubsection{Energy conservation (EC)}
In order to simplify the engergy conservation equations the total stream enthalpies
were approximated by several new variables referred to as ''energy parmeters''.
Here separate models are employed for vapour and liquid enthalpies. 

The total vapour enthalpy can be expressed as 
\Eq{eq:io:vapenth}{
	H^V_n =\sum_i y_i \, H^0_{i, n} + \Delta H^V_n (\vec{y}_n, p_n, T_n), \eqannote{i = 1, \dots N_s}
}  
where $H^0_{i, n}$ is the ideal gas enthalpy of species $i$ on tray $n$ and $\Delta H^V_n$
the pressure and composition correction, which can be expressed in terms of the vapour fugacity 
\Eq{eq:io:vapenthcorr}{
	\Delta H^V_n (\vec{y}_n, p_n, T_n) = R \sum_i y_i \, \left( \fracddpart{\log \varphi_{i, n}}{\frac{1}{T}} 
		\right)_{p, \vec{y}_n}
} 
To avoid the computaionally exopensive porperty calls the correction term was approximaed by
\Eq{eq:io:vapenthapprox}{
	\Delta H^V_n (\vec{y}_n, p_n, T_n) = \Phi_y + \Phi_T (T - T^{\ast})
}
where 
\Eq{}{
	\Phi_y & = \frac{\Delta H^V_n (\vec{y}_n^{\ast}, p_n, T_n) - \Delta H^V_n (\vec{y}_n^{\ast}, p_n, T_n^{\ast})
		}{T - T^{\ast}} \\
	\Phi_T & = \Delta H^V_n (\vec{y}_n^{\ast}, p_n, T_n^{\ast}) + \Delta H^V_n (\vec{y}_n, p_n, T_n)
		- \Delta H^V_n (\vec{y}_n^{\ast}, p_n, T_n)
}
\Eq{}{
	H_n = \Gamma \, \Theta + \Phi_y + \Phi_T \, (\Gamma - \Gamma^{\ast}) + \sum_i y_i \, H^0_{b, i}
}
\Eq{}{
	\Phi_i & = \frac{H^0_i - H^0_{b, i}}{T - T_b} \\
	\Theta & = \Theta_b \, \Theta_r \\
	\Theta_r & = \sum_i y_{r, i} \Phi_{r, i} \\
	\Theta_b & = \sum_i y_i \, \Phi_{b, i} \\
	\Phi_{b, i} & = \lim_{T \rightarrow T_b} \Phi_{i} \\
	\Phi_{r, i} & = \frac{\Phi_i}{\Phi_{b, i}} \\
	y_{r, i} & = \frac{y_i \, \Phi_{b, i}}{\sum_j y_j \, \Phi_{b, j}} 
}

An equivalent approach is chosen to model the liquid enthalpy
\Eq{}{
	H^L_n = h^0_i + h^E_i
}

\Eq{}{
	\phi_i & = \frac{h^0_i - h^0_{b, i}}{T - T_b} \\
	\theta & = \theta_b \, \theta_r \\
	\theta_r & = \sum_i x_{r, i} \phi_{r, i} \\
	\theta_b & = \sum_i x_i \, \phi_{b, i} \\
	\phi_{b, i} & = \lim_{T \rightarrow T_b} \phi_{i} \\
	\phi_{r, i} & = \frac{\phi_i}{\phi_{b, i}} \\
	x_{r, i} & = \frac{x_i \, \phi_{b, i}}{\sum_j x_j \, \phi_{b, j}} 
}


With those the modified energy conservation equation is formulated
\Eq{eq:io:ec}{
	E_n - a_n \, S_,^L - a'_n \, (M_{T, n-1} + S^V_n) + a'_{n+1} \, M_{T, n}
		= - (a_{n-1} - a'_n) \, L_{n-1} + (a_n - a'_{n+1}) \, L_n
}
where 
\Eq{}{
	a_n & = \Gamma_n \, \theta_n + h^E_n - \lambda^0_{b, n} \\
	a'_n & = \Gamma_n \, \Theta_n + \Phi_{y, n} + \Phi_{T, n} (\Gamma_n - \Gamma_n^{\ast}) 
}