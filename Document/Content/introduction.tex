\chapter{Introduction}
\label{chp:intro}
Distillation columns are to this day the, most prominent way for separating liquid mixtures. Vast amounts of energy
are used during the separation of various product streams. In fact often the majority of the energy required in
a given process is consumed in the separation stage. Therefore separation sequences have been under investigation
for many years. The synthesis of separation sequences poses great challenges to this day, especially if complex
multi-component mixtures are considered.

Several shortcut methods have been developed for screening of potential separation sequences. If several
flowsheet alternatives are constructed it becomes necessary to further investigate, which is the most
suited to a given task. For this rigourous process models which incorporate tray by tray balances and
non-ideal equilibria are employed. One approach, which is currently still employed in industry practice,
is to carry out a sequence of process simulations to find the most promising configuration in a manual, trial
and error based procedure. However this approach is unlikely to yield near optimal results.

The more reliable alternative to synthesize near optimal flowsheets is to perform mathematical optimization
on a given process superstructure. In that context continuous decisions such as operating pressure or temperatures,
as well as discrete decisions associated with structural choices on the process equipment have to be made. The
resulting problems are called mixed-integer non-linear programs (MINLP), which pose considerable challenges
in their solution process. Due to that structural optimization is carried out by experts, with careful consideration
when setting up the problem and optimization in specialized programs.

On the other hand process simulation in flowsheeting environments such as \aspen or \gproms is an integral
part of industrial practice. Hence the scope of this thesis is, to implement a flexible, easily employable
model for process optimization in the aforemention modeling environment \gproms which is capable of
automated model initialization as well as structural optimization.

The implemented models will then be applied to the cryogenic air separation process. This process is used to
produce highly pure nitrogen, oxygen and noble gases such as argon. While the components themselves
do not display any severe non-idealities, the process is highly integrated and coupled in terms of energy
and material streams and contains columns with multiple feeds and side draws, and complex column configurations.
This makes it especially challenging in terms of process optimization.

The following thesis is structured as follows. First a brief description of the cryogenic air separation
process will be given. Subsequently, as economic considerations will always have to be considered,
general approaches to estimate the economics associated with process design will be discussed.

In the third chapter a detailed overview of the models implemented in \gproms will be given. Afterwards,
the optimization of chemical processes will be dealt with. Within that section, first some general theoretical
basics of optimization theory are introduced followed by a closer look at solution algorithms for
discrete-continuous problems. Finally optimizations will be carried out on the cryogenic air separation
process. Using that example, the performance of different solution approaches and optimization objectives
in terms of optimality and efficiency will be compared. Finally some possible directions for future research
are mentioned.
