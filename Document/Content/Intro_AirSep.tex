\chapter{Introduction}
\label{chp:intro}
Distillation columns are to this day the most prominent way for separating liquid mixtures. Vast amounts of energy
are used during the separation of various product streams. In fact often the majority of the energy required in
a given chemical process is consumed in the separation stage. Therefore separation sequences have been under investigation
for many years. The synthesis of separation sequences poses great challenges, especially if complex
multi-component mixtures are considered.

Several shortcut methods have been developed for screening of potential separation sequences. If several
flowsheet alternatives are constructed it becomes necessary to further investigate, which is the most
suited for requirements of the separation task at hand. To do so rigourous process models which incorporate tray-by-tray balances and
non-ideal equilibria are employed. One approach, which is currently still carried out in industry practice,
is to conduct a sequence of process simulations to find the most promising configuration in a manual, trial
and error based procedure. However this approach is unlikely to yield near optimal results.

The more reliable alternative to synthesize near optimal flow sheets is to perform mathematical optimization
on a process superstructure. In that context continuous decisions such as operating pressure or temperatures,
as well as discrete decisions associated with structural choices on the process equipment have to be made. The
resulting problems can be formulated as mixed-integer non-linear programs (MINLP), which pose considerable challenges
in their solution process. Due to that structural optimization is carried out by experts, with careful consideration
when setting up the problem and optimization in specialized programs.

On the other hand process simulation in flowsheeting environments such as \aspen or \gproms is an integral
part of industrial practice. Hence the scope of this thesis is to implement a flexible, easily deployed
model for process optimization in the aforementioned modeling environment \gproms, which is capable of
automated model initialization as well as structural optimization.
The implemented models will then be applied to the cryogenic air separation process. This process is used to
produce highly pure nitrogen, oxygen and noble gases such as argon. While the components themselves
do not display any severe non-idealities, the process is highly integrated and coupled in terms of energy
and material streams. The column configuration features multiple feeds and side draws in conjunction with
a side stripping section. This makes it especially challenging in terms of process optimization.

The following thesis is structured in six chapters. First a brief description of air separation technologies
will be given. Subsequently general approaches to estimate the economics associated with process design will be discussed,
as economic considerations are of major concern when designing a chemical process.
The third chapter consists of a detailed overview of the models implemented in \gproms. Afterwards,
the optimization of chemical processes will be dealt with. Within that section, first some general theoretical
basics of optimization theory are introduced followed by a closer look at solution algorithms for
discrete-continuous problems. Finally optimization results for the cryogenic air separation
process are presented. For that example, the performance of different solution approaches and optimization objectives
in terms of optimality and efficiency will be compared.
The final chapter summarizes significant parts of the thesis and provides suggestions for further investigation in
the matter.
