% Task Description
\thispagestyle{plain}

\begin{center}
	\LARGE
	Diploma Thesis\\
	for\\
	Mr. Robert Pack\\
	\vspace{4ex}

	\vspace{3ex} Task Description
\end{center}

\normalsize

Current air separation process optimization is carried out sequentially with various discrete or continuous process parameters,
and then iterated with equipment sizing and costing. Typical process parameters include temperature, pressure, flow, pressure
drop, product purity, distillation column feed and/or draw locations staging, and etc. Major equipment used in cryogenic air
separation include air compressors, turbines, plate and fin heat exchangers, cooler, distillation columns, condensers, reboilers,
and etc. Once process optimum condition is identified, individual equipment is sized and cost estimated. Capital cost is then feed
back to process to verify the original optimum condition. If the original optimum condition changes, we will have to re-optimize
the process parameter and identify the new process optimum condition. New equipment sizing and costing will have to be
adjusted accordingly.

The deficiency of the current process is apparent: manual, sequential and iterative. We would like to streamline the optimization
process, incorporate the capital equipment cost into the process optimization. The process optimizer we target is gPROMS. The
goal is to simultaneously optimize process and equipment to minimize the total process cost with user defined objective functions.

 \vfill{
\begin{flushright}
	\begin{minipage}[r]{10cm}
		\begin{center}
			\hrule\vspace*{2ex}Prof. Efstratios Pistikopoulos\\
			\vspace{2cm}
			\hrule\vspace*{2ex}Prof. Dr.-Ing. Wolfgang Marquardt\\
			\vspace{1cm}
		\end{center}
	\end{minipage}
\end{flushright}
}

Betreuer der Arbeit: Dipl. Ing. Mirko Skiborowski\\
Tag der Abgabe: xx.xx.2012
\todoil{color=AVTLight2Blue}{Add due date}
\clearpage