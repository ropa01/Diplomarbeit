\chapter{Mixed-integer optimization}
\label{chp:MINLP}
In this section some general considerations about mixed-integer non-linear programming (MINLP) will be presented.
First the most commonly applied solution techniques will briefly be discussed on the basis of a review on the
subject by Grossmann \cite{Grossmann.2002}. Then then an alternative approach to solve this class of problems,
based on a continuous reformulation of the discrete decision variables, which has recently been proposed
\cite{Kraemer.2010,Stein.2004} will also be introduced. This reformulation is also the basis for a comparison of different
approaches to solve this class of problem within the process simulation environment \gproms.

    \section{solution techniques}

    As previously stated, the following discussion of general solution techniques for MINLP's is largely based
    in the comprehensive review by I. E. Grossmann \cite{Grossmann.2002}.

    Several approaches to tackle this type of problem have successfully been applied to a multitude of cases for
    several years now. In general four major types of solution algorithms can be distinguished.
    \begin{itemize}
        \item branch and bound
        \item outer approximation
        \item generalized benders
        \item extended cutting plane
    \end{itemize}
    All these algorithms make use of a limited number of common subproblems, which are then solved
    in different configurations. Therefore these subproblems can previously be discussed and will be referred
    to when going elaborating on different algorithms. Furthermore it needs to be emphasised, that the presented
    solution techniques are designed for convex problems, and only in those cases an optimal solution can
    be found with a degree of confidence. While they can be applied to the more general non-convex case, only
    local optimality can be assured.

    The general case of a MINLP takes the form
    \program{prog:stdMINLP}{
        & \minimize{x,y} & & C = f(x,y) \\
        & \st & & g_j(x,y) \leq 0, \; \; j \in \mathset{J} \\
        & & & x \in \mathset{X}, \; y \in \mathset{Y}
    }

    If the discrete variables in the discrete-continuous program are relaxed, the resulting NLP relaxation
    can be written as
    \program{prog:relaxedNLP}{
        & \minimize{x,y} & & C_{LB}^k = f(x,y) \\
        & \st & & g_j(x,y) \leq 0, \; \; j \in \mathset{J} \\
        & & & x \in \mathset{X}, \; \; y \in \mathset{Y}_R \\
        & & & y_i \leq \alpha_i^k, \; i \in \mathset{I}_{FL}^k \\
        & & & y_i \geq \beta_i^k, \; i \in \mathset{I}_{FU}^k
    }
    Here $\mathset{Y}_R$ denotes the relaxed set of the integer set $\mathset{Y}$, $\mathset{I}_{FL}^k$ and
    $\mathset{I}_{FU}^k$ are subsets of the indices denoting the entire set of integer variables.
    The relaxed integers contained in these sets are  bounded to the values $\alpha_i^k$ and $\beta_i^k$ respectively.
    These bounds are lower and upper bound taken from the previous iteration of the algorithm in question. If
    $\mathset{I}_{FL}^k = \emptyset$ and $\mathset{I}_{FU}^k = \emptyset$ are empty sets, problem \progref{prog:relaxedNLP}
    denotes the fully relaxed problem, initially solved in all algorithms. The optimal solution to this initially solved
    problem ($C_{LB}^0$) poses an absolute lower bound to \progref{prog:stdMINLP} since all variables are fully relaxed.

    If all discrete variables are fixed at a given value, the NLP subproblem for fixed $y^k$ results, as
    only continuous variables are being considered. 
    \program{prog:fixedyNLP}{
        & \minimize{x} & & C_{LB}^k = f(x,y^k) \\
        & \st & & g_j(x,y^k) \leq 0, \; \; j \in \mathset{J} \\
        & & & x \in \mathset{X}
    }
    If \progref{prog:fixedyNLP} is infeasible the NLP feasibility subproblem for fixed $y^k$ can be solved. 
    \program{prog:feasibleNLP}{
        & \minimize{x} & & u \\
        & \st & & g_j(x,y^k) \leq u, \; \; j \in \mathset{J} \\
        & & & x \in \mathset{X}, \; \; u \in \mathbb{R}
    }
    This NLP returns a strictly positive value for $u$. 
    
    Aside from the presented NLP's a linearized version of \progref{prog:stdMINLP} is regularly solved 
    \program{prog:MILP}{
        & \minimize{x, y} & & C_L^k = \alpha \\
        & \st & & f(x^k,y^k) + \nabla f(x^k,y^k)^T \begin{bmatrix} x - x^k \\ y - y^k \end{bmatrix} \leq \alpha \\
        & & & g_j(x^k,y^k) + \nabla g_j(x^k,y^k)^T \begin{bmatrix} x - x^k \\ y - y^k \end{bmatrix} \leq 0, \; \; j \in \mathset{J} \\
        & & & x \in \mathset{X}, \; \; y \in \mathset{Y}, \; \; k = {1 \dots K}
    }
    The linearized problem can be constructed in several ways from the set of points $K$ attained in previous iterations. 
    Sometimes only violated or active constraints are linearized. When objective function and constraints are convex, the 
    objective function is underestimated, while the constraints are overestimated. From the overestimated constraints stems 
    the name outer approximation. 
    
        \subsection{Branch \& bound}
        The branch and bound (B\&B) algorithm has originally been proposed for linear problems, but has since been extended to 
        handle non-linear objectives and constraints. As an initial step for the branch and bound the fully relaxed 
        NLP \progref{prog:relaxedNLP} is solved, which, as mentioned before, provides an absolute lower bound to 
        the program in question. In the rare case that all relaxed integer variables assume integer values the optimal 
        solution has been found and the algorithm can terminate. Otherwise a tree search is performed, exploring the 
        space of the integer variables. In each step an increasing number of integer variables is fixed such that 
        a NLP in the form of \progref{prog:fixedyNLP} needs to be solved. The solution of these subproblems is 
        a new lower bound for all descendant nodes. Further exploration of branches can be stopped, once a 
        given subproblem returns a value grater than the current upper bound or becomes infeasible. 
            
        Due to the large number of NLP subproblems that have to be solved within the tree search, the branch and bound 
        algorithm is most attractive, if the solution is computationally inexpensive, or few nodes have to be explored. 
        
        \subsection{Outer approximation}
        The outer approximation (OA) algorithm relies on consecutively solving \progref{prog:fixedyNLP} and \progref{prog:MILP}. 
        Each solution of the NLP with fixed $y^k$ yields a new point $(x^k,y^k)$ which is used to construct an updated version 
        of the MILP. The MILP generally includes linearized versions of all constraints and the objective function. As more and 
        more points become available during the iterative process, new constraints are constructed for each available point. 
            
        The main theorem for the derivation of the OA algorithm states, that the optimal solution of the problem \progref{prog:MILP}
        constructed from all points $(x^k,y^k), k \in K^{\ast}$. Where $K^{\ast}$ is made up of all optimal solutions of \progref{prog:fixedyNLP}
        where the current $y^k$ yields a feasible solution, and \progref{prog:feasibleNLP} where infeasible solutions of the NLP with fixed $y^k$
        are encountered. It should again be emphasized, that this theorem holds only for convex a objective function and constraints. 
        
        As the points necessary to construct the aforementioned system are not available when the solution process commences, a smaller 
        systems is constructed and extended as more points become available. The first point again results from solving a fully relaxed
        system. \todo{is this true, that the fully relaxed system is initially solved????} This again yields an absolute lower bound 
        for the original problem. The solution of each consecutive MILP gives a new lower bound which will always be greater than the 
        bounds from previous iterations. Without any prove this argument is supported by the fact that adding new linear constraints 
        will limit the feasible region of the problem and hence further restrict the possible solutions for the continuous variables.
        
        The optimal points attained from the NLP subproblems with fixed discrete values form upper bounds on the optimal solution. 
        Here no statement can be made about the quality of the bound in each step, but rather is the current upper bound updated, 
        once a lower value is encountered. 
        
        The iterative process terminates, once the 
     
        
    \section{Continuous reformulation}