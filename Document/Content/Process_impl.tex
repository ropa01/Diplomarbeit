As mentioned the presented models have been implemented in the process simulator \gproms. Although application 
to the cryogenic air separation process served as case where the model would be applied, the aim was to develop 
-- especially in the case of the column models -- a flexible model which could be used for a multitude of 
problems while trying to achieve a reasonable amount of complexity, such that a user mainly used to a pure flowsheeting 
environment should be able to apply the models with relative ease. 

In the following sections first some general aspects of modeling in \gproms will be pointed out and the structure 
of the implemented model discussed in more detail. Subsequently several strategies for initializing the column model 
based on the previously model structure will be presented.  

\subsection{Model structure}
    \begin{figure}
        \center
        \begin{tikzpicture}
    \node (A) [rectdg] at (0,0) {core \\ equations} ;
    \node (B) [rectdg] at (-7,-2.5) {concentration \\ init profiles} ; 
    \node (C) [rectdg] at (-3.5,-2.5) {temperature \\ init profiles} ; 
    \node (D) [rectdg] at (-0,-2.5) {murphee \\ efficiencies} ; 
    \node (E) [rectdg] at (3.5,-2.5) {cost \& \\ sizing} ; 
    \node (F) [rectdg] at (7,-2.5) {hydraulics} ; 
    \node (sec) at (0,-1.25) {} ; 
    \draw [stdline] (A.south) -- (D.north) ;
    \draw [stdline] (B.north) -- (B.north |- sec) -- (F.north |- sec) -- (F.north) ;
    \draw [stdline] (E.north) -- (E.north |- sec) ;
    \draw [stdline] (C.north) -- (C.north |- sec) ;
    \node (E1) [rectch] at (3.75,-4.5) {trayed } ;
    \node (E2) [rectch] at (3.75,-6.5) {structured \\ packing} ;
    \node (F1) [rectch] at (7.25,-4.5) {trayed } ;
    \node (F2) [rectch] at (7.25,-6.5) {structured \\ packing} ;
    \draw [stdline] (2.25,-3.25) -- ++(0,-1.25) -- ++(0.25,0) ;
    \draw [stdline] (2.25,-4.5) -- ++(0,-2) -- ++(0.25,0) ;
    \draw [stdline] (5.75,-3.25) -- ++(0,-1.25) -- ++(0.25,0) ;
    \draw [stdline] (5.75,-4.5) -- ++(0,-2) -- ++(0.25,0) ;
\end{tikzpicture}

        \caption{Hierarchical model structure.}
        \label{fig:mathpro:modelstruct}
    \end{figure}

\subsection{Initialization procedures}
