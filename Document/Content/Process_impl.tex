As mentioned the presented models have been implemented in the process simulator \gproms. Although application
to the cryogenic air separation process served as case where the model would be applied, the aim was to develop
-- especially in the case of the column models -- a flexible model which could be used for a multitude of
problems while trying to achieve a reasonable amount of complexity, such that a user mainly used to a pure flowsheeting
environment should be able to apply the models with relative ease.

In the following sections first some general aspects of modeling in \gproms will be pointed out and the structure
of the implemented model discussed in more detail. Subsequently several strategies for initializing the column model
based on the previously model structure will be presented.

\subsection{Model structure}
\label{sec:mathpro:implementation:modelstructure}

    When discussing the structure of \gproms models first some aspects of the modeling language have to be discussed.
    In this context CASE selectors, IF statements and the implementation INITIALIZATION\_PROCEDURE's in \gproms are most
    important. Within this section we will first briefly discuss those features and how they are used in \gproms.
    Afterwards it will be shown, how these structures are used, to create a model which can quickly be tailored the
    specific needs of a given user, and how distillation columns can be initialized within that framework.

    IF statements and case selectors have many similarities as they can distinguish between different cases within a
    given model. For many applications they might even be used equivalently. As in almost all contexts the IF statements
    evaluates a condition and depending on wether it is fulfilled or not and evaluates differen formulae accordingly.
    A  CASE statement distinguishes different branches between the model can switch during simulation or initialization.
    One major difference ist, that the number of equations in each CASE branch must match while for IF statements that
    condition is not enforced. However care must be taken when using these structures -- especially during simulation --
    as they will in many cases introduce discontinuities. Furthermore when using IF statements one must distinguish,
    wether the evaluated statement is a parameter, or a variable in the model. Given it is a parameter, the statement will
    remain true (or false) for each instance of simulation. This introduces some flexibility in terms of diverging
    in the number of equations within model variants.

    INITIALIZATION\_PROCEDURE's are used -- as the name suggests -- to initialize models. The basic idea is to automatically
    solve a sequence of increasingly complex models, using the solution of a previous model as starting point for
    converging the current one. Furthermore the switch between models can be done within a either a so called JUMP\_TO
    or MOVE\_TO statement. Within the former there is a hard cut between the model variants, while in the latter
    a homotopic approach is employed to get form one model branch to another. While this should be the more robust
    solution, if the path of the homotopy remains in a feasible region, it necessitates greater care in terms of model
    robustness and scaling, as the internal integrator is used to move along the homotopic path.

    \begin{landscape}
        \begin{figure}
            \center
            \begin{tikzpicture}
%    \node (A) [rectdg] at (0,0) {core \\ equations} ;
%    \node (B) [rectdg] at (-7,-2.5) {concentration \\ init profiles} ;
%    \node (C) [rectdg] at (-3.5,-2.5) {temperature \\ init profiles} ;
%    \node (D) [rectdg] at (-0,-2.5) {murphee \\ efficiencies} ;
%    \node (E) [rectdg] at (3.5,-2.5) {cost \& \\ sizing} ;
%    \node (F) [rectdg] at (7,-2.5) {hydraulics} ;
%    \node (sec) at (0,-1.25) {} ;
%    \draw [line width=0.5pt] (A.south) -- (D.north) ;
%    \draw [line width=0.5pt] (B.north) -- (B.north |- sec) -- (F.north |- sec) -- (F.north) ;
%    \draw [line width=0.5pt] (E.north) -- (E.north |- sec) ;
%    \draw [line width=0.5pt] (C.north) -- (C.north |- sec) ;
%    \node (E1) [rectch] at (3.75,-4.5) {trayed } ;
%    \node (E2) [rectch] at (3.75,-6.5) {structured \\ packing} ;
%    \node (F1) [rectch] at (7.25,-4.5) {trayed } ;
%    \node (F2) [rectch] at (7.25,-6.5) {structured \\ packing} ;
%    \draw [line width=0.5pt] (2.25,-3.25) -- ++(0,-1.25) -- ++(0.25,0) ;
%    \draw [line width=0.5pt] (2.25,-4.5) -- ++(0,-2) -- ++(0.25,0) ;
%    \draw [line width=0.5pt] (5.75,-3.25) -- ++(0,-1.25) -- ++(0.25,0) ;
%    \draw [line width=0.5pt] (5.75,-4.5) -- ++(0,-2) -- ++(0.25,0) ;
    
%    \node (G) [model,inner sep=0] at (0,0) {MESH equations} ;

    \draw [line width=0.5pt] (-0.25,1.75) rectangle (16.75,2.75) node [align=center] at (8.25,2.25) {main model (MESH equations)} ;
    \draw [line width=0.5pt] (-0.25,1.75) -- ++(0,-6.25) -- ++(17,0) -- ++(0,6.25) ;
    \draw [arrow] (8.25,1.75) -- ++(0,-0.5) -- ++(6.25,0) -- ++(0,-0.5) ; 
    \draw [arrow] (8.25,1.25) -- ++(-6.25,0) -- ++(0,-0.5) ; 
    \draw [arrow] (8.25,-4.5) -- ++(0,-1) ; 
    \draw [arrow] (8.25,-5) -- ++(-4.5,0) -- ++(0,-0.5) ;
    \draw [arrow] (8.25,-5) -- ++(4.5,0) -- ++(0,-0.5) ;
    \draw [arrow] (12.75,-5) -- ++(4.5,0) -- ++(0,-0.5) ;
    \draw [arrow] (17.25,-5) -- ++(2.5,0) -- ++(0,-2) -- ++(-2.5,0) -- ++(0,-0.5) ;
    \draw [arrow] (3.75,-5) -- ++(-4.5,0) -- ++(0,-0.5) ;
    
    \draw [line width=0.5pt] (0,0) rectangle (3.75,0.75) node [align=center] at (1.875,0.375) {material} ; 
    \draw [line width=0.5pt] (4.25,0) rectangle (8,0.75) node [align=center] at (6.125,0.375) {equilibrium} ;
    \draw [line width=0.5pt] (8.5,0) rectangle (12.25,0.75) node [align=center] at (10.375,0.375) {enthalpy} ;
    \draw [line width=0.5pt] (12.75,0) rectangle (16.5,0.75) node [align=center] at (14.625,0.375) {summation} ;
    
    \draw [arrow] (0.15,0) -- ++(0,-0.625) -- ++(0.35,0) ; 
    \draw [arrow] (0.15,-0.625) -- ++(0,-1) -- ++(0.35,0) ; 
    \draw [arrow] (0.15,-1.625) -- ++(0,-1) -- ++(0.35,0) ; 
    \draw [line width=0.5pt] (0.5,-1) rectangle (3.5,-0.25) node [align=center] at (2,-0.625) {rigorous} ; 
    \draw [line width=0.5pt] (0.5,-2) rectangle (3.5,-1.25) node [align=center] at (2,-1.625) {flowrates} ; 
    \node [rectangle,minimum height=0.75cm,minimum width=3cm,fill=AVTLightBlue,draw] at (2,-2.625) {initialize} ;
%    \draw [line width=0.5pt] (0.5,-3) rectangle (3.5,-2.25) node [align=center] at (2,-2.625) {initialize} ;
    
    \draw [arrow] (4.4,0) -- ++(0,-0.625) -- ++(0.35,0) ; 
    \draw [arrow] (4.4,-0.625) -- ++(0,-1) -- ++(0.35,0) ; 
    \draw [arrow] (4.4,-1.625) -- ++(0,-1) -- ++(0.35,0) ; 
    \draw [arrow] (4.4,-2.625) -- ++(0,-1) -- ++(0.35,0) ; 
    \draw [line width=0.5pt] (4.75,-1) rectangle (7.75,-0.25) node [align=center] at (6.25,-0.625) {rigorous} ; 
    \draw [line width=0.5pt] (4.75,-2) rectangle (7.75,-1.25) node [align=center] at (6.25,-1.625) {flowrates} ; 
    \draw [line width=0.5pt] (4.75,-3) rectangle (7.75,-2.25) node [align=center] at (6.25,-2.625) {iterate} ; 
    \node [rectangle,minimum height=0.75cm,minimum width=3cm,fill=AVTLightBlue,draw] at (6.25,-3.625) {initialize} ; 
%    \draw [line width=0.5pt] (4.75,-4) rectangle (7.75,-3.25) node [align=center] at (6.25,-3.625) {initialize} ; 
    
    \draw [arrow] (8.65,0) -- ++(0,-0.625) -- ++(0.35,0) ; 
    \draw [arrow] (8.65,-0.625) -- ++(0,-1) -- ++(0.35,0) ; 
    \draw [line width=0.5pt] (9,-1) rectangle (12,-0.25) node [align=center] at (10.5,-0.625) {rigorous} ;
    \node [rectangle,minimum height=0.75cm,minimum width=3cm,fill=AVTLightGreen,draw,line width=0.5pt ] at (10.5,-1.625) {initialize} ; 
%    \draw [line width=0.5pt] (9,-2) rectangle (12,-1.25) node [align=center] at (10.5,-1.625) {initialize} ; 
    
    \draw [arrow] (12.9,0) -- ++(0,-0.625) -- ++(0.35,0) ; 
    \draw [arrow] (12.9,-0.625) -- ++(0,-1) -- ++(0.35,0) ; 
    \draw [line width=0.5pt] (13.25,-1) rectangle (16.25,-0.25) node [align=center] at (14.75,-0.625) {rigorous} ; 
    \draw [line width=0.5pt] (13.25,-2) rectangle (16.25,-1.25) node [align=center] at (14.75,-1.625) {initialize} ; 
    
    \draw [line width=0.5pt] (15.25,-5.5) rectangle (19.25,-6.5) node [align=center] at (17.25,-6) {Murphee efficiency} ;
    \draw [line width=0.5pt] (15.25,-7.5) rectangle (19.25,-8.5) node [align=center] at (17.25,-8) {three phase} ;
    
    \node [rectangle,minimum height=1cm,minimum width=4cm,fill=AVTLightBlue,draw] at (-0.75,-6) {concentration profile} ;
%    \draw [line width=0.5pt] (1.25,-5.5) rectangle (-2.75,-6.5) node [align=center] at (-0.75,-6) {concentration profile} ;
    \draw [arrow] (-2.6,-6.5) -- ++(0,-0.625) -- ++(0.35,0) ;
    \draw [line width=0.5pt] (-2.25,-6.75) rectangle (1.0,-7.5) node [align=center] at (-0.625,-7.125) {constant} ;
    \draw [arrow] (-2.6,-7.125) -- ++(0,-1) -- ++(0.35,0) ;
    \draw [line width=0.5pt] (-2.25,-7.75) rectangle (1.0,-8.5)  node [align=center] at (-0.625,-8.125) {linear} ;
    \draw [arrow] (-2.6,-8.125) -- ++(0,-1) -- ++(0.35,0) ;
    \draw [line width=0.5pt] (-2.25,-8.75) rectangle (1.0,-9.5)  node [align=center] at (-0.625,-9.125) {TP-flashes} ;
    \draw [arrow] (-2.6,-9.125) -- ++(0,-1) -- ++(0.35,0) ;
    \draw [line width=0.5pt] (-2.25,-9.75) rectangle (1.0,-10.5)  node [align=center] at (-0.625,-10.125) {inactive} ;
    \draw [line width=0.5pt] (-2.75,-6.5) -- ++(0,-4.5) -- ++(4,0) -- ++(0,4.5) ;
    
    \node [rectangle,minimum height=1cm,minimum width=4cm,fill=AVTLightGreen,draw] at (3.75,-6) {temperature profile} ;
%    \draw [line width=0.5pt] (5.75,-5.5) rectangle (1.75,-6.5) node [align=center] at (3.75,-6) {temperature profile} ;
    \draw [arrow] (1.9,-6.5) -- ++(0,-0.625) -- ++(0.35,0) ;
    \draw [line width=0.5pt] (2.25,-6.75) rectangle (5.5,-7.5) node [align=center] at (3.875,-7.125) {constant} ;
    \draw [arrow] (1.9,-7.125) -- ++(0,-1) -- ++(0.35,0) ;
    \draw [line width=0.5pt] (2.25,-7.75) rectangle (5.5,-8.5)  node [align=center] at (3.875,-8.125) {linear} ;
    \draw [arrow] (1.9,-8.125) -- ++(0,-1) -- ++(0.35,0) ;
    \draw [line width=0.5pt] (2.25,-8.75) rectangle (5.5,-9.5)  node [align=center] at (3.875,-9.125) {non-linear} ;
    \draw [arrow] (1.9,-9.125) -- ++(0,-1) -- ++(0.35,0) ;
    \draw [line width=0.5pt] (2.25,-9.75) rectangle (5.5,-10.5)  node [align=center] at (3.875,-10.125) {custom} ;
    \draw [line width=0.5pt] (1.75,-6.5) -- ++(0,-4.5) -- ++(4,0) -- ++(0,4.5) ;
    
    \draw [line width=0.5pt] (6.25,-5.5) rectangle (10.25,-6.5) node [align=center] at (8.25,-6) {hydraulics} ;
    \draw [arrow] (6.4,-6.5) -- ++(0,-0.625) -- ++(0.35,0) ;
    \draw [line width=0.5pt] (6.75,-6.75) rectangle (10,-7.5) node [align=center] at (8.25,-7.125) {trayed} ;
    \draw [arrow] (6.9,-7.5) -- ++(0,-0.625) -- ++(0.35,0) ;
    \draw [line width=0.5pt] (7.25,-7.75) rectangle (9.75,-8.5)  node [align=center] at (8.5,-8.125) {calculated} ;
    \draw [arrow] (6.9,-8.125) -- ++(0,-1) -- ++(0.35,0) ;
    \draw [line width=0.5pt] (7.25,-8.75) rectangle (9.75,-9.5)  node [align=center] at (8.5,-9.125) {constant} ;
    \draw [arrow] (6.4,-7.125) -- ++(0,-3) -- ++(0.35,0) ;
    \draw [line width=0.5pt] (6.75,-9.75) rectangle (10,-10.5)  node [align=center] at (8.25,-10.125) {structured} ;
    \draw [arrow] (6.9,-10.5) -- ++(0,-0.625) -- ++(0.35,0) ;
    \draw [line width=0.5pt] (7.25,-10.75) rectangle (9.75,-11.5)  node [align=center] at (8.5,-11.125) {calculated} ;
    \draw [arrow] (6.9,-11.125) -- ++(0,-1) -- ++(0.35,0) ;
    \draw [line width=0.5pt] (7.25,-11.75) rectangle (9.75,-12.5)  node [align=center] at (8.5,-12.125) {constant} ;
    \draw [line width=0.5pt] (6.25,-6.5) -- ++(0,-6.5) -- ++(4,0) -- ++(0,6.5) ;
    
    \draw [line width=0.5pt] (10.75,-5.5) rectangle (14.75,-6.5) node [align=center] at (12.75,-6) {costing \& sizing} ;
    \draw [arrow] (10.9,-6.5) -- ++(0,-0.625) -- ++(0.35,0) ;
    \draw [line width=0.5pt] (11.25,-6.75) rectangle (14.5,-7.5) node [align=center] at (12.875,-7.125) {trayed} ;
    \draw [arrow] (10.9,-7.125) -- ++(0,-1) -- ++(0.35,0) ;
    \draw [line width=0.5pt] (11.25,-7.75) rectangle (14.5,-8.5)  node [align=center] at (12.875,-8.125) {structured} ;
    \draw [line width=0.5pt] (10.75,-6.5) -- ++(0,-2.5) -- ++(4,0) -- ++(0,2.5) ;
    
\end{tikzpicture}

            \caption{Distillation column model structure including sub-models and CASE branches.}
            \label{fig:mathpro:modelstruct}
        \end{figure}
    \end{landscape}
    \todo{include initial specifications in model structure}

    The structure of the developed model is depicted in \figref{fig:mathpro:modelstruct}. The main block denote separate model
    units that can, with the exception of the main model be activated or deactivated. The connections inside the main blocks
    denote selectors along with their possible branches.

    The selective deactivation of sub-models is done by creating sub flowsheets and then assigning array of a respective model
    rather than a specific entity. By assigning an array size of zero, the model is effectively deactivated. This is done to reduce
    the number of variables and computational effort, if certain model capabilities are superfluous. for the rest of this section
    various model branches will be briefly elaborated upon. Within the main model, all branches labeled ''rigorous'' correspond
    to the equations as presented in \secref{sec:mathpro:steady:distmodel} and \secref{sec:mathpro:dynamic}. For the ''flowrates''
    branches the mass balance has been written in terms of component flowrates rather than total flowrates and mole fractions.
    The ''initilaize'' branches of the material, equilibrium and energy selectors rely on the concentration profile and temperature
    profile sub-models to generate values, which are assigned straight forward to the respective variables. The reason those guesses
    were not directly implemented in the main model was to retain a level of flexibility of implementing alternative approaches
    without altering the general model structure. The meaning of the ''iterate'' branch within the equilibrium selector
    will be discussed in further detail in the next section.

    The profile sub-models give several choices for initial profiles. In the case of the concentration profiles, the edge points
    will always be determined by a TP or vapour fraction flash of the mixed feed. While for temperature profile model the
    edge points can be either bubble and dew point temperature of the mixed-feed, or the feed temperatures of the highest
    (in terms of feed location) and lowest feed.

    Having variable pressure along the column, sometimes greatly complicates initial convergence. Therefore can all hydraulics
    models be turned inactive, in which case they will return a constant previously assigned pressure drop for each stage.

    The sub-model ''three-phase'' somewhat emulates calculations for the case when to liquid phases are present,
    as it needs to be ensured, that no wrongful information about the phase equilibrium is attained b y
    the model, which can for instance cause severe errors in the mass balance, while the overall model
    might still converge. It is simply assumed, that the liquid phases, while separate, are distributed
    in such a way, that the liquid phase might still be treated as a single phase. Hence in that case,
    the concentrations and enthalpies of both phases are mixed, while information about both phases is stored
    in appropriate variables.

\subsection{Initialization procedures}
    \todo{include gPROMS code for IP's}
    \todo{mention TP flashes branch of concentration sub-unit similar to adiabatic column, as all streams are in equil..}

    This section deals with different initialization strategies for the distillation columns. Before going
    deeper into that topic, it needs to be pointed out, that physical property estimation in \gproms is done
    by means of so called foreign objects. Those are external objects, to which function calls along with
    the corresponding parameters can be passed and a return value or vector is returned. These calls range
    from simple properties like molecular weight of components to flash calculations in many variations.

    As an initial remark it should be noted, that for all initialization procedures, if the hydraulic
    sub-model is activated, first an initial profile is assumed, and after the model has converged,
    the selector within the hydraulics sub-model is switched to the calculated pressure drops. No problems
    with the hydraulics models were encountered, if the specified column geometry allowed for feasible
    column operations.

    \subsubsection{Initial specifications}
        One issue that comes up when giving specifications for a column is, that not all specifications
        are compatible with the the used initialization procedure. Specifically this is the case for
        specified duties purities or temperatures as they cannot be considered in the initialize branches.

        For those cases it is necessary to supply alternative specifications, namely a reflux or boil-up ratio
        or any flowrate, which can be fulfilled during initial calculations. Additionally, if ''extreme''
        specifications are given such as very high purities, or extremely low flowrates, convergence might
        with a given procedure might fail. One strategy, that has proven fruitful in those cases, was to
        supply a more moderate specification, initialize and converge the model up to the rigorous
        branches and then switch to the more complex specification. Especially when the switch is made in the
        rigorous branch, chances are hight, that the solution, if a MOVE\_TO switch is employed, will only
        pass through feasible regions and a final solution can be attained.

        These initial specification can be used in conjunction with all initialization procedures discussed
        in the following sections.

    \subsubsection{Flash initialization}
        The flash initialization, is the simplest, but nevertheless a very effective and efficient way
        of initializing the model. As initial guess constant profiles are assigned for vapour and liquid
        concentrations taken from the aforementioned feed flash. For the temperatures a linear profile
        between dew and bubble point temperature of the feed flash is used. The vapour and liquid
        flowrates are computed from the constant molar overflow assumption while considering
        sub-cooled or super-heated feed streams. In terms of the model this strategy translates into
        starting all selectors in the initialize branches and then simultaneously switching all selectors
        to the rigorous branches.

    \subsubsection{Linearized material balance}
        As discussed in the example given in \secref{sec:mathpro:steady:distmodel} this initialization
        procedure relies on on first initializing the model as for the flash initialization. Afterwards,
        in an intermediate step, the enthalpy and summation branch are kept in the initialize branch,
        which means, the initial temperature profile is kept fixed, and the constant molar overflow assumption
        retained. The material and equilibrium selectors are then switched to the flowrates branches, where
        a linearized material balance is solved with constant equilibrium ratios and molar flowrates.
        The constant equilibrium ratios are enforces, by passing the values from the concentration profile
        sub-model to the physical property calls, rather that the actual computed values for molar
        concentrations.

        The main reason for formulating this mass balance in terms of component flowrates is, that due to
        the assumptions a solution will yield discrepant results. Especially if the equilibrium ratios
        in the converged solution span a large range of values. Due to these discrepancies some
        molar fractions might even become negative. As the corresponding variable type is bounded to zero
        or a small negative number, this might cause the solver to fail during convergence of this step.

        Furthermore the molar fractions would also have to be re-normalized, as the summation to one
        is not enforced in this case.

        Afterwards all selectors are again switched to the rigorous branch.

\begin{lstlisting}
INITIALISATION_PROCEDURE
  START
    material    := initialize ;
    equilibrium := initialize ;
    energy      := initialize ;
    summation   := initialize ;
    side_scale  := 0 ;
    hydraulics  := constant ;
  END
  NEXT
    material    := flowrates ;
    equilibrium := flowrates ;
  END
  NEXT
    REVERT material ;
    REVERT equilibrium ;
    REVERT energy ;
    REVERT summation ;
  END
  NEXT 
    MOVE_TO
      side_scale := 1 ;
    END
  END
  NEXT 
    REVERT hydraulics ;
  END
\end{lstlisting}



    \subsubsection{Iterative approach}
        Previously it has been said, that the inside-out algorithm has proven very robust when
        it comes to solving complex distillation problems. However it is based on an modular
        iterative approach which is incompatible with equation based process simulators.

        One of the main aspects is to construct simplified models for enthalpies and the equilibrium,
        which only give accurate results for the point they have been constructed at, but are much
        better behaved, than rigourous enthalpy and equilibrium correlations. In eat step of the iteration
        a new model for the physical properties is constructed, at the point, where the previous model
        evaluation with the previous physical properties model converged to. The iterative process terminates,
        when no further (or very little) changes occur in the newly constructed parameters for the
        new physical property models.

        In order to mimic such an approach in \gproms, the branch ''iterate'' within the equilibrium
        selector has been created. There the physical properties are called with an new set of concentration
        variables, which are assigned arbitrary values in all other branches. To do the iterations anther
        structure of \gproms is used. So called TASKS. With that it becomes possible, to converge
        a the model, with a given concentration stored in these variables, and after convergence, replace those
        values with the current values of liquid and vapour mole fractions.

        The drawback however, is that the number of iterations to be carried out needs to be specified in
        advance. This strategy, while rather time consuming due to large overhead in computational time
        during execution of the TASK, has generated very promising results and with it it was possible
        to converge columns with very complex multi-component systems.
\begin{lstlisting}
SEQUENCE
  i := 1 ;
  WHILE i < number_of_iterations DO
    SEQUENCE
      PARALLEL
        REASSIGN
          column_section.x_iterate(,) := OLD(column_section.x(,));
          column_section.y_iterate(,) := OLD(column_section.y(,));
        END
        SWITCH
          column_section.phase_equilibrium
            := column_section.iterate_to_rigorous;
        END
      END
      i := i + 1 ;
    END
  END
  SWITCH
    column_section.phase_equilibrium := column_section.rigorous;
  END
END
\end{lstlisting}
    \subsubsection{Side draws}
        During construction of process flowsheets and application of the developed models it turned
        out, that columns with side draws add extra complexity in terms of initialization. To accommodate that
        fact side draw versions of the previously described initialization procedures have been developed.
        The only difference is, that the side draws are initially set to zero once convergence has been
        achieved, they are again set to their specified values. Either by an instant change or again
        in a continuous fashion.
