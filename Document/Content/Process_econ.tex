As discussed earlier economic consideration play a major role in process design. In order to account
for the process economics the cost of the process to be implemented needs to be estimated at the design
level. However as limited information is available estimation methods have to be employed. In
\secref{chp:ProcesEconomics} the general approach for cost estimation of process equipment was
introduced, where a specific value such as heat-exchange area or vessel size is used to approximate
equipment cost. However for more specific units extended models are available, where statistical
data is employed to yield a more realistic fit to cost data. The cost functions and correction
factors presented in this chapter are, if not stated otherwise, taken from \cite{Seider.2010}.
Also unless otherwise stated the unit cost is given for the year 2006 ($CE = 500$).
	
\subsection{Destillation column}
    The cost of a given distillation column $C_{column}$ is comprised of three main parts, the cost 
    for the vessel or tower itself $C_{tower}$, the cost for platforms, ladders, manholes, and nozzles
    $C_{platform}$ and the cost for the internals $C_{internals}$
    \Eq{}{
        C_{column} = C_{tower} + C_{platform} + C_{internals}. 
    }
    \ncr{C_{column}}{column cost}{\$}
	\ncr{C_{tower}}{tower cost}{\$}
    \ncr{C_{platform}}{platform cost}{\$}
    \ncr{C_{internals}}{cost of column internals}{\$}
	
    The determining factors for the cost of a tower are the construction material and the weight of the vessel. 
    The material is considered by multiplying a base cost by a material factor $f_M$. Thus the cost for 
    the tower can be approximated by
    \Eq{eq:cost:column:vessel}{
	   C_{tower} = f_M \cdot \exp\left[ 7.2756 + 0.18255 \cdot \ln(W_{tower}) + 0.02297 \cdot \pow{\ln(W_{tower})}{2} \right], 
            \eqannote{9000 \leq W_{tower} \leq 2.5 \cdot 10^6}.
	}
    \ncr{W_{tower}}{weight of  a tower}{lbs}
    In this correlation the weight $W$ is measured in pounds ($[lbs]$)
    
    For the support structures the following equation has bee presented 
	\Eq{eq:cost:column:support}{
	   C_{platform} = 300.9 \cdot \pow{d_{column}}{0.63316} \cdot \pow{l_{column}}{0.80161},
	}	
    where both the diameter $d_{column}$ and length $l_{column}$ are measured in feet ($[ft]$). 
	
    The weight of the column is attained by calculation its volume and multiplying it by the 
    mass density of the construction material $\tilde{\varrho}$
    \Eq{eq:cost:column:weight}{
	   W_{column} = \pi (d_{column} + t_s)(l_{column} + 0.8 \cdot d_{column}) t_s \cdot \tilde{\varrho}.
	}	
    \ncr{t_s}{tower shell thickness}{in}
    It needs to be noted, that in this formula, all measurements have to be adjusted to the measurement 
    of the density. Newly appearing is the shell thickness $t_s$ of the tower. In order to compute 
    this, several aspects can be considered. 
    
    The ASME pressure vessel code formula is used to compute the minimum thickness due to the design 
    pressure $p_D$ of a tower
    \Eq{eq:cost:column:wallthickness}{
		t_p = \frac{p_D \, d_{column}}{2 \, S \, E - 1.2 p_D},
	}
    where $S$ denotes the maximum allowable stress and $E$ the fractional weld efficiency. 
    \ncr{S}{maximum allowable stress}{psi}
    \ncr{E}{fractional weld efficiency}{-}
    \ncr{t_p}{shell thickness die to pressure}{in}
    
    The design pressure is computed from the operating pressure $p_O$
    \Eq{eq:cost:column:designpressure}{
		p_D = \max\left\{10,\exp\left[ 0.60600 + 0.91615 \cdot \ln(p_O) + 0.0015655 \cdot \left(\ln(p_O)\right)^2 \right]\right\}, 
	}
    \ncr{p_D}{design pressure}{psig}
    \ncr{p_O}{operating pressure}{psig}
    It should be pointed out the the pressures in the previous equation are measured in pound per square inch 
    gauge ($[psig]$). 
    
    In addition to the thickness due to pressure, it should be considered -- especially for tall vessels -- 
    that the vessel might have to withstand external forces such as strong winds or earthquakes. 
    To account for external effects a security addition  $t_w$ to the pressure thickness can be computed 
    \Eq{}{
        t_w = \frac{0.22 (d_{column} + 18)l_{column}^2}{S d_{column}} . 
    }
    \ncr{t_w}{thickness wind and earthquake correction}{in}
    
    Therefor the total shell thickness amounts to 
    \Eq{}{
        t_s = t_p + t_w. 
    }
    
    In the previous equations the column diameter and length have frequently been used, however no mention 
    has been made as to how to attain these values. While for a given design these values will be fixed, 
    in the context of an optimization they need to be considered as decision variables. These decisions 
    are closely linked to the choice of column internals. In case of the column height the procedure 
    is very similar for columns with trays and structured packings. In case of the diameter however 
    a clear distinction has to be made. This is due to the fact, that a decision regarding the diameter 
    will have to ensure feasible column operations, which means it will have to be large enough to 
    avoid flooding within the column. Consequently the procedures for determining height and diameter 
    will be dealt with separately for each type of internals. 
    
    \subsubsection{Trays}
        For a trayed column the height can be determined from the number of stages $n_S$ used multiplied by 
        the plate spacing $h_{plate}$. In addition however it needs to be ensured, that control of the 
        columns is still possible. At the sump of the tower liquid will accumulate during operations. 
        While in an ideal case the liquid level would be constant, that cannot be assumed. Regularly 
        there are three different liquid levels defined in a column. The high level (HLL), normal 
        level (NLL) and the low level (LLL). These levels are defined such that it sufficient time 
        for the liquid to reach these levels ($t_{min}^{HLL}, t_{min}^{NLL}, t_{min}^{LLL}$) , 
        if no liquid id withdrawn anymore, or no liquid comes down from the column. Furthermore 
        the times for reaching the reboiler inlet and the lowest plat will ne necessary. 
        What duration to reach those levels will be sufficient needs to be decided by a control engineer. 
        
        With these times and heights the height of the tower can be expressed as 
        \Eq{}{
            l_{column} = n_S \cdot h_{plate} + \left(\sum_i t^{i}_{min} H^{i} \right) \cdot 
                \frac{L_{n_S}}{A_{column} \varrho^L}. 
        } 
        \ncr{h_{plate}}{plate spacing}{m}
        
        This leaves the diameter to be determined. The most important factor to that regard is the vapour velocity 
        within the tower. It is to be chosen such that no flooding or entrainment will occur. The following equation 
        to compute these operation boundaries are taken from \cite{Lygeros.1986}. 
        
        First the fractional entrainment factor $ent_j$ for 80\% flooding at each stage $j$ needs to be calculated 
        \Eq{}{
            ent_j = 2.24 \cdot 10^{-3} \cdot 2.377 \exp\left[ -9.394 \cdot FLV_j^{0.314}\right], \eqanns . 
        }
        \ncr{ent_j}{fractional entrainment for 80\% flooding on stage $j$}{-}
        
        Therein the Sherwood flow parameter $FLV_j$ is used
        \Eq{}{
            FLV_j = \frac{\tilde{L_j}}{\tilde{V}_j} \cdot \sqrt{\frac{\tilde{\varrho}^V_j}{\tilde{\varrho}^L_j}}, \eqanns. 
        }
        \ncr{FLV_j}{Sherwood flow parameter on stage $j$}{-}
        
        \Eq{}{
            v_j^{flood} = 60 \cdot \left( \frac{\sigma_j^L}{20} \right) \cdot K1_j \cdot 
                \sqrt{\frac{\tilde{\varrho}_j^L - \tilde{\varrho}_j^V}{\tilde{\varrho}^V_j}}, \eqanns. 
        }
        \ncr{v_j^{flood}}{flooding velocity on stage $j$}{\frac{m}{s}}
        \todo{check time constant (60) for flooding velocity}
        
        \Eq{}{
            K1_j = 1.05 \cdot 10^{-2} + 0.1496 \cdot h_{plate}^{0.755} \cdot \exp\left[ -1.463 \cdot FLV_j^{0.842} \right], \eqanns. 
        }
        \ncr{K1}{empirical coefficient}{-}
        
        With those values the minimum required column area for each stage $A_j^{min}$ can be calculated
        \Eq{}{
            A_j^{min} = \frac{V_j}{\Psi_{flood} \varrho_j^V v_j^{flood}}, \eqanns, 
        }
        \ncr{A_j^{min}}{minimum column area on stage $j$}{m^2}
        \ncg{\Psi_{flood}}{fractional allowed flooding}{-}
        where $\Psi_{flood}$ is a design parameter which determines the degree of flooding allowed and will usually 
        have a value around $0.8$. 
     
    \subsubsection{Structured packings}
        The calculation of the column height is equivalent to the trayed case, with the one difference, that 
        rather than using the plate spacing as the height for each stage a value called height equivalent 
        to to theoretical plate ($HETP$) ist used. In terms of the column diameter the flooding velocity 
        is determined by the flooding pressure drop as calculated in \secref{sec:mathpro:dynamic:strpackhyd}. 
        \todo{check results for minimum column diameter when using flooding pressure drop, or fitted flooding velocity}

\subsection{Centrifugal pump}
		Pumps are among the most common units of process equipment. While there are several different
		kinds of pumps that can be used, the centrifugal pump is one of the most popular choices and
		denotes a very likely choice for the process conditions considered in this application. Hence
		other pump types will not be considered at this point.
		
		\paragraph{Pump}
			In terms of operations pumps are best described by the volumetric flow transported $Q$ as
			well as the pump head $H$, the hight that needs to be overcome. Data taken from the company
			Mosanto was used to correlate the pump cost to a specific value
			\Eq{eq:cost:pump:SpecVal}{
				S = Q \sqrt{H}.
			}
			As a reference unit the base price $C_B$ is estimated for a cast iron single-stage
			vertically split case at 3600 $rpm$
			\Eq{eq:cost:pump:PumpWOMotor}{
				C_B = \exp \left\{ 9.7171 - 0.6019 \cdot \ln[S] + 0.0519 (\ln[S])^2 \right\},
					\eqannote{400 \leq S \leq 100000}.
			}
			
			The most influential addition factors for the pump price are the material, which is accounted
			for in the material factor $f_m$, as well as the rotation, case split orientation (horizontal
			and vertical), the number of stages, covered flow rate range, pump head range and maximum
			motor power, which are all agglomerated in the type factor $f_T$. Values for these factors
			are given in \tabref{tab:pump:Type} and \tabref{tab:pump:Material}.
			
			\stdtab{Tables/PumpFactorsType}{Pump type factors \cite{Seider.2010}.}{tab:pump:Type}
			\stdtab{Tables/PumpFactorsMaterial}{Pump material factors \cite{Seider.2010}.}{tab:pump:Material}
			
		\paragraph{Electric motor}
			Separately from the pump itself the motor to drive the compression is considered. While the
			volumetric flow and the pump head certainly are valid choices to correlate motors for pumps
			especially, the power consumption is a more general specific value
			\Eq{eq:cost:pump:MotorPowerConsumption}{
				P_C = \frac{P_T}{\eta_P \eta_M} = \frac{P_B}{\eta_M}
			}
			
			It can be calculated from the theoretic power of the pump $P_T$ and the efficiencies $\eta_P$
			$\eta_M$. While an estimate for the expected power consumption might be already available at
			rather early design stages, the efficiencies will have to be correlated as well if resorting
			to average values is considered too coarse. Those correlations rely on the volumetric flow in
			gallons per minute ([gpm]) and the brake horse power $P_B = \frac{P_T}{\eta_P}$.
			
			\Eq{eq:cost:pump:MotorEfficiancyP}{
				\eta_P = -0.316 + 0.24015 \cdot \ln[Q] - 0.01199 \cdot (\ln[Q])^2 \eqannote{50 \leq Q \leq 5000}
			}
			\Eq{eq:cost:pump:MotorEfficiancyM}{
				\eta_M = 0.80 + 0.0319 \cdot \ln[P_B] - 0.00182 \cdot (\ln[P_B])^2 \eqannote{1 \leq P_B \leq 1500}
			}
			
			After having calculated the power which the motor needs to supply its base cost of an open,
			drip-proof enclosed motor at 3600 $rpm$ can be approximated by
			\Eqml{eq:cost:pump:Motor}{
				C_B = \exp\big\{ 5.8259 + 0.13141 \cdot \ln[P_C] + 0.053255 \cdot (\ln[P_C])^2 \\
					+ 0.028628 \cdot (\ln[P_C])^3 - 0.0035549 \cdot (\ln[P_C])^4 \big\} \eqannote{1 \leq P_C \leq 700}
			}
			
			To adjust the cost for different types of electric motors the type factors from \tabref{tab:pump:MotorTypes}
			
			\stdtab{Tables/MotorTypeFactors}{Type factors for different motor types.}{tab:pump:MotorTypes}
	
	\subsubsection{Compressor}
		The cost of compressors is correlated with their respective power consumption measured in horsepower.
		Although not the most efficient type of compressor, centrifugal compressors are very popular in the
		process industry, as they are easily controlled an deliver a very steady flow. However as different
		types might be employed as well base cost correlations for centrifugal, reciprocation and screw
		compressors are given.
		
			\paragraph{Centrifugal compressor}
				\Eq{eq:cost:compressor:centrifugal}{
					C_B = \exp\big\{ 7.5800 + 0.80 \cdot (\ln[P_C]) \big\} \eqannote{200 \leq P_C \leq 30000}
				}
			
			\paragraph{Reciprocating compressor}
				\Eq{eq:cost:compressor:centrifugal}{
					C_B = \exp\big\{ 7.9661 + 0.80 \cdot (\ln[P_C]) \big\} \eqannote{200 \leq P_C \leq 20000}
				}
				
			\paragraph{Screw compressor}
				\Eq{eq:cost:compressor:centrifugal}{
					C_B = \exp\big\{ 8.1238 + 0.7243 \cdot (\ln[P_C]) \big\} \eqannote{200 \leq P_C \leq 750}
				}
		
		Again as with most other equipment types correction factors are used to adjust for different realization
		of this piece of equipment. Here type of motor as well as the construction material have the biggest
		effects on the unit price and are explicitly considered.
		\Eq{eq:cost:sompressor:factors}{
			C_p = f_D \, f_M \, C_B
		}
		
		The alternatives to the electric motor ($f_D = 1.0$) are a steam turbine ($f_D = 1.15$) or a gas turbine
		($f_D = 1.25$). It should however be noted that aside from being the cheapest choice, the electric motor
		is also the most efficient. Thus the turbines are mostly considered, when process steam or combustion gas
		is easily available, such that the drawbacks might be eliminated by not having to supply the electric
		energy for the electric motor. In terms of construction material all base costs are for cast iron or
		carbon steel. Some appliances may require more resistant and also more expensive materials such as
		stainless steel ($f_M = 2.5$) or an nickel alloy ($f_M = 5.0$).
		
	\subsubsection{Reboiler / condenser}
		Reboiler and condenser can be characterized as heat exchangers, and be handled in the same way,
		as the main difference is weather heat is transferred to or from the process stream. In that sense
		they must be distinguished when considering the operating cost, as the cost for hot or cold
		auxiliary streams might differ significantly. As customary for heat exchangers the specific
		quantity for cost correlations is the necessary heat exchange area $A$ measured in $ft$.
		
		Again the construction material as well as the operating conditions have an effect on the
		final cost
		\Eq{eq:cost:condreb:total}{
			C_p = f_P \, f_M \, C_B.
		}
		
		The correction for pressures $f_P$ takes into account the operating pressure $P_o$ and
		is computed by
		\Eq{eq:cost:condreb:pcorrect}{
			f_P = 0.8510 + 0.1292 P_o + 0.0198 * P_o^2.
		}
		
		The material correction factor $f_M$
		\Eq{}{
			f_M =
		}
		
		\paragraph{Shell and tube heat exchanger}
		\Eq{eq:cost:condreb}{
			C_B = \exp\big\{ 11.667 - 0.8709 \cdot \pow{\ln[A]}{} + 0.09005 \cdot \pow{\ln[A]}{2} \big\}
		}
		
		\paragraph{Double pipe}
		\Eq{}{
			C_B = \exp\left\{ 7.146 + 0.1600 \cdot \pow{\ln[A]}{} \right\}
		}