    Aside from the actual model equations care must be taken as to how to implement any set of equations,
    such that a numerical solver can find feasible solutions. The most prominent issues one is faced with in
    that context are scaling and singularities within the model. Both aspects will be briefly discussed here.

        \subsection{Scaling}
        When evaluating a model, or integrating over a given time span, Taylor expansion is used to construct
        an approximate linear process model. This means, that a system of the form
        \Eq{eq:lin_sys}{
            \vec{A} x = b,
        }
        is solved to a specified absolute accuracy $\varepsilon$. For a single equation the choice of an appropriate
        value for $\varepsilon$ is rather simple, given that the magnitude of $\vec{A} x$ and $b$ is known.
        Here one needs to consider, that if either magnitude is much grater than $\varepsilon$ a large number
        of iterations will be necessary to solve the system in question. On the other hand, if they are of similar
        magnitude relative error in the attained solution might become unacceptably high.

        To address these issues it is wise to consider the model scaling while implementing the equations.
        One approach would be to introduce scaling factors on both sides of an equation. While this
        may certainly improve scaling for the equation itself, the opposite might be true for the respective derivative.
        Consider the following simple example \cite{MarkPinto.2008} where to holdups are added to a total holdup
        \Eq{}{
            f(M_i) = M_1 + M_2 = M_T
        }
        Assuming both holdups are of magnitude $M_1 = M_2 = 10^5kg$, then the function value is $2 \cdot 10^5$, while for
        the derivative
        \Eq{}{
            \fracddpart{f}{M_1} = \fracddpart{f}{M_2} = \fracddpart{f}{M_T} = 1
        }
        holds. Introducing a scaling factor of $10^{-5}$ to both sides of the equation would lead to a function value
        of $2$ but have an adverse effect on the derivatives $\fracddpart{f}{M_i} = 10^{-5}$.

        A more robust approach would be to scale the units of the given equation which can simultaneously improve
        scaling for the function and its derivatives. Writing the same equation in tonnes would lead to  function value of
        $200$ while maintaining the well scaled derivatives. Alternatively one can introduce further variables and equations
        with the aim to get more linear equation and rescale some parts of the original equation. A more linear function
        leads to more constant Jacobian elements and reduces the need for Jacobian updates which can be very expensive in
        computational terms.
        
        When it comes to the dynamic models, the issue of scaling is extend with another aspect. While for the steady-state
        models, one could rewrite each equation with different units, as long as each equation is in itself consistent, 
        the same cannot be said, for the dynamic models. Here it is imperative, to maintain the same time scale for every 
        dynamic equation. Due to the very large flow-rates, the steady-state and dynamic balances were written in 
        $\frac{kmol}{hr}$ to achieve a well scaled equation. Considering the various dynamic effects discussed in 
        the previous section, this seems to coarse to capture all relevant phenomena. Maintaining one time unit as one 
        hour would lead to integrator steps of $\frac{1}{3600}$ for one second. Considering furthermore the small changes 
        in the model variables, intuitively this would lead to very badly scaled problems. To somehow accommodate these 
        considerations, it was decided, to introduce a scaling factor solely on the left hand side of the dynamic equations. 
        By those means, the time scale of the process model can easily be adjusted for seconds, minutes, hours or days.  

        \subsection{Singularities}
        Singularities in the models considered here often occur, when a variable is raised to a non-integer power,
        or the logarithm of a variable is evaluated.
        Expressions like have no solution within the rational domain, when the respective variable becomes negative.
        Furthermore when negative exponents are involved a value of zero would lead to a division by zero which is
        undefined. Even before that the function value might become very large as the variable approaches zero.

        In the context of a distillation column, this issue becomes especially relevant, when optimizations or simulations
        with inactive trays are carried out. Some flowrates in the inactive section of the model will be zero, depending
        on whether the top or bottom reflux is optimized, those will be either vapour or liquid flowrates. While a
        solution of zero might cause issues in some cases, a negative solution introduces even more complications. As the
        equations are solved within numerical accuracy, this might also lead to (slightly) negative solutions, which will
        cause the numerical solver to fail. Such scenarios have to be anticipated while implementing the model and
        appropriate measures have to be taken to ''protect'' such variables from hindering model evaluation.

        As \gproms supports state transition networks, one way is to introduce conditional statements. However
        those statements will in almost all cases lead to discontinuities in the model. While in principal such
        discontinuities can be detected by numerical solvers \cite{Pantelides.2003}, they are a further source
        for instabilities. Hence they should in the best case only be introduces, when there is a physical correspondence
        in the system behaviour.

        In this thesis the problem was approached by decoupling the flowrates within the column, and the ones used for evaluation
        of the hydraulic model. With the help of the split fractions active and inactive trays can be distinguished.
        \Eq{}{
            V_j^{\text{hyd}} = \left( \sum_{k=1}^j \zeta^R_k \right) \cdot V_j + \left(1 - \left( \sum_{k=1}^j \zeta^R_k \right)\right) \cdot V_1
        }
        Here it should be pointed out, that it only needs to be ensured, the the hydraulic model can be evaluated for inactive trays.
        What results are returned for these trays has no effect for the modelled column operation.