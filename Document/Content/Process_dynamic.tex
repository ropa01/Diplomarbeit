 \todo{explain model assumptions.}

\subsection{Distillation column model}
    The previously described column model is based on a steady state assumption. This means that all variables do not
    change with time. While a model like that offers valuable insight into the operation of a process many
    aspects remain unclear. In order to gain further insight into the process the dynamics have to be considered.

    Due to that in this section a dynamic model of the ASU process will be developed.

    First the balance equations have to be rewritten in dynamic form. To do so reservoir terms or holdups
    are introduced. Namely the component holdups $n_{ij}$ and internal energy $U_j$ for each stage.
    with that the component balance equations as presented in the previous section can now be rewritten as
        \Eqml{eq:col:CompBalance_opt}{
    		\left(1 - \sum_{k=1}^{j-1} \zeta^R_k \right) \fracdd{n_{ij}}{t} = \left(1 + s_j^V\right) 
                \cdot V_j \cdot y_{i,j} + \left(1 + s_j^L\right) \cdot L_j \cdot x_{i,j} - V_{j+1} 
                \cdot y_{i,j+1} \\ \hfill - L_{j-1} \cdot x_{i,j-1}
                - \sum_{l=1}^{n_F} \zeta_{lj} \cdot F_j \cdot z_{i,j} - \zeta^R_j \cdot V_N \cdot y_{iN}, \hfill%
                \\ \eqannote{i = 1 \dots n_C-1, \quad j = 1 \dots N, \quad k = 1 \dots n_F, \quad l = 1 \dots n_F}.
    	}%
    While the internal energy balances become
        \Eqml{eq:col:EnergyBalance}{
    		\left(1 - \sum_{k=1}^{j-1} \zeta^R_k \right) \fracdd{U_j}{t} = \left(1 + s_j^V\right) \cdot V_j \cdot h^V_{j} + \left(1 + s_j^L\right)
                \cdot L_j \cdot h^L_{j} - V_{j+1} \cdot h^V_{j+1} \\ \hfill - L_{j-1} \cdot h^L_{j-1}
                - \sum_{k=1}^{n_F} \zeta_{kj} \cdot F_k \cdot h^{F}_{j} - \zeta^R_j \cdot V_N \cdot h^V_{N}, \hfill
                \\ \eqannote{i = 1 \dots n_C, \quad j = 1 \dots n_S, \quad k = 1 \dots n_F}.
    	}%
    \ncr{n_{ij}}{holdup of component $i$ on stage $j$ }{mol}
    \ncr{U_j}{internal energy holdup on stage $j$}{J}

    In addition to the balance equations further constituent equations need to be introduced. From the
    steady state model we know the equilibrium equations
    \Eq{eq:col:Kxy}{
        y_{ij} & = K_{ij} \cdot x_{ij}, \eqanncs,
    }%
    %
    and the summation equations
    %
    \Eq{eq:col:summ_dyn}{
        1 & = \sum_i^{n_c} y_{ij} \eqanns, \\
        1 & = \sum_i^{n_c} x_{ij} \eqanns.
    }%
    %
    Furthermore the accumulation of moles in each stage in vapour $n_j^V$ and liquid $n_j^L$
    need to be considered with
    %
    \Eq{eq:col:holdup}{
        n_{ij} & = x_{ij} n_j^L + y_{ij} n_j^V \eqanncs, \\
    }%
    \ncr{n_j^L}{liquid molar holdup on stage $j$}{mol}
    \ncr{n_j^V}{vapour molar holdup on stage $j$}{mol}
    %
    These holdups are linked by the volume of a given stage $V_{stage}$. Thus the volume constraint
    can be written as
    %
    \Eq{eq:dyn:volconstraint}{
        V_{stage} = \frac{n_j^V}{\varrho_j^V} + \frac{n_j^L}{\varrho_j^L} \eqanns.
    }%
    \ncr{V_{stage}}{stage volume}{m^3}
    %
    The internal energy in a stage corresponds to its enthalpy, reduced by pressure term
    \Eq{eq:dyn:internergyholdup}{
        U_j = n^L_j \cdot h^L_j + n^V_j \cdot h^V_j - p_j \cdot V_{stage}
    }
    %
    As we are no longer dealing in steady state hydraulic equations need to be introduces,
    which determine the liquid and vapour flow rates leaving a separation stage. As the
    mechanisms driving these flows might be very different depending on the type of internals
    used, it is not surprising that the corresponding equations are also very different. In the
    given model both trayed columns and columns with structured packing are employed.

    \subsubsection{Trayed hydraulics}
    \label{sec:mathpro:dynamic:trayhyd}
        \begin{figure}
            \begin{tikzpicture}
    \pgfmathsetmacro{\colwidth}{8}
    \pgfmathsetmacro{\downwidth}{1/6*\colwidth}
    \pgfmathsetmacro{\actwidth}{4/6*\colwidth}
    \pgfmathsetmacro{\platespace}{3}
    \pgfmathsetmacro{\colheight}{3*\platespace}
    \pgfmathsetmacro{\weirheight}{0.3*\platespace}
    \pgfmathsetmacro{\downlen}{\weirheight + 0.9 *\platespace}
    \draw [thline] (0,0) -- ++(\downwidth,0) node (a) {} node (downrb) [yshift=1mm,xshift=-1mm] {} node (traylt1) [yshift=-1mm,xshift=1mm] {} ;
    \node (downlt) [xshift=1mm,yshift=-1mm] at (0,\platespace) {} ;
    \draw [dashtray] (a) -- ++(\actwidth,0) node (vapin) [pos=0.5] {} node (trayrb) [yshift=1mm,xshift=-1mm] {} node (downlt1) [yshift=-1mm,xshift=1mm] {};
    \draw [thline] ($(0,0) + 0.5*(0,\colheight)$) -- ($(0,0) - 0.5*(0,\colheight)$) node (e) [xshift=1mm, inner sep=0] {} ;
    \draw [thline] ($(\colwidth,0) + 0.5*(0,\colheight)$) -- ($(\colwidth,0) - 0.5*(0,\colheight)$) node (g) [pos=0,xshift=-1mm,inner sep=0] {} ;
    \draw [thline] (\colwidth,\platespace) -- ++(-\downwidth,0) node (b) {} node (f) [yshift=1mm,xshift=1mm,inner sep=0] {} node (h) [yshift=1mm,xshift=-1mm,inner sep=0] {};
    \draw [dashtray] (b) -- ++(-\actwidth,0) node (traylt) [yshift=-1mm,xshift=1mm] {} ;
    \draw [thline] (\colwidth,-\platespace) -- ++(-\downwidth,0) node (c) {} node (trayrb1) [yshift=1mm,xshift=-1mm] {} node (downrb1) [pos=0,yshift=1mm,xshift=-1mm] {} ;
    \draw [dashtray] (c) -- ++(-\actwidth,0) node (d) [yshift=-1mm,xshift=-1mm, inner sep=0] {} ;
    \draw [thline] ($(\downwidth,0) + (\actwidth,0) + (0,\weirheight)$) -- ++(0,-\downlen) ;
    \draw [thline] ($(\downwidth,\platespace) + (0,\weirheight)$) -- ++(0,-\downlen) ;
    \draw [thline] ($(\downwidth,\weirheight-\platespace)$) -- ($(\downwidth,-0.5*\colheight)$) ;
    \draw [thline] ($(b) + (0,0.1*\platespace)$) -- ($(\downwidth+\actwidth,0.5*\colheight)$) ;
    \draw [dashvol] (trayrb) rectangle (traylt) ;
    \draw [dashvol] (trayrb1) rectangle (traylt1) ;
    \draw [dashvol] (downrb) rectangle (downlt) ;
    \draw [dashvol] (downrb1) rectangle (downlt1) ;
    \draw [dashvol] (e) -- (e |- d) -- (d) -- (e -| d) ;
    \draw [dashvol] (g) -- (g |- f) -- (f) -- (g -| f) ;
    \draw [dashvol] (g -| h) -- (h) -- (traylt |- h) -- (traylt |- g) ;
    \node at ($0.5*(\colwidth,\platespace)$) {$n$} ;
    \node at ($0.5*(\colwidth,-\platespace)$) {$n+1$} ;

    % arrows
    \node (liq_in_start) at ($0.5*(\downwidth,0) + 0.2*(0,\platespace) $) {} ;
    \node (liq_in_end) at ($(a) + 0.2*(\actwidth,0) + 0.6*(0,\weirheight)$) {} ;
    \draw [arrow] (liq_in_start) .. controls +(-70:0.8) and +(-150:0.5) .. (liq_in_end) node [pos=0.3,left] {\scriptsize $L_{n-1}$};
    \node (liq_out_start) at ($(a) + 0.8*(\actwidth,\weirheight)$) {} ;
    \node (liq_out_end) at ($(\downwidth+\actwidth,0) + 0.5*(\downwidth,0) - 0.4*(0,\platespace)$) {} ;
    \draw [arrow] (liq_out_start) .. controls +(30:1.2) and +(90:2.5) .. (liq_out_end) ;
    \node (liq_in_start1) at ($(\actwidth+1.5*\downwidth,0) - 0.8*(0,\platespace) $) {} ;
    \node (liq_in_end1) at ($(c) - 0.2*(\actwidth,0) + 0.6*(0,\weirheight)$) {} ;
    \draw [arrow] (liq_in_start1) .. controls +(-110:0.8) and +(-30:0.5) .. (liq_in_end1) node [pos=0.3,right] {\scriptsize $L_{n}$};
    \node (liq_in_start2) at ($(\actwidth+1.5*\downwidth,0) + 1.2*(0,\platespace) $) {} ;
    \node (liq_in_end2) at ($(b) - 0.2*(\actwidth,0) + 0.6*(0,\weirheight)$) {} ;
    \draw [arrow] (liq_in_start2) .. controls +(-110:0.8) and +(-30:0.5) .. (liq_in_end2) node [pos=0.3,right] {\scriptsize $L_{n}$};
    \draw [arrow] ($(\downwidth+0.5*\actwidth,-0.8*\weirheight)$) -- ($(\downwidth+0.5*\actwidth,0.6*\weirheight)$) node [pos=0.3,right] {\scriptsize $V_{n+1}$} ;
    \draw [arrow] ($(\downwidth+0.5*\actwidth,\platespace-0.8*\weirheight)$) -- ($(\downwidth+0.5*\actwidth,\platespace+0.6*\weirheight)$) node [pos=0.3,right] {\scriptsize $V_{n}$} ;

    % bemaßung
\end{tikzpicture}

            \caption{Column tray}
            \label{fig:col_tray}
        \end{figure}
        \todo{add graphic for tray measuremets.}

        Trayed column hydraulics can be approximated by the following system of equations.
        All equations presented here were taken from \cite{Lockett.2009}.

        The liquid flow rates are calculated from the well established francis formula ,
        derived from the law of Bernoulli and taking effects like bubbling into account
        %
        \Eq{eq:col:francis}{
            L_j = \frac{2}{3} \sqrt{2 g} \varrho^L_j \ell_{weir} \Phi h_{ow}^{\frac{2}{3}} \eqanns.
        }%
        \ncc{g}{gravitational constant}{\frac{m}{s^2}}
        \ncr{\ell_{weir}}{length of tray weir}{m}
        \ncr{h_{ow}}{height over weir}{m}
        \ncg{\Phi}{relative froth density}{-}
        %
        Where $h_{ow}$ denotes the height of the liquid over the weir, which can be calculated from the
        froth height $h_f$ and the weir height $h_w$
        \Eq{}{
            h_{ow} = h_f - h_w.
        }
        \ncr{h_f}{froth height}{m}
        \ncr{h_w}{weir height}{m}
        %
        While the weir height is a tray design parameter the froth height is computed from the clear
        liquid height and the relative froth density
        \Eq{}{
            h_f = \frac{n^L \, MW^L}{A_{active} \, \varrho^L \, \Phi}.
        }
        \ncr{A_{active}}{active tray area}{m^2}
        %
        Lastly in terms of liquid flow rates, the relative froth density is dependent on the
        degree of aeration within the liquid, expressed by the aeration factor $\beta$
        from an empirical equation
        \Eq{}{
            \beta_j & = 1 - 0.3593 \left( \frac{V_{j-1} \, MW^V_{j-1}}{A_{active} \, \sqrt{\varrho_{j-1}}} \right)^{0.177709} \eqanns, \\
            \Phi_j & = 2 \beta_j - 1.
        }
        \ncg{\beta}{aeration factor}{-}

        The pressure difference between stages is the driving force for the vapour streams. The pressure drop
        is modeled as having two contributions, the dray and wet pressure drop. While dry pressure drop
        stems from the vapour flowing through the holes with in tray, the wet pressure drop is caused by the liquid
        holdup on the stage.
        \Eq{eq:col:dp_dyn}{
            p_j - p_{j+1} = h_{j}^{liq} \varrho_j^L g + 0.5 \xi \varrho_{j+1}^V \left( \frac{q_{j+1}^V}{A_h} \right)^2
        }%

    \subsubsection{Structured packing hydraulics}
    \label{sec:mathpro:dynamic:strpackhyd}
        \begin{figure}
            \begin{tikzpicture}
    \draw (0,0) node (A) [diamond,minimum size=3cm,shape aspect=1,draw] {} ;
    \draw [] (0,0) node (B) [diamond,minimum size=2.5cm,decorate,decoration={random steps,segment length=3pt,amplitude=1pt},draw] {} ;
\end{tikzpicture}

        \end{figure}

        \todoil{}{write somewhere that stage indices will be omitted for convenience.}
        Structured packings and their hydraulic behaviour are currently still under investigation. The number
        of available correlations for calculation of internal flow-rates is much more limited than for trays or even
        dumped packings. Among the most established models is the one developed by Bravo et al. \cite{Rocha.1993} at the
        University of Texas. This model has been extended to be valid in the loading region and account for different types
        of packing material \cite{Gualito.1997}. As main linking factor between vapour and liquid flow-rates as well as
        the pressure drop, the liquid holdup has bee identified by the authors. It is expressed in dimensionless form $h_t$ in
        terms of the corrugation side $S$, and the film thickness $\delta$
        \Eq{eq:dyn:dimlessholdup}{
            h_L = \frac{4}{S} \, \delta.
        }
        \ncr{h_L}{dimesionless holdup}{-}
        \ncg{\delta}{film thickness}{m}
        \ncr{S}{corrugation side}{m}
        %
        One very important factor while estimating the hydraulic behaviour is the dry pressure drop per meter packing $\delta p_{dry}$.
        Within in the presented model it is estimated by
        \Eq{eq:dyn:drydp}{
            \delta p_{dry} = \left( \frac{\varrho^G}{\varrho_{air, 1 bar}} \right)^{0.4} \left( \frac{C_1 \varrho^G v_G^2}{S \epsilon^2
                (\sin \Theta)^2} + \frac{C_2 \mu_G v_G}{S^2 \epsilon \sin \Theta} \right)
        }
        \ncg{\Theta}{corrugation angle}{rad}
        \ncg{\mu_G}{vapour viscosity}{Pa \cdot s}
        \ncg{\epsilon}{packing void fraction}{-}
        \ncr{v_G}{vapour velocity}{\frac{m}{s}}
        \ncr{\delta p_{dry}}{dry pressure drop}{\frac{Pa}{m}}
        \ncr{C_1}{packing constant}{}
        \ncr{C_2}{packing constant}{}
        %
        Another perquisite for calculating the holdup is the knowledge of the amount of wetted area of the available
        surface area within the packing. It seems reasonable to assume that this will be dependent on the characteristic
        of the liquid flow through the packing. To characterize the current, well known dimensionless numbers are
        used. Namely \emph{Weber} ($We$), \emph{Froude} ($Fr$) and Reynolds ($Re$) numbers
        \Eq{}{
            We & = \frac{v_L^2 \varrho_L S}{\sigma}, \\
            Fr & = \frac{v_L^2}{S g}, \\
            Re & = \frac{v_L S \varrho_L}{\mu_L}
        }
        \ncg{\mu_L}{liquid viscosity}{Pa \cdot s}
        \ncr{v_L}{liquid velocity}{\frac{m}{s}}
        %
        With that an approximation for the holdup correction factor $F_t$ due to partial wetting can be expressed as
        \Eq{eq:dyn:wettedarea}{
            F_t = \frac{A_1 \left( We \, Fr \right)^{0.15} S^{A_2}}{Re^{0.2} \, \epsilon^{0.6}
                (1 - 0.93 \cos \gamma) (\sin \Theta)^{0.3} }
        }
        %
        \Eq{}{
            h_L = \left( \frac{4 F_t}{S} \right)^{\frac{2}{3}} \left\{ \frac{3 \mu_L v_L}{\varrho_L \epsilon \sin \Theta
                g \left[ \left( \frac{\varrho_L - \varrho_G}{\varrho_L} \right) \left(1 - \frac{\delta p}{\delta p_{flood}} \right) \right]} \right\}^{\frac{1}{3}}
        }
    %
    \todo{mention delft, sti, hydrodynamic continuum}

\subsection{Integrated condenser / reboiler unit}
    \todo{include Integrated condenser / reboiler unit}

    As it has been stated on several occasions the condenser reboiler unit is essential in the cryogenic air separation process.
    When it comes to dynamics the pressure profile deserves much attention, as it is crucial for feasible operations of
    the entire process. The pressure difference between the low and high pressure column enables the heat exchange
    in this unit.
    \Eq{}{
    
    }


