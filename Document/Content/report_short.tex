
\chapter*{ASU Project - progress report}

\section*{Process model}
\begin{itemize}
    \item Process flowsheet converges and can initialize on its own. 
    \item Some cases do not initialize, which can be ''moved\_to'' from other converged solutions. Most likely reason are poor guesses in the recycle breaker.
    \item Optimizations for feed locations and tray numbers can be carried out on process flowsheet. 
    \item Flowsheet does not initialize with costing active for Argon column, as the cost model cannot deal with negative flow rates as inputs. 
    \item Costing can be active for other columns during initialization. 
    \item So far no satisfactory process configuration in terms Argon purity. This should however be attainable with appropriate optimization.
    \item Some specifications are not yet possible (purity, product component flow rates) can be added when needed with moderate effort. 
\end{itemize} 

\section*{Possible future directions}
\begin{itemize}
    \item Define objective functions that can be used for optimizations. 
    \item Models could be altered to dynamic models to allow for inclusion of control elements in process model (I cannot estimate effort. Since steady state initialization works a solid foundation should be there???)
    \item Uncertainty optimization. Since deterministic optimization works a scenario based approach should be implementable. Problem : find realistic scenarios, as Rodrigo -- to my knowledge -- cannot offer his expertise. 
    \item Explore further uncertainty approaches and see if and how they can be implemented in gPROMS. 
    \item While sizing is largely incorporated in the costing models no real check for operation boundaries (flooding, weeping, etc..) is included. Explore those. 
    \item include tray efficiencies 
\end{itemize}