    The total cost of a project is a very important measure to decide whether to embark on a certain endeavor.
    However in a complex financial system it cannot be taken as the sole factor to compare investment
    alternatives. Different other indices are used to measure the attractiveness or feasibility of an investment. One
    main distinction can be made between different measurements. Those that work with averaged cash flows
    and consider the project as a singe time period (\secref{sec:SinglePeriod}) and models that take into
    account cash flows made at different points in time -- so called multi-period models (\secref{sec:MultiPeriod}).
    Some examples for both types of indices are the subject of the rest of this section.


    \subsection{Single period estimation methods}
    \label{sec:SinglePeriod}
        Among the most prominent single period methods are the payout time (POT) and the return on
        investment (ROI) which will be discussed in this section.

        \subsubsection{Payout time}
            Payout or amortization  time is the time necessary to earn the total investment of the process. It is also
            often referred to as the break even point. A profitable venture will from that point on begin to make money.
            A shorter payback time can be considered a measure for an more attractive investment.
            \eq{
            	POT = \frac{\textnormal{capital expenditure}}{\textnormal{incoming cash flow / period}}
            }%
            \ncr{POT}{payout time}{a}
            However many aspects especially in potential profits are not included here.

        \subsubsection{Return on investment}
            The return on investment is defined as
            \eq{
            	ROI = \frac{\textnormal{average return / period}}{\textnormal{capital expenditure}} \cdot 100 \%.
            	\label{eq:roi}
            }%
            \ncr{ROI}{return on investment}{-}
            It denotes the equivalent to an interest rate if the earned interest is not reinvested such that the same
            investment is considered in every period. The average return is calculated form the expected returns
            over certain duration of time. The time considered might either be the total life-cycle of the process or
            the expected write-off time. As capital expenditure most likely the total investment will be used. However
            it might give a more precise picture if write-off or the time of individual payments is considered when
            evaluating the capital expenditure.

    \subsection{Multi period estimation methods}
    \label{sec:MultiPeriod}
        As discussed earlier, when using only averaged values and considering only one time period,
        a distorted picture of the financial situation will evolve. The sources of the money to be invested
        play a critical role in the evaluation process. Cash reserves, loans, preferred stock or other financial
        instruments are all conceivable sources for investment capital. Especially for large scale projects often a
        mixture of many different sources is used to allocate all necessary funds.

        A generic structure of the
        cash-flows during the life-cycle of a given project is shown in \figref{fig:CashFlow}. When the project
        commences no revenue is generated and only investments are made. As process operations begin revenue
        is generated and  the slope of the accumulated cash flow curve switches to an upward directions
        -- assuming the project is profitable. As time progresses the produced product might reduce in value
        through competitors entering the market or other factors. When the curve intersects the abscissa the
        project investment is earned and the break even point is reached. The two investment alternatives highlight
        that the amount of initial investment will affect the expected revenue. As more information is gathered
        during operations, opportunities may arise to optimize the process and leverage so far dormant potentials.
        These so called retrofit investments would ideally lead to an improved cash flow structure as indicated
        by the green line.

        \begin{figure}
            \center
            \begin{tikzpicture}
	\begin{axis} [
		xlabel={time},
		ylabel=accumulated cash flow,
		width=13cm,
		axis lines=middle,
		axis line style={->},
		xtick={0},
		ytick={0},
		every axis y label/.style={at={(ticklabel cs:0.7)},rotate=90,anchor=near ticklabel},
		%every axis x label/.style={at={(ticklabel cs:1,10pt)},anchor=near ticklabel},
		yscale=0.6
		]
		\addplot [color=ImperialDarkBlue] coordinates {
			(0, 0)
			(4, -1)
		};
		\addplot [color=ImperialDarkBlue] coordinates {
			(0, 0)
			(8, -1.2)
		};
		\addplot [color=ImperialDarkBlue, domain=4:50] {ln(x-2)-1.693};
		\addplot [color=ImperialDarkBlue, domain=8:50] {ln(1.8*(x-6.88889))-1.893};
	\end{axis}
\end{tikzpicture}
            \caption{Accumulated cash flows over project life-cycle.\cite{Marquardt.2008}}
            \label{fig:CashFlow}
        \end{figure}

        Before further investigating multi-period models some basic concepts of financial mathematics
        and the treatment of interest should be reviewed. When investing the capital $C_0$ at compounded
        interest for $n$ years at a rate of $i$ \%, the compound amount will yield
        \eq{
        	C_n = C_0 \cdot q^n.
        	\ncr{q}{interest factor}{-}
        	\ncr{C_n}{final value of an investment}{\$}
        	\ncr{C_0}{initial value of an investment}{\$}
        }%
        Where $q$ denotes the interest factor
        \eq{
        	q = 1 + \frac{i}{100}.
        	\ncr{i}{interest rate}{\%}
        }
        On the other hand the current value of an investment that will yield $C_n$ in $n$ years can be calculated
        by
        \eq{
        	C_0 = C_n \cdot q^{-n}
        }
        The above considerations always assume a single payment at the beginning or end of the entire period.
        Annuities however are usually in several trances with regular payments to be made or received at
        predefined instances. Assuming those payments are of equal size $a$, the final value can be computed by
        \eq{
        	C_n = a \cdot q^{\alpha} \cdot \frac{q^n - 1}{q - 1},
        	\ncr{a}{annuity}{\$}
        }%
        while the current value of an annuity yields
        \eq{
        	C_0 = a \cdot q^{\alpha - n} \cdot \frac{q^n - 1}{q - 1},
        }%
        The factor $\alpha$ in the previous equations denotes whether payments are made at the beginning
        of a period ($\alpha = 1$ or at the end ($\alpha = 0)$

        Given those basic considerations, multi-period model can be discussed. Here not only different alternatives can be compared,
        but also the profitability in comparison with investments in financial products can be assessed.

        \subsubsection{Net present value}
            The net present value ($NPV$) describes the amount of money to be invested if all cash flows --
            incoming and outgoing -- are discounted to the project start ($t = 0$). The present value of
            all expenses is then
            \eq{
            	C_{0e} = e_0 + e_1 \cdot q^{-1} + \dots + e_n \cdot q^{-n} = \sum_i^n e_i \cdot q^{-i}.
            	\ncr{C_{0e}}{Present value of all expenses}{\$}
            }%
            Here it is given that all expenses are paid at the beginning of each period, as it is the most common case.
            If revenues $r_i$ are realized then as well an equivalent formula would be attained. In most cases
            revenues will come in at the end of a period in which case the present value becomes
            \eq{
            	C_{0r} = r_0 \cdot q^{-1} + r_1 \cdot q^{-2} + \dots + r_n \cdot q^{-(n+1)} = \sum_i^n r_i \cdot q^{-(i+1)}.
            	\ncr{C_{0r}}{Present value of all revenues}{\$}
            }%
            The $NPV$ of a project is then derived from the present values of all expenses and revenues
            \eq{
            	NPV = C_{0r} - C_{0e}.
            	\ncr{NPV}{Net present value}{\$}
            }%
            For any project to be considered as investment alternative the present value needs to be positive since
            otherwise the investment would yield losses.

        \subsubsection{Discounted cash flow rate of return}
            The formula for the discounted cash flow rate of return is very similar to the one for a net present value.
            The difference is, that rather than computing the net present value with given interest rates, the present
            value is set to zero and the resulting interest rate its then calculated. Other than with the previous methods
            no analytical solution can be presented but rather an iterative approach to find a solution to
            \eq{
            	0 = \sum_i^n (r_i - e_i) \cdot q^{-i}.
            }%
            Here its was assumed that all payments -- incoming and outgoing -- are made at the beginning of a period.
            The only variable is the interest rate which is included in the interest factors $q$. This method gives
            the interest rate which would be necessary to earn all time dependent expenses within $n$ years. In this
            case a higher $DCFRR$ indicates a more attractive investment.

        \subsubsection{Annuity method}
            Within the annuity method two different annuities are calculated and then compared. First the annuity $a$
            of an investment of $C_0$ at market conditions is of interest.
            \eq{
            	a = C_0 \cdot \frac{q^{n-\alpha} (q - 1)}{q^n - 1}
            }%
            This annuity is the amount that could be paid out each period if $C_0$ is invested, compound interest
            is considered and all funds are used up at the end of $n$ years.

            This is then compared to the annuity of the considered project
            \eq{
            	a_P = \sum_i^n (r_i - e_i) \cdot q^{-i} \cdot \frac{q^{n-\alpha} (q - 1)}{q^n - 1}.
            }%
            Only if $a_P > a$ is the project more profitable than simply investing the required capital in a
            financial product.

        \subsubsection{Interest Rates}
            With all the presented models it was implicitly assumed that interest rates for debit and credit are
            equal. The reality however is much different. It will therefore be prudent to use different rates of interest
            for each case. The basic calculations however remain unchanged. Especially the interest rate that can be
            earned when investing a certain amount will need to be estimated. As certain products with a know $ROI$
            will not necessarily be the most attractive options on the capital market. Thus in many companies
            there is an internal value given for this interest rate, which is calculated from historical data.

