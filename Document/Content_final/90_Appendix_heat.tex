\chapter{Simultaneous heat integration}
\label{app:heat}

    Traditionally heat integration has been carried out in a sequential manner, where it is for the purposes of process
    optimization assumed, that all heating and cooling is done by external utilities. After an (locally) optimal
    process configuration is identified, all hot and cold streams within the process are identified, and their
    temperature intervals fixed. %In the context of heat integration hot and cold streams do not refer to whether a given
    %stream has a subjectively high or low temperature, but rather whether a given stream needs to be cooled (hot) or
    %heated (cold) within the process. %
    In subsequent steps first the minimum utility requirements, and maximum number of heat exchangers are identified,
    and a specific heat exchange network (HEN) is designed.
    While this approach has been successfully applied to a multitude of processes and led to substantial savings,
    it is questionable if such a sequential approach will yield an optimal or near optimal solution.

    Therefore some efforts have been made to develop efficient strategies for simultaneous process optimization
    and heat integration. Two general approaches can be distinguished. The first one based on the pinch concept.
    \todoil{}{elaborate on pinch concept?} These methods are able to identify the minimum heat requirement
    as well as stream temperatures during process optimization. The first model along these lines was published
    by Duran and Grossmann \cite{Duran.1986} in 1986. They introduced a limited number of quite well behaved
    constraints into the optimization model to ensure no minimum driving force violations. Recently
    this model has been extended to handle phase changes and by fixing the utilities to zero been applied
    to the design of a multi-stream heat exchanger \cite{Kamath.2012}. The major drawback with these methods
    \todoil{}{read about and mention transshipment model by Moriari (sequential)}
    is that one cannot target area of a given heat exchanger as the approach temperatures are not computed
    by the model. Therefore the sometimes substantial trade-off between the cost for heat exchange area
    and utility cost cannot be regarded.

    A second approach employs superstructures of a HEN to find optimal matchings of process streams. No pinch
    point calculations are required, as the actual heat exchange is more or less modeled explicitly. This
    leads to the benefit, that approach temperatures as well as exchanged heat duties between each stream coupling
    are known within the model, and the cost of the designed unit can be considered in an economic objective
    function.

    \section{HEN superstructure}
        The approach and respective superstructure adapted in this thesis were first published by Yee and Grossmann
        \cite{Yee.1990}. \Figref{fig:HX_super} shows the stage wise superstructure for a HEN consisting of two hot
        and two cold streams.
    
        \begin{figure}
                \begin{tikzpicture}
        \pgfmathsetmacro{\HXsep}{1.5}
        \pgfmathsetmacro{\HXsepup}{1.0}
        \pgfmathsetmacro{\HXrad}{0.5}
        \pgfmathsetmacro{\HXinsep}{0.7}
        \pgfmathsetmacro{\Xsep}{2}
        \pgfmathsetmacro{\Stagesep}{7}
        
        \node (H1_in) [inner sep=0] at (0,0) {} ; 
        \draw ($(H1_in) + (\HXsep,\HXsepup) $) circle (\HXrad) node (HX11) [inner sep=\HXinsep,anchor=center] {\scriptsize $H_1 / C_1$} ; 
        \draw ($(H1_in) + (\HXsep,-\HXsepup) $) circle (\HXrad) node (HX12) [inner sep=\HXinsep,anchor=center] {\scriptsize $H_1 / C_2$} ; 

        \node (H2_in) [inner sep=0] at ($(H1_in) + 3*(0,-\HXsep)$) {} ;
        \draw ($(H2_in) + (\HXsep,\HXsepup) $) circle (\HXrad) node (HX21) [inner sep=\HXinsep,anchor=center] {\scriptsize $H_2 / C_1$} ; 
        \draw ($(H2_in) + (\HXsep,-\HXsepup) $) circle (\HXrad) node (HX22) [inner sep=\HXinsep,anchor=center] {\scriptsize $H_2 / C_2$} ; 
        
        \draw [arrow,RWTHRed] (H1_in) -- (H1_in |- HX11.south west) -- (HX11.south west) ;
        \draw [arrow,RWTHRed] (H1_in) -- (H1_in |- HX12.south west) -- (HX12.south west) ;
        \draw [arrow,RWTHRed] (H2_in) -- (H2_in |- HX21.south west) -- (HX21.south west) ;
        \draw [arrow,RWTHRed] (H2_in) -- (H2_in |- HX22.south west) -- (HX22.south west) ;
        
        \node (H1_in_a) [inner sep=\Xsep] at ($(H1_in) + (-0.5,0)$) {} ;
        \node (H1_in_b) [inner sep=\Xsep] at (H1_in_a |- HX12.north west) {} ;
        \node (H1_in_c) [inner sep=\Xsep] at (H1_in |- HX12.north west) {} ;
        \node (H2_in_b) [inner sep=\Xsep] at (H2_in |- HX22.north west) {} ;
        \node (H2_in_a) [inner sep=\Xsep] at ($(H2_in) + (-1,0)$) {} ;
        
        \draw [arrow,RWTHBlue] (HX21.north west) -- (H1_in_a |- HX21.north west) ;
        \draw [arrow,RWTHBlue] (HX11.north west) -- (H1_in_a |- HX11.north west) ;
        \draw [stdline,RWTHBlue] (H1_in_a |- HX21.north west) -- (H1_in_a |- HX11.north west) ;
        
        \draw [arrow,RWTHBlue] (HX12.north west) -- (H1_in_c) -- (H1_in_b) -- (H2_in_a |- H1_in_b) ;
        \draw [arrow,RWTHBlue] (HX22.north west) -- (H2_in_b) -- (H2_in_b -| H2_in_a) ;
        \draw [stdline,RWTHBlue] (H2_in_b -| H2_in_a) -- (H2_in_a |- H1_in_b) ;
        
        \node (H1_out) [inner sep=0] at ($(H1_in) + (4,0)$) {} ;
        \node (H1_out_a) [inner sep=\Xsep] at ($(H1_out) + (0,-0.5)$) {} ;
        \node (H1_out_b) [inner sep=\Xsep] at ($(HX11.south east) + (1.5,0)$) {} ;
        \node (H1_out_c) [inner sep=\Xsep] at ($(HX12.south east) + (1,0)$) {} ;
        \node (H1_out_d) [inner sep=\Xsep] at (H1_out_b |- H1_out_c) {} ;
        
        \node (H2_out) [inner sep=0] at ($(H2_in) + (4,0)$) {} ;
        \node (H2_out_a) [inner sep=\Xsep] at ($(H2_out) + (0,-0.5)$) {} ;
        \node (H2_out_b) [inner sep=\Xsep] at (H1_out_c |- HX21.south east) {} ;
        \node (H2_out_c) [inner sep=\Xsep] at (H1_out_c |- HX21.north east) {} ;
        
        \draw [arrow,RWTHRed] (HX11.south east) -- (H1_out_b) -- (H1_out_b -| H1_out) ;
        \draw [arrow,RWTHRed] (HX12.south east) -- (H1_out_c) -- (H1_out_d) -- (H1_out_d -| H1_out) ;
        \draw [stdline,RWTHRed] (H1_out_b -| H1_out) -- (H1_out_d -| H1_out) ;
        
        \draw [arrow,RWTHRed] (HX21.south east) -- (H2_out_b) -- (H2_out_b -| H2_out) ;
        \draw [arrow,RWTHRed] (HX22.south east)  -- (HX22.south east -| H2_out) ;
        \draw [stdline,RWTHRed] (H2_out_b -| H2_out) -- (HX22.south east -| H2_out) ;
        
        \draw [arrow,RWTHBlue] (HX22.north east -| H2_out_b) -- (HX22.north east) ;
        \draw [arrow,RWTHBlue] (HX12.north east -| H2_out_b) -- (HX12.north east) ;
        \draw [stdline,RWTHBlue] (HX12.north east -| H2_out_b) -- (HX22.north east -| H2_out_b) ;
        
        \draw [arrow,RWTHBlue] (HX21.north east -| H1_out_d) -- (H2_out_c) -- (HX21.north east) ; 
        \draw [arrow,RWTHBlue] (HX11.north east -| H1_out_b) -- (HX11.north east) ;
        \draw [stdline,RWTHBlue] (HX11.north east -| H1_out_b) -- (HX21.north east -| H1_out_d) ;
        
        \node (H1_mid) [inner sep=0] at ($(H1_in) + (\Stagesep,0)$) {} ; 
        \draw ($(H1_mid) + (\HXsep,\HXsepup) $) circle (\HXrad) node (HX31) [inner sep=\HXinsep,anchor=center] {\scriptsize $H_1 / C_1$} ; 
        \draw ($(H1_mid) + (\HXsep,-\HXsepup) $) circle (\HXrad) node (HX32) [inner sep=\HXinsep,anchor=center] {\scriptsize $H_1 / C_2$} ; 

        \node (H2_mid) [inner sep=0] at ($(H1_mid) + 3*(0,-\HXsep)$) {} ;
        \draw ($(H2_mid) + (\HXsep,\HXsepup) $) circle (\HXrad) node (HX41) [inner sep=\HXinsep,anchor=center] {\scriptsize $H_2 / C_1$} ; 
        \draw ($(H2_mid) + (\HXsep,-\HXsepup) $) circle (\HXrad) node (HX42) [inner sep=\HXinsep,anchor=center] {\scriptsize $H_2 / C_2$} ; 
        
        \draw [arrow,RWTHRed] (H1_mid) -- (H1_mid |- HX31.south west) -- (HX31.south west) ;
        \draw [arrow,RWTHRed] (H1_mid) -- (H1_mid |- HX32.south west) -- (HX32.south west) ;
        \draw [arrow,RWTHRed] (H2_mid) -- (H2_mid |- HX41.south west) -- (HX41.south west) ;
        \draw [arrow,RWTHRed] (H2_mid) -- (H2_mid |- HX42.south west) -- (HX42.south west) ;
        
        \node (H1_mid_a) [inner sep=\Xsep] at ($(H1_mid) + (-0.5,0)$) {} ;
        \node (H1_mid_b) [inner sep=\Xsep] at (H1_mid_a |- HX32.north west) {} ;
        \node (H1_mid_c) [inner sep=\Xsep] at (H1_mid |- HX32.north west) {} ;
        \node (H2_mid_b) [inner sep=\Xsep] at (H2_mid |- HX42.north west) {} ;
        \node (H2_mid_a) [inner sep=\Xsep] at ($(H2_mid) + (-1,0)$) {} ;
        
        \draw [arrow,RWTHBlue] (HX41.north west) -- (H1_mid_a |- HX41.north west) ;
        \draw [arrow,RWTHBlue] (HX31.north west) -- (H1_mid_a |- HX31.north west) ;
        \draw [stdline,RWTHBlue] (H1_mid_a |- HX41.north west) -- (H1_mid_a |- HX31.north west) ;
        
        \draw [arrow,RWTHBlue] (HX32.north west) -- (H1_mid_c) -- (H1_mid_b) -- (H2_mid_a |- H1_mid_b) ;
        \draw [arrow,RWTHBlue] (HX42.north west) -- (H2_mid_b) -- (H2_mid_b -| H2_mid_a) ;
        \draw [stdline,RWTHBlue] (H2_mid_b -| H2_mid_a) -- (H2_mid_a |- H1_mid_b) ;
        
        \node (H3_out) [inner sep=0] at ($(H1_mid) + (4,0)$) {} ;
        \node (H3_out_a) [inner sep=\Xsep] at ($(H3_out) + (0,0.5)$) {} ;
        \node (H3_out_b) [inner sep=\Xsep] at ($(HX31.south east) + (1.5,0)$) {} ;
        \node (H3_out_c) [inner sep=\Xsep] at ($(HX32.south east) + (1,0)$) {} ;
        \node (H3_out_d) [inner sep=\Xsep] at (H3_out_b |- H3_out_c) {} ;
        
        \node (H4_out) [inner sep=0] at ($(H2_mid) + (4,0)$) {} ;
        \node (H4_out_a) [inner sep=\Xsep] at ($(H4_out) + (0,0.5)$) {} ;
        \node (H4_out_b) [inner sep=\Xsep] at (H3_out_c |- HX41.south east) {} ;
        \node (H4_out_c) [inner sep=\Xsep] at (H3_out_c |- HX41.north east) {} ;
        
        \draw [arrow,RWTHRed] (HX31.south east) -- (H3_out_b) -- (H3_out_b -| H3_out) ;
        \draw [arrow,RWTHRed] (HX32.south east) -- (H3_out_c) -- (H3_out_d) -- (H3_out_d -| H3_out) ;
        \draw [stdline,RWTHRed] (H3_out_b -| H3_out) -- (H3_out_d -| H3_out) ;
        
        \draw [arrow,RWTHRed] (HX41.south east) -- (H4_out_b) -- (H4_out_b -| H4_out) ;
        \draw [arrow,RWTHRed] (HX42.south east)  -- (HX42.south east -| H4_out) ;
        \draw [stdline,RWTHRed] (H4_out_b -| H4_out) -- (HX42.south east -| H4_out) ;
        
        \draw [arrow,RWTHBlue] (HX42.north east -| H4_out_b) -- (HX42.north east) ;
        \draw [arrow,RWTHBlue] (HX32.north east -| H4_out_b) -- (HX32.north east) ;
        \draw [stdline,RWTHBlue] (HX32.north east -| H4_out_b) -- (HX42.north east -| H4_out_b) ;
        
        \draw [arrow,RWTHBlue] (HX41.north east -| H3_out_d) -- (H4_out_c) -- (HX41.north east) ; 
        \draw [arrow,RWTHBlue] (HX31.north east -| H3_out_b) -- (HX31.north east) ;
        \draw [stdline,RWTHBlue] (HX31.north east -| H3_out_b) -- (HX41.north east -| H3_out_d) ;
        
        \draw [arrow,RWTHRed] ($(H1_in) + (-2.5,0)$) node [left,black] {\footnotesize $H_1$} -- (H1_in_a) -- (H1_in) ;
        \draw [arrow,RWTHRed] ($(H2_in) + (-2.5,0)$) node [left,black] {\footnotesize $H_2$} -- (H2_in_a) -- (H2_in) ;
        \draw [arrow,RWTHBlue] ($(H1_in) + (-0.5,0.5)$) -- ($(H1_in) + (-2.5,0.5)$) node [left,black] {\footnotesize $C_1$} ;
        \draw [arrow,RWTHBlue] ($(H2_in) + (-1,0.5)$) -- ($(H2_in) + (-2.5,0.5)$) node [left,black] {\footnotesize $C_2$} ;
        
        \draw [arrow,RWTHRed] (H1_out) -- (H1_mid_a) -- (H1_mid) ;
        \draw [arrow,RWTHRed] (H2_out) -- (H2_mid_a) -- (H2_mid) ;
        \draw [arrow,RWTHBlue] (H2_mid_a |- H2_out_a) -- (H2_out_a) -- ++(-1,0) ;
        \draw [arrow,RWTHBlue] (H1_mid_a |- H1_out_a) -- (H1_out_a) -- ++(-0.5,0) ;
        
        \draw [arrow,RWTHBlue] ($(H3_out_a) + (1.5,-1)$) node [right,black] {\footnotesize $C_1$} -- ($(H3_out_a) + (0,-1)$) -- ++(-0.5,0) ;
        \draw [arrow,RWTHBlue] ($(H4_out_a) + (1.5,-1)$) node [right,black] {\footnotesize $C_2$} -- ($(H4_out_a) + (0,-1)$) -- ++(-1,0) ;
        \draw [arrow,RWTHRed] ($(H3_out_a) + (0,-0.5)$) -- ++(1.5,0) node [right,black] {\footnotesize $H_1$} ; 
        \draw [arrow,RWTHRed] ($(H4_out_a) + (0,-0.5)$) -- ++(1.5,0) node [right,black] {\footnotesize $H_2$} ; 
        
        \draw[dashvol] (-2,2) -- (-2,-7) ;
        \draw[dashvol] ($(-2,2) + (\Stagesep,0)$) -- ($(-2,-7) + (\Stagesep,0)$) ;
        \draw[dashvol] ($(-2,2) + 2*(\Stagesep,0)$) -- ($(-2,-7) + 2*(\Stagesep,0)$) ;
        
        \node at (-2,-7.3) {\footnotesize $\ell = 1$} ;
        \node at ($(-2,-7.3) + (\Stagesep,0)$) {\footnotesize $\ell = 2$} ; 
        \node at ($(-2,-7.3) + 2*(\Stagesep,0)$) {\footnotesize $\ell = 3$} ;
        
        \node at ($(HX11) + (0,1.5)$) {Stage 1} ; 
        \node at ($(HX31) + (0,1.5)$) {Stage 2} ;
        
    \end{tikzpicture}

            \caption{Superstructure for multi-stream heat exchanger. \cite{Yee.1990}}
            \label{fig:HX_super}
        \end{figure}
    
        In this stage wise structure each hot stream can exchange heat with each cold stream within each stage.
        The following assumptions were made when the model was developed
        \begin{itemize}
            \item Constant heat capacities
            \item Constant heat transfer coefficients
            \item Countercurrent heat exchangers
            \item Isothermal mixing at each stage.
        \end{itemize}
    
        The assumption of constant heat capacities is a common one in the design of HEN's. When no phase boundaries are passed
        and the temperature range of the involved is not too wide, it is a reasonably good approximation of the real conditions.
        While constant heat transfer coefficients a re assumed the model leaves the flexibility to define different
        coefficients for each pairing of hot and cold streams. Countercurrent heat exchangers are common in industrial practice.
        This assumption however does not really pose a limitation, as the model can easily be altered to account for concurrent
        units.
    
        The last assumption of isothermal mixing is a more major one. It has been introduced as it allows for significant
        simplifications and leads to a model, where all constraints are linear and all non-linearities are restricted
        to the objective function. While that is certainly not true for the entire process model, it should at least
        allow for some reductions in the model complexity. The assumption states, that regardless of which streams a
        given stream exchanges heat with, it will leave at the same temperature. Due to that all energy balances around
        each unit in the superstructure can be eliminated as well as the subsequent mixing of the streams.
    
        \paragraph{Model equations}
    
        First of all a heat balance at each stage is necessary
        \Eq{}{
            F_i (T_{i, \ell} - T_{i, \ell+1}) = \sum_j q_{ijk} \\
            f_j (T_{i, \ell} - T_{i, \ell+1}) = \sum_i q_{ijk}
        }
    
        The heat exchange area $A_{hx}$ can be computed from the exchanged energy, the heat transfer coefficients $\alpha_{ij}$
        and the logarithmic mean temperature difference $LMTD$.
        \Eq{}{
            A_{hx} = \sum_i \sum_j \sum_k \frac{q_{ijk}}{\alpha_{ij} * LMDT_{ijk} + \delta}
        }
    
        While the small number $\delta$ is included to avoid problems in the program, when $LMDT$ becomes zero
        In order to avoid further numerical difficulties when the approach temperatures $\Delta T_{ij}$ at each side of an exchange
        unit approach zero, it was proposed to use an approximation introduced by Chen \cite{Chen.1987}
        \Eq{}{
            LMDT_{ijk} \approx \left[ \Delta T_{ijk} \cdot \Delta T_{ijk+1} \frac{\Delta T_{ijk} + \Delta T_{ijk+1}}{2} \right]^{\frac{1}{3}}.
        }
    
        While the approach temperatures are defined as
        \Eq{}{
            \Delta T_{ijk} = \max \left\{0, T_{ik} - T_{jk}\right\}
        }
        As the $\max$ function is non-smooth and thus non differentiable at the points $T_{ik} = T_{jk}$, \gproms internally uses a smooth
        approximation. The exact for of which is unknown to the author.
        
    \section{Pinch concept based}
