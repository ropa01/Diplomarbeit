   
    The issue of heat integration is essential to the economic performance of cryogenic air separation. Foremost
    one must consider the special column configuration used in the process. Since operation of the condenser in the
    low pressure section only becomes possible if the reboiler in the high pressure section functions as heat sink,
    no external utilities are supplied to either unit. Rather they are combined into a single heat exchange unit. Thus
    the absolute value of the reboiler energy must matched by the energy recovered from the condenser
    (see \secref{sec:mathpro:steady:conreb}). Furthermore the material streams entering the process can -- and should -- 
    exchange heat with the process streams leaving it. The combined condenser \& reboiler for the double effect column is
    assumed as a given heat exchange. This makes sense insofar, as this is a necessity in terms of the actual physical
    implementation of the process units. Also the usage of the oxygen rich liquid from the HPC as coolant in the Argon
    condenser is assumed as fixed.

    This leaves the process stream leaving the compression stage of the process as well as all product and waste streams
    leaving the process. All those streams are -- for simulation purposes and also in some process implementations  -- fed
    into a single multi-stream heat exchange unit. In actual processes all heat exchange and much of the process operations
    take place in the so called ''cold box''. As such a heavily insulated area is referred to. This is done to minimize
    heat exchange with the surroundings. Therefore and for further reasons compact heat exchange units such as plate-fin
    multi-stream heat exchangers are favoured when dealing with cryogenic processes in general and the cryogenic air septation
    in particular.

    Due to the importance of heat integration to the ASU process some thought should be given as to what modelling approach
    should be employed. Although the field of heat integration is one of the most intensively studied within process engineering,
    only  a limited amount of approaches is available in open literature \cite{Kamath.2012}.

    A very simple approach is to employ enthalpy and material balances around the entire unit.
    \Eq{}{
        0 & = \sum_j \left(H_j^{in} - H_j^{out} \right), \\
        0 & = x_{ji}^{in} - x_{ji}^{out}, \\
        0 & = p_i^{in} - p_j^{out} - \Delta p_j, \\
        0 & = F_j^{in} - F_j^{out}.
    }
    Here all entering process streams are variable but determined by the respective model equations they are originating from.
    Hence all but one outlet stream temperature along with all pressure drops must be specified.
