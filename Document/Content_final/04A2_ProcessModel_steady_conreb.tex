        \begin{figure}
            \center
            \begin{tikzpicture}
    \pgfmathsetmacro{\unitwidth}{5} ;
    \pgfmathsetmacro{\unitheight}{3} ;
    \pgfmathsetmacro{\domerad}{0.5*\unitwidth/2} ;
    \pgfmathsetmacro{\lang}{76} ;
    \pgfmathsetmacro{\rang}{180-\lang} ;
    \pgfmathsetmacro{\conheight}{0.7*\unitheight} ;
    \pgfmathsetmacro{\conwidth}{0.45*\unitwidth} ;
    
    \draw [line width=0.5pt,rounded corners] (-0.5*\unitwidth,0) -- ++(0,0.5*\unitheight) .. controls +(\lang:\domerad) and +(\rang:\domerad) .. (0.5*\unitwidth,0.5*\unitheight) node (top) [pos=0.5,inner sep=0] {} -- ++(0,-\unitheight) .. controls +(-\rang:\domerad) and +(-\lang:\domerad) .. (-0.5*\unitwidth,-0.5*\unitheight) node (bot) [pos=0.5,inner sep=0] {} -- cycle ;
    
    \draw [line width=0.5pt] (-0.5*\conwidth,0.5*\conheight) rectangle (0.5*\conwidth,-0.5*\conheight) ;
    \node (cin) [shape=semicircle, shape border rotate=90, minimum height=0.25cm,draw, inner sep=0] at ($(-0.5*\conwidth,0.5*\conheight) - (0.11,0.25)$) {} ;
    \node (cout) [shape=semicircle, shape border rotate=270, minimum height=0.25cm,draw, inner sep=0] at ($(0.5*\conwidth,-0.5*\conheight) + (0.11,0.25)$) {} ;
    
    \draw [arrow] ($(cin) - (2.5,0)$) -- (cin) node [above,pos=0.2] {$V_{con}^{in}$};
    \draw [arrow] (cout) -- ($(cout) + (2.5,0)$) node [above,pos=0.8] {$L_{con}^{out}$} ;
    \draw [arrow] (top) -- ++(0,1) -- ++(2,0) node [pos=0.5,above] {$V_{reb}^{out}$} ;
    \draw [arrow] (bot) -- ++(0,-1) -- ++(2,0) node [pos=0.5,above] {$L_{reb}^{out}$} ;
    \node (A) at ($(0.5*\unitwidth,0.5*\unitheight)$) {} ;
    \node (B) at ($(cout) + (2.5,0)$) {} ;
    \draw [arrow] (A -| B) -- ($(0.5*\unitwidth,0.5*\unitheight) - (1,0)$) node [above,pos=0.2] {$L_{reb}^{in}$} ;
\end{tikzpicture}
            \caption{Integrated condenser \& reboiler unit.}
            \label{fig:mathpro:conreb_ss}
        \end{figure}

        The integrated condenser reboiler unit is a peculiarity of the ASU process. The fact that condenser and
        reboiler are linked form the close coupling between both sections of the double effect column.
        This stems from the fact, that each change in the parameters and operation of this unit will reverberate
        through the entire process, as they will simultaneously affect boil-up ratio and reflux
        ratio of the column, both major operational parameters with great effect.

        Therefore a separate model for this unit was developed, rather that simply establishing a couple between the
        the condenser and the reboiler in the case for two separately modeled column sections.

        The reboiler side is modeled much like an equilibrium stage in the column, considering the entering
        and leaving streams as depicted in \figref{sec:mathpro:steady:conreb}. However in contrast to the
        equations supplied for a column section, no integer optimization terms will have to be considered.
        Furthermore no side draws, or separate feeds are recognized. With that the material balances are
        \Eq{}{
            0 = L_{reb}^{in} x_{i,reb}^{in} - L_{reb}^{out} x_{i}^{reb} - V_{reb}^{out} y_{i}^{reb}, \eqannc.
        }
        The energy balance taking into account the transferred energy ($Q_{CR}$) becomes therefore
        \Eq{}{
            0 = L_{reb}^{in} h_{reb}^{in} - L_{reb}^{out} h^L_{reb} - V_{reb}^{out} h^V_{reb} + Q_{CR}.
        }
        The equilibrium and summation equations are analogous to the ones presented for the column model.
        A pressure drop again is assigned with relation to the lowest column stage.

        The condenser is modeled as a total condenser. The mass and component balances therefore become trivial
        \Eq{}{
            x_{i,con} & = y_{i,con}^{in}, \eqannc, \\
            V_{con}^{in} & = L_{con}^{out}.
        }
        The temperature is set to the bubble temperature of the given mixture at an assigned pressure. The
        enthalpies of the entering and leaving stream can therefore be calculated from the given EOS. And
        the energy balance determines the transferred energy
        \Eq{}{
            0 = V_{con}^{in} h_{con}^{in} - L_{con}^{out} h^L_{con} - Q_{CR}.
        }