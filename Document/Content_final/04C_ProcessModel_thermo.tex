    Aside from the unit operation models, the behaviour of materials in a process needs to be adequately
        accounted for. This is done by means of so called equations of state (EOS) and excess Gibbs energy
        models. In terms of thermodynamics there are only a limited amount of variables. Namely the pressure,
        density and temperature as well as composition. While equations of state can model a given system in
        the vapour as well as liquid phase, excess Gibbs energy models only account for the behaviour of a liquid
        and need to be used in conjunction with other models for the vapour phase. However they have shown
        considerable better performance for highly non-ideal systems \cite{AndreasPfennig.2003}. As mentioned
        earlier (\secref{sec:mathpro:steady:comp}) it is essential to accurately capture the non-idealities of air
        in order to capture the liquefaction process. In the case of cryogenic air separation, the Peng-Robinson
        as well as the Benders equation of state have shown satisfactory performance. The Peng-Robinson equation
        was chosen to be used in the presented model
        \Eq{eq:peng_rob}{
            p & = \frac{RT}{V-b} - \frac{a_c \left[1+m\left(1-\sqrt{T_r}\right)\right]^2}{V^2+2bV-b^2} \\
            m & = 0.37464 + 1.54226 \omega - 0.26992 \omega^2 \\
            a_c & = 0.45724 \frac{R^2T_c^2}{p_c} \\
            b & = 0.077796 \frac{RT_c}{p_c} \\
            \omega & = -1 - \log_{10} \, (p_r^{sat})_{T_r = 0.7}
        }
        \ncr{p}{pressure}{Pa}
        \ncr{m}{parameter in Peng-Robinson EOS}{-}
        \ncr{a_c}{parameter in Peng-Robinson EOS}{\frac{m^5}{mol^2 s^2}}
        \ncr{b}{parameter in Peng-Robinson EOS}{\frac{m^3}{mol}}

        However the Peng-Robinson EOS relies on the so called one-fluid theory which models each fluid as pure.
        To model mixtures the pure component parameters have to be ''mixed''
        \Eq{}{
            a & = \sum_{i=1}^C \sum_{j=1}^C y_i y_j a_{ij}, \\
            a_{ij} & = \sqrt{a_i a_j} (1 - k_{ij}), \\
            b & = \sum_{i=1}^C y_i b_i.
        }

        From that EOS numerous relevant properties such as excess enthalpy, fugacity coefficients or densities
        can be calculated. For a list of some relevant equations refer to \secref{app:peng_rob_deriv}.
