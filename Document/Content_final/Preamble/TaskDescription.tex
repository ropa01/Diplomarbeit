% Task Description
\thispagestyle{plain}

\begin{center}
	\Large

	\vspace{2ex} Task Description
\end{center}

\normalsize

Conceptual process design is a task of major importance within the development of new and the retrofit
of existing processes. While there has been a lot of research on optimization-based process design such
approaches are rarely integrated in industrial practise. Up to now simulation software is still the standard
tool for conceptual design. In order to ease the implementation of optimization-based process design in
industrial practise, the main goal of this thesis is to facilitate a superstructure-based process optimization
within the process modelling environment \gproms, which is widely applied in industry. Therefore, firstly a
superstructure formulation for a distillation column model needs to be implemented. In addition, as initialization
plays a significant role during simulation and optimization, a robust, automated initialization procedure is an
essential part of the model development. To finally determine an optimal process design an economic model
of the process needs to be formulated and a solution strategy together with an adequate optimization
algorithm need to be determined.

In order to validate the capabilities of the structural optimization approach the design of a cost-optimal cryogenic
air separation process is chosen as an industrially relevant and complex case study. The process is highly coupled,
in terms of energy and material streams, as it employs a double effect column along with a coupled side stripper column.
This gives rise to challenges in terms of simulating and operating the process. Due to these complexities the process
serves as an excellent example to illustrate the effectiveness of optimization strategies.

In addition to the structural optimization different aspects of process design or control can be coupled to or even
directly be integrated within the optimization. Among those aspects are potentials for heat integration or the question
of a suitable control structure. Furthermore several uncertainties arise from unknown process parameters such as changing
feed conditions, product demands or the development of the market conditions. For all these aspects different methodologies
have been proposed in the literature, which similar to the structural optimization are not really integrated in industrial
practice yet.  Most or at least some of the methodologies should therefore also be investigated within this thesis in order
to identify their potential ease of integration and application in the process modelling environment \gproms to
simultaneously optimize process and equipment to minimize the total process cost with user defined objective functions.

 \vfill{
\begin{flushright}
	\begin{minipage}[r]{10cm}
		\begin{center}
			\hrule\vspace*{2ex}Prof. Efstratios Pistikopoulos\\
			\vspace{1.5cm}
			\hrule\vspace*{2ex}Prof. Dr.-Ing. Wolfgang Marquardt\\
			\vspace{0.5cm}
		\end{center}
	\end{minipage}
\end{flushright}


Betreuer der Arbeit: Dipl. Ing. Mirko Skiborowski\\
Tag der Abgabe: 23.01.2013
}
\clearpage
