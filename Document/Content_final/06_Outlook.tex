\chapter{Conclusion and further research}
\label{chp:outlook}

Within the presented thesis several models have been implemented, which are suitable to be used for process
optimization. The most elaborate model concerns distillation columns. Both steady-state and dynamic
models have been implemented. The core of these models is formed by the so called MESH equations which
denote the energy and material balances along with the governing equations for the vapour liquid equilibrium
and a closure condition for the mole fractions.

While these equations are capable of predicting steady-state operations for a given column, there are further
aspects of great interest to the process designer. For the dynamic models the column dynamics, in form of
hydraulic equations, have to be considered in order to attain a closed model, whereas for the steady state case
these can be replaced by respective specifications. Hence these hydraulic equations are implemented as an elective
module for these column models. To extend the capabilities of these models other modules have also been added which
are elective for both dynamic and steady-state models. Among those are Murphee tray efficiencies, which account for
non-equilibrium stages and costing models which yield approximations for an expected size and cost for the
simulated columns.

When it comes to simulating unit operations with these models the question of model initialization becomes relevant.
Due to the highly non-linear nature of the equation system, the solver needs to be provided with a good initial
guess to start convergence to a feasible solution. Complex column configurations with multiple feeds and side draws
as well as highly non-ideal mixtures further complicate the matter. To approach this problem the model environment
\gproms offers initialization procedures which allow to solve a sequence of model variants with increasing complexity.
Several approaches suitable for different systems have been implemented.

Process optimization has been the driving force when implementing these models. Especially structural decisions
associated with column design such as number of separation stages or feed and side draw locations.
The resulting mixed-integer non-linear programs are particularly hard to solve. Convergence is is highly
dependent on model robustness and problem formulation. As discussed in \secref{sec:opt:MINLP} the solution of
these programs can be approached in several ways. To allow for the the greatest flexibility the model superstructures
can be adjusted to have variable top or bottom reflux locations. Furthermore a so called differentiable distribution
function can be incorporated to facilitate convergence to a solution.

From the perspective of an industrial end-user of these models, a drag and drop solution which can easily be deployed
and yields reliable results is an ideal solution. An effort was made within this thesis to ascertain whether
if and how such a drag and drop solution might work.

HX integration model have been developed