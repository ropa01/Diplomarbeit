    An important factor in every given project is the total cost. During the design of a chemical process many
    important aspects of the future cost structure are not known a priori, as the final design is still under development.
    In general the total cost of implementing and operating a production site can be broken down into
    several subcategories.
    \regitemize{%
    	\item{Battery limit investment}
    	\item{Utility investment}
    	\item{Off-Site investment}
    	\item{Engineering fees}
    	\item{Working capital}
    }%
    The subsequent sections will present an overview as to how these different costs can be approximated
    depending on the level of detail of the information provided.

        \subsection{Before process design}
        \label{sec:before}
        Before any details about the process to be implemented are known, an estimate can only give an
        order of magnitude towards cost to be expected. The cost of the new process $C_P$ can be
        related to the cost of a reference process $C_{P}^0$ by
        \eq{
        	C_P = C_P^0 \cdot \left( \frac{Q_P}{Q_P^0} \right)^D,
        	\label{eq:cost}
        }%
        \ncr{C_P}{total process cost}{\$}
        \ncr{C_P^0}{reference process cost}{\$}
        \ncr{Q_P}{process capacity}{\frac{kg}{h}}
        \ncr{Q_P^0}{reference process capacity}{\frac{kg}{h}}
        \ncr{D}{degression coefficient}{-}
        where the quantity $Q$ refers to a specific process quality. In most cases a production rate or total
        annual capacity will be used. The degression coefficient $D$ needs to be correlated from historical data.

        As the overall price structure will change over time, the reference price will not reflect the current
        market situation. In order to adjust for that shortcoming several price indices are published
        all over the word. Some of those tailor to special branches of the industry, others give a picture
        of the price-development in an economy as a whole. The ratio of prices levels at different times
        is then assumed to be equal to the ratio of the price indices at the respective times
        \eq{
        	\frac{C_1}{C_2} = \frac{I_1}{I_2}.
        	\label{eq:index_ratio}
        }%
        \ncr{C_1}{Reference equipment cost at time $1$}{\$}
        \ncr{C_2}{Reference equipment cost at time $2$}{\$}
        \ncr{I_1}{Cost index at time $1$}{-}
        \ncr{I_2}{Cost index at time $2$}{-}
        For each index an somewhat arbitrary reference year is chosen. Among the most common indices are the
        Marshall \& Swift index, the Nellson-Farrar-Index or the Chemical Engineering index. Some exemplary
        values for these indices are given in \tabref{tab:PriceIndices}. As one can see the development within the
        process industry very well matches the development in the economy as a whole, which is why for this rough
        estimate general indices should suffice.

        \begin{table}
        	\center
        	\footnotesize
\rowcolors{2}{}{lightblue}
\begin{tabular}{cC{0.2\textwidth}C{0.2\textwidth}C{0.2\textwidth}C{0.2\textwidth}}
	 & \multicolumn{2}{c}{\begin{minipage}{0.42\textwidth} \center \footnotesize Marshall \& Swift�Installed Equipment Index \end{minipage}}
		& \begin{minipage}{0.20\textwidth} \center \footnotesize Nelson-Farrar Refinery Construction Index \end{minipage}
		& \begin{minipage}{0.20\textwidth} \center \footnotesize Chemical Engineering Plant Cost Index \end{minipage}\\
\rowcolor{white}	 & \multicolumn{2}{c}{\begin{minipage}{0.42\textwidth} \center \footnotesize $1926 \equiv 100$ \end{minipage}}
		& \begin{minipage}{0.20\textwidth} \center \footnotesize $1946 \equiv 100$ \end{minipage}
		& \begin{minipage}{0.20\textwidth} \center \footnotesize $1957 \equiv 100$ \end{minipage}\\ \hline
	 year &  all industries &  process industry & & \\ \hline
	 1975 & 444 & 452 & 576 & 182 \\
	 1980 & 560 & 675 & 823 & 261 \\
	 1985 & 790 & 813 & 1074 & 325 \\
	 1990 & 915 & 935 & 1226 & 358 \\
	 1995 & 1027 & 1037 & 1392 & 381 \\
	 2000 & 1089 & 1103 & 1542 & 394 \\
	 2001 & 1093 & 1107 & 1565 & 396 \\ \hline
\end{tabular}
\normalsize

        	\caption{Price indices and their development.\cite{Coulson.1999}}
        	\label{tab:PriceIndices}
        \end{table}

    \subsection{During process design}
        Once the future design has been broken down to fewer potential options and first process
        flowsheets are available, a more elaborate approach, resulting in much improved estimates, becomes possible.
        In contrast to the most general case, now the aforementioned investment areas can be distinguished.
        The following sections describe how estimates are attained for the respective investment category.

    \subsubsection{Battery limit investment}
        The battery limit investment denotes all investments necessary to have all required equipment
        for process operations installed on-site. This includes structures necessary to house the process as well as
        delivery and installation of all individual assets.  One major part of these cost will the the process
        equipment. As the exact manufacturers and models of the equipment will not be known in early design
        stages, an approximation similar to the one in \secref{sec:before} still needs to be employed. In contrast to
        before now the cost for individual pieces of process equipment will be considered explicitly.
        Each piece of equipment will have an specific feature which most heavily affects its cost. For vessels and
        reactors this might be volume, while for heat exchangers the required heat exchange area determines size and
        price of the unit.
        Accordingly, the price for an piece of equipment  $i$ $C^{i}_E$ can again be approximated by a simple power law
        \eq{
        	C^{i}_E = C^{i}_B \left( \frac{Q^{i}_E}{Q^{i}_B} \right)^{M^{i}}.
        	\label{eq:equip_cost}
        	\ncr{C^{i}_E}{cost of equipment $i$}{\$}
        	\ncr{C^{i}_B}{reference cost of equipment $i$}{\$}
        	\ncr{Q^{i}}{specific quantity for equipment $i$}{variable}
        	\ncr{Q^{i}_B}{equipment specific reference quantity}{variable}
        	\ncr{M^{i}}{equipment specific factor \nomunit}{-}
        }%
        A reference price $C^{i}_B$ is multiplied by the determining quantity $Q^{i}$ normalized to a reference state
        $Q^{i}_B$ and raised to the power $M^{i}$ specific to each piece of equipment. Reference prices and
        quantities for various installations can be obtained from literature (e.g. \cite{Seider.2010}).

        If more detailed information on the process conditions are available, they should also find their way into the cost estimation.
        Aside from the mere size of the equipment the process conditions will also influence the expected (and actual)
        cost. The predominant factors to that respect are pressure, temperature, corrosiveness
        and reactive activity, which will require single units to be manufactured from more resistant materials. Knowledge
        of a specific category for a piece of equipment to be installed will also lead to refined estimates. For example an plate-fin heat
        exchanger might be more expensive than a tubular model with the same heat-exchange area.
        In order to account for all those effect a form factor $f_F$ can be applied to the equipment cost
        \eq{
        	C^{i}_E = C^{i}_B \left( \frac{Q^{i}_E}{Q^{i}_B} \right)^{M^{i}} f_F^{i}
        		=   C^{i}_B \left( \frac{Q^{i}_E}{Q^{i}_B} \right)^{M^{i}}
                    \underbrace{\left(1 + f^{i}_C + f^{i}_M + f^{i}_P + f^{i}_T\right)}_{= f_F^{i}}.
        	\label{eq:equip_cost_mpt}
        	\ncr{f^{i}_C}{design complexity correction to equipment cost}{-}
        	\ncr{f^{i}_M}{material selection correction to equipment cost}{-}
        	\ncr{f^{i}_P}{pressure correction to equipment cost}{-}
        	\ncr{f^{i}_T}{temperature correction to equipment cost}{-}
        }%
        Where $f_F$ denotes the form factor, $f_C$ corrects for design complexity, $f_M$ for material selection,
        $f_P$ adjusts for extreme pressures and $f_T$ for temperature.

        As in the previous section all reference prices need to be scaled for the temporal price
        development. Again the aforementioned indices are used. Furthermore can the prices be corrected for regional
        differences in price structure. Once more correction factors to prices in an reference region in
        the world (e.g. USA) are employed \cite{Peters.2003}.

        In addition to the purchased cost, the costs for installing the process equipment have an significant effect
        on the total needed investment of the process. These installation costs include:
        \regitemize{%
        	\item Installations costs
        	\item{Piping ,valves and electrical wiring}
        	\item{Control system}
        	\item{Structures and foundations}
        	\item{Insulation and fire proofing}
        	\item{Labour fees}
        }%
        To incorporate those additional costs, again correction factors to the main equipment price are used.
        Depending on the status
        of the information available they can be expressed as one unified factor or broken down to each specific
        category. One however needs to bear in mind, that costs for piping and valves -- pieces of equipment
        in direct contact with process media -- will be affected by the process conditions in a similar
        way as the actual equipment, whereas the other categories are more likely to remain unchanged. Thus
        attention needs to be paid, in which fashion the factors will be applied.

        A word should be said about the cost for the control system. Most obtainable data will most likely refer to
        a decentralized control system, as it has been in use for many years. With ever more powerful computers
        a centralized approach, namely model predictive control (MPC) is becoming more relevant. As the
        structure for such a control system may vary significantly from the common designs, the cost factors may
        as well.

    \subsubsection{Services}
        The utility investments and off-site investments are often referred to as services. Therein included are
        all measures necessary to supply the process with the media consumed during operation. This includes but is
        not necessarily limited to generation and distribution of energy, steam and process gases. The utility investments
        in this context refer to all investments within the greater production site but out of the battery limits of the
        specific process. Off-site investments contain everything not contained in the site such as roads, power cables,
        communication systems or waste disposal. All these costs are expressed as fractions of the equipment
        cost at moderate temperature and pressure. This means, when applying these fractions, the factors $f_M$,
        $f_P$ and $f_T$ should not be considered at this point.

        Once again in early design stages one has to resort to factors derived for statistical data, to calculate
        the cost of raw materials, energy and support media such as lubricants, heat or catalysts. If more
        detailed information on process streams is available the approach should be refined.

        The cost of raw materials $C_{RM}$ can then be calculated if the material streams of individual raw materials
        $m_i$ as well as their specific cost are known.
        \eq{
        	C_{RM} = \left( \sum_i^{N_{RM}} \dot{m}_i \cdot C_{RM}^{i} \right) \cdot t_{op}.
        	\ncr{n_{RM}}{number of different raw materials}{-}
        	\ncr{C_{RM}}{total cost of raw materials}{\$}
        	\ncr{C^{i}_{RM}}{mass specific cost of raw material $i$}{\frac{\$}{kg}}
        	\ncr{\dot{m}_i}{mass flow of component $i$}{\frac{kg}{s}}
        	\ncr{t_{op}}{time of process operations}{s}
        }%

        Much in the same way the cost for energy can be approximated. As for the raw materials, this is once
        more done for each individual process unit, rather then for the process as a whole. Hence
        with the needed energy for equipment $i$ with respect to energy carrier $j$ $e_{ij}$ along with
        the price for energy carrier $j$ $C_{EC}^j$ the total energy cost $C_{EC}$ can be assessed
        \eq{
        	C_{EC} = \left( \sum_i^{N_{E}} \sum_j^{N_{EC}} \dot{e}_{ij} \cdot C_{EC}^{i} \right) \cdot t_{op}.
        	\ncr{n_{E}}{number of process units}{-}
        	\ncr{C_{EC}}{total cost of energy}{\$}
        	\ncr{C^{i}_{EC}}{mass specific cost of energy carrier $i$}{\frac{\$}{kW}}
        	\ncr{\dot{e}_{ij}}{energy flow of to equipment $i$ from energy carrier $j$}{kW}
        }%
        All cost related to utilities consumed during the operation of the process and not any eventual
        down time are calculated using the operating time $t_{op}$

    \subsubsection{Working capital}
        The working capital includes all investments necessary for process operations. This means raw
        materials, payroll, extended credit to customers and so on. In contrast to all other costs the
        working capital can partially be retrieved when the process ceases operations. How different
        types of cash flows, extended or owed credit should be handled will be disused in
        \secref{sec:InvestmentCriteria}. In addition to the cost of raw materials needed during process
        operations, which generate a product stream, raw materials are also needed to fill all vessels, reactors,
        columns and piping that make up the process. The cost of these should be considered as investment
        and not as operation cost \cite{Coulson.1999}, since more or less the same amount will remain bound until
        the process ceases operations and it can (partially) be retrieved.
        \eq{
        	C_{FILL} = \sum_i^{N_{RM}} V_i \cdot C_{RM}^{i}.
        	\ncr{C_{FILL}}{cost of raw materials to fill the process}{\$}
        }%

    \subsubsection{Total investment}
        When all contributions to the total investment are considered an estimate for the total price of the
        process can be calculated
        \eq{
        	C_P = \sum_i^{N_E} \left[ C_{B}^{i} \left( \frac{Q^{i}}{Q^{i}_{B}} \right)^{M^{i}}
        		\left(f_F \cdot f_{PIPE} + \sum_j f_j \right)\right] + C_{RM} + C_{EC} + C_{FILL}
        }%
        It should be emphasized that in early design stages these calculations will at best yield an order of
        magnitude estimate for the expected cost of implementing a chemical process. Most literature sources
        give an accuracy of $\pm 30 \%$  \cite{Peters.2003}. As the project progresses more and more information
        becomes available an a more accurate estimate can be prepared. The most refined ones rely on actual
        proposals from prospective manufacturers and suppliers.